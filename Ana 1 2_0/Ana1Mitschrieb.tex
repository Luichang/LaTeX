\documentclass[12pt,a4paper]{article} % using article ensures it starts at 1 and does not have odd numberings for section
\usepackage{graphicx}
\usepackage[utf8]{inputenc} % this way umlaute are included from the get go
\usepackage[ngerman]{babel} % german spell check
\usepackage{datetime}

\usepackage{amsmath} % this package is one option for math lines
\usepackage{amssymb} % package for Natural Number sign etc
\usepackage{tasks}

\usepackage{hyperref} % these two lines are so that the table of content is clickable
\setcounter{tocdepth}{3}
\hypersetup{linktoc=all}

\newcounter{Definition}[section]
\newcounter{Lemma}[section]
\newcounter{Korollar}[section]
\newcounter{Satz}[section]

\newcommand{\Definition}[1][]{
	\stepcounter{Definition}
	\subsubsection*{Definition \thesection.\theDefinition: #1}
	\addcontentsline{toc}{subsubsection}{\protect\numberline{}Definition \thesection.\theDefinition: #1}
}

\newcommand{\Lemma}[1][]{
	\stepcounter{Lemma}
	\subsubsection*{Lemma \thesection.\theLemma: #1}
	\addcontentsline{toc}{subsubsection}{\protect\numberline{}Lemma \thesection.\theLemma: #1}
}

\newcommand{\Korollar}[1][]{
	\stepcounter{Korollar}
	\subsubsection*{Korollar \thesection.\theKorollar: #1}
	\addcontentsline{toc}{subsubsection}{\protect\numberline{}Korollar \thesection.\theKorollar: #1}
}

\newcommand{\Satz}[1][]{
	\stepcounter{Satz}
	\subsubsection*{Satz \thesection.\theSatz: #1}
	\addcontentsline{toc}{subsubsection}{\protect\numberline{}Satz \thesection.\theSatz: #1}
}

\newdateformat{gerDate}{\THEDAY \space \monthname[\THEMONTH], \THEYEAR} % german date format

\begin{document}
	\begin{titlepage} % good tital page template, only needs the class X notes part to be changed for each new class
		\centering
		\includegraphics[width=0.40\textwidth]{UniLogo}\par\vspace{1cm}
		{\scshape\LARGE Universität Heidelberg \par}
		\vspace{1cm}
		{\Huge\bfseries Analysis 1 \par}
		\vspace{1cm}
		{\LARGE\bfseries Prof. Dr. Peter Albers \par}
		\vspace{1cm}
		{\huge Wintersemester 17/18 \par}
		\vspace{2cm}
		{\Large\itshape by Charles Barbret \par}
		
		\vfill
		
		% Bottom of the page
		{\large \gerDate\today\par}
	\end{titlepage}
	
	\tableofcontents % creats a table of contents, ensured already that it is clickable
	\newpage % starts the actual document on a new page so there is no weird colision of text and toc
	\setcounter{section}{-1}
	\section{Ablauf der Vorlesung}
	\subsection{Moodle}
	Modle passwort: Ableitung
	\subsection{Übungsbetrieb}
	Donnerstags kommen die neuen Zettel \newline
	Zettel sollen in Zweiergruppen abgegeben werden \newline
	Abgabeschluss ist Donnerstags 09:15 im Mathematikon bei den Briefkästen \newline
	Das erste Blatt wird spätestenz am 19.10.2017 veröffentlicht 
	
	\subsection{Plenarübung}
	Donnerstags 16:00 bis 18:00 im KIP HS1 \newline
	Erste Plenarübung findet bereits am 19.10.2017
	
	\subsection{Klausur}
	50\% der Gesamtpunkte der Aufgabenblätter und einmal vorgerechnet haben.
	
	Klausurtermine: 23.02.2018 09:00 bis 13:00 Uhr und 9.4.2018 09:00 bis 13:00 Uhr
	
	Nicht erscheinen bei der ersten Klausur ohne Abmeldung gilt als 5.0, man kann dann aber immer noch in die Nachklausur.
	
	\section{Grundlagen}
	\subsection{Mengen und Aussagen}
	\Definition
	Eine \underline{Menge} ist eine Zusammenfassung von wohldefinierten und wohlunterschiedenen Objekten zu einem Ganzen. Die Objekte heißen \underline{Elemente} der Menge. \newline
	\underline{wohlbestimmt}: Von jedem Objekt steht fest, ob es Element der Menge ist oder nicht \newline
	\underline{wohl unterschieden}: Jedes Objekt kommt höchstens einmal in der Menge vor \newline
	\begin{tabular}{l l}
		Beschreibung der Menge & a) Durch Aufzählung $\mathbb{N}: = \{1, 2, 3, ...\} $\\
		& \parbox[t]{8cm}{b) Durch Angabe einer charakteristischen Eigenschaft in Form einer Aussage, d.h. eines Satzes, von dem eindeutig feststeht, ob er wahr oder falsch ist. Der Wahrheitsgehalt muss zum gegebenen Zeitpunkt nicht bekannt sein.
			Bsp A(u):= u ist eine Primzahl (D.h. $u \ge 2$)} \\
		& c) Durch Beschreibung der Elemente 
	\end{tabular}
	\Definition
	Es sein M und N Mengen
	
	[\ldots] 
	
	\underline{Bemerkung}
	"oder" bedeutet das einschließliche oder, also nicht "{}entweder \dots oder \dots" $\rightarrow$ Wahrheitstabellen \newline
	Seien A und B Aussagen. Wir leiten ab \footnote{w = wahr, f = falsch}
	
	\begin{tabular}{c | c | c | c | c}
		A & B & A und B & A oder B & "{}Entweder A oder B" \\ \hline
		w & w & w & w & f \\
		w & f & f & w & w \\
		f & w & f & w & w \\
		f & f & f & f & f
	\end{tabular}
	
	\underline{Implikation} zwischen Aussagen
	
	\begin{tabular}{l l}
		A $\Rightarrow$ B & steht für: A impliziert B \\
		& Falls A gilt, dann gilt auch B \\
		& A ist \underline{hinreichende} Bedingung für B \\
		& B ist \underline{notwendige} Bedingung für A
	\end{tabular}
	
	Formal definieren wir A $\Rightarrow$ B ist wahr, falls $\neg$ A oder B
	
	D.h.
	\begin{tabular}{c | c | c}
		A & B & A $\Rightarrow$ B \\ \hline
		w & w & w \\
		w & f & f \\
		f & w & w \\
		f & f & w
	\end{tabular}
	
	Außerdem kürzen wir ab:
	
	A $\Leftrightarrow$ B steht für A $\Rightarrow$ B $\land$ A $\Leftarrow$ B
	
	Beispiel:
	
	[\dots]
	
	\underline{Bemerkung}:\newline
	Für alle Mengen M gilt $\emptyset \subset M$ \newline
	$\forall M: \emptyset \subset M$.
	
	\newpage
	\section{Reelle Zahlen übersprungen}
	\newpage
	\section{Folgen}
	\Definition[Folge komplexer Zahlen]
	Eine Folge komplexer Zahlen ist eine Abbildung 
	
	\begin{align*}
	a: & \mathbb{N} \rightarrow \mathbb{C} \\
	& n \rightarrow a(n) = a_n
	\end{align*}
	
	\underline{Notation} \(a = (a_n)_{n \in \mathbb{N}}\).
	
	Analog: \(a_{n_0}, a_{n_0 + 1}, a_{n_0 + 2}, ... = (a_n)_{n \ge n_0}\)
	
	Heißt soviel wie, da die Folge von einem Term in Abhängigkeit von n steht, dass man eine Bijektion mit den Natürlichen Zahlen bilden kann.
	
	\subsection{Konvergent und Grenzwert}
	\Definition[Konvergente Folgen]
	\label{def:Konvergenz}
	\begin{enumerate}
		\item Eine Folge \(a_n)_{n \in \mathbb{N}}\) heißt konvergent, falls $a \in \mathbb{C}$ mit folgender Eigenschaft existiert:
		\[\forall \epsilon > 0 \exists n_0 = n_0(\epsilon) \in \mathbb{N} \forall n \ge n_0: |a_n - a| < \epsilon\]
		Dann heißt a \underline{Grenzwert} oder \underline{Limes} von $(a_n)$.
		
		\begin{tabular}{l l}
			Schreibweise: & $a_n \rightarrow a, a_n \xrightarrow{n \rightarrow \infty} a$ \\
			& $\lim a_n = a, \lim\limits_{n \rightarrow \infty} a_n = a$ \\
		\end{tabular}
		\item Hat eine Folge keinen Grenzwert, heißt sie \underline{divergent}.
		\item Gilt $a_n \rightarrow 0$, so heißt $(a_n)$ \underline{Nullfolge}.
	\end{enumerate}
	
	\underline{WICHTIG} Die Aussagen "für alle bis auf endlich viele Folgenglieder" und "für unendlich viele Folgenglieder" sind nicht äquivalent.
	
	\Lemma
	\label{lem:KonvergentBeschrnkt}
	\begin{tasks}
		\task Der Grenzwert einer konvergenten Folge ist eindeutig bestimmt.
		\task Jede konvergente Folge ist beschränkt: $\exists s \in \mathbb{R} \forall n \in \mathbb{N}: |a_n| \le s$
	\end{tasks}
	
	
	\Satz[Regeln für n-te Wurzel]
	\label{satz:nthRoot}
	\begin{tasks}
		\task Für $a \in \mathbb{R}^+$ gilt: $\lim\limits_{n \rightarrow \infty} \sqrt[n]{a} = 1$.
		\task $\lim\limits_{n \rightarrow \infty} \sqrt[n]{n} = 1$
	\end{tasks}
	
	\subsection{Rechenregeln für Grenzwerte}
	\Lemma
	Es seien $(a_n)$ und $(b_n)$ \underline{\hyperref[def:Konvergenz]{konvergente}} Folgen mit $a_n \rightarrow a$ und $b_n \rightarrow b$. Dann sind $(\lambda a_n)_{n \in \mathbb{N}}, \lambda \in \mathbb{C}, (a_n + b_n)_{n \in \mathbb{N}}, (a_n * b_n)_{n \in \mathbb{N}}$ und, falls $b \ne 0: (\frac{a_n}{b_n})_{n \in \mathbb{N}}$ konvergent.
	
	Heißt soviel wie, sollten die Folgen jeweils konvergieren, kann man die Grenzwerte miteinander verrechnen und skalieren.
	
	\Lemma
	Es seien $(a_n)$ und $(b_n)$ konvergente Folgen. Es gelte $a_n \le b_n$ für unendlich viele $n \in \mathbb{N}$ \newline
	Dann gilt: $\lim a_n \le \lim b_n$.
	
	Hiermit können wir Folgen abschätzen, was manchmal ganz hilfreich sein kann.
	
	\Lemma[Sandwich-Lemma]
	\label{lem:Sandwich}
	Es seien $(a_n), (b_n), (c_n)$ reelle Folgen mit:
	\begin{tasks}[counter-format=(tsk[r]), label-width=4ex]
		\task $(a_n)$ und $(c_n)$ sind konvergent mit: $\lim a_n = \lim c_n$
		\task Für alle bis auf endlich viele $n \in \mathbb{N}$ gilt: $a_n \le b_n \le c_n$
	\end{tasks}
	Dann ist $(b_n)$ konvergent mit: $\lim a_n = \lim b_n = \lim c_n$ \newline
	Wenn wir eine Minorante und Majorante finden können, mit dem gleichen Grenzwert, so konvergiert die Folge gegen den selben Grenzwert
	
	\subsection{Häufungspunkt, Cauchy-Folge, Teilfolgen}
	\Definition
	Eine Folge $(a_n)_{n \in \mathbb{N}}$ reeller Zahlen heißt \underline{\hyperref[def:Monoton]{monoton wachsend}}, falls gilt:
	\[\forall n \in \mathbb{N}: a_{n + 1} \ge a_n\]
	Monoton Fallend analog.
	
	\Satz[Monotoniekriterium]
	Jede beschränkte \hyperref[def:Monoton]{monotone} Folge $(a_n)_{n \in \mathbb{N}}$ ist konvergent mit
	
	$\lim\limits_{n \rightarrow \infty} a_n = \left\lbrace 
	\begin{array}{l l}
	\sup \{a_n | n \in \mathbb{N}\} & \text{falls monoton wachsend} \\
	\inf \{a_n | n \in \mathbb{N}\} & \text{sonst}
	\end{array}
	\right.$ \newline
	Monoton + Beschränkt = Konvergent
	
	\Definition[Häufungspunkte]
	\label{def:Haufungspunkte}
	Sei $(a_n)$ eine Folge. Dann heißt $a \in \mathbb{C}$ Häufungspunkt der Folge, falls gilt: \newline
	$\forall \epsilon > 0$ gilt $|a_n - a| < \epsilon$ \newline für unterschiedlich viele $n \in \mathbb{N}$ \newline
	Heißt soviel wie: wenn \hyperref[def:Teilfolge]{Teilfolgen} konvergieren, sind die Grenzwerte der Teilfolgen die Häufungspunkte der Folge.
	
	\Definition[Teilfolge]
	\label{def:Teilfolge}
	Sei $(a_n)_{n \in \mathbb{N}}$ eine Folge komplexer Zahlen und $(n_l)_{l \in \mathbb{N}}$ eine Folge natürlicher Zahlen mit \(\forall l \in \mathbb{N}: n_{l + 1} > n_l\). Dann heißt $(a_{n_l})_{l \in \mathbb{N}}$ eine Teilfolge von $(a_n)_{n \in \mathbb{N}}$
	
	\Lemma[Häufungspunkte]
	$a \in \mathbb{C}$ ist genau dann ein Häufungspunkt der Folge $(a_n)_{n \in \mathbb{N}}$, wenn es eine konvergente Teilfolge $(a_{n_l})_{l \in \mathbb{N}}$ gibt mit $\lim\limits_{n \rightarrow \infty} a_{n_l} = a$.
	
	\hyperref[def:Haufungspunkte]{Also genau was ich gemeint hatte.}
	
	\Satz[Q dicht in R]
	$\mathbb{Q}$ liegt dicht in $\mathbb{R}$
	
	\Lemma
	Jede Folge reeller Zahlen besitzt eine \hyperref[def:Monoton]{monotone} \hyperref[def:Teilfolge]{Teilfolge}.
	
	Heißt, bei jeder Folge reeller Zahlen kann man sich vereinzelnd Elemente raus suchen, welche eine \hyperref[def:Monoton]{monotone} Teilfolge bilden.
	
	\Satz[Bolzano-Weierstraß für $\mathbb{R}$]
	Jede beschränkte Folge reeller Zahlen besitzt eine konvergente \hyperref[def:Teilfolge]{Teilfolge}
	
	Gilt natürlich auch in komplexen Zahlen.
	
	\Definition[Cauchy-Folge]
	\label{def:Cauchy}
	Eine Folge $(a_n)_{n \in \mathbb{N}}$ komplexer Zahlen heißt Cauchy-Folge, falls gilt:
	\[\forall \epsilon > 0 \exists n_0 = n_0(\epsilon) \in \mathbb{N} \forall m, n \ge n_0: |a_m - a_n| < \epsilon \]
	
	\Satz[Cauchy Kriterium]
	Die Folge $(a_n)$ ist genau dann konvergent, wenn $(a_n)$ eine Cauchy-Folge ist.\newline
	\underline{Bemerkung} Es gibt Cauchy-Folgen $(a_n)$ mit $a_n \in \mathbb{Q} \forall n \in \mathbb{N}$ die nicht gegen eine rationale Zahl konvergieren.
	
	Als Beispiel eine Folge die gegen e konvergiert, da $e \notin \mathbb{Q}$
	
	\subsection{Limes superior und Limes inferior}
	Es sei $(a_n)_{n \in \mathbb{N}}$ eine beschränkte Folge, d.h. $\exists s \in \mathbb{R} \forall n \in \mathbb{R}: |a_n| \le s$.
	\begin{tabular}{l l}
		\underline{Beobachtung} Sei & $B_n := \sup\{a_k | k \ge n\}$ \\
		& $b_n := \inf\{a_k | k \ge n\}$
	\end{tabular}
	
	Dann ist die Folge $(B_n)_{n \in \mathbb{N}}$ \hyperref[def:Monoton]{monoton fallend} und $(b_n)_{n \in \mathbb{N}}$ \hyperref[def:Monoton]{monoton wachsend}.
	
	Außerdem gilt: $\forall n \in \mathbb{N}: b_n \le B_n; |b_n|, |B_n| \le s$
	
	Also konvergieren $(b_n)$ und $(B_N)$ mit \(\lim\limits_{n \rightarrow \infty} b_n \le \lim\limits_{n \rightarrow \infty}limits_{n \rightarrow \infty} B_n\)
	
	\Definition[lim sup und lim inf]
	Wir definieren:
	\[\limsup\limits_{n \rightarrow \infty} a_n:= \overline{\lim\limits_{n \rightarrow \infty}} a_n:= \lim\limits_{n \rightarrow \infty} B_n \text{ \hyperref[def:Konvergenz]{limes superior}}\]
	\[\liminf\limits_{n \rightarrow \infty} a_n:= \underline{\lim\limits_{n \rightarrow \infty}} a_n:= \lim\limits_{n \rightarrow \infty} b_n \text{ limes inferior}\]
	
	\Definition
	Falls es eine Teilmenge $A \subset \mathbb{R}$ nicht nach oben beschränkt ist, definieren wir $\sup A:= +\infty$.
	
	Falls es eine Teilmenge $A \subset \mathbb{R}$ nicht nach unten beschränkt ist, definieren wir $\inf A:= -\infty$.
	
	Außerdem setzen wie $\sup \emptyset := -\infty, \inf \emptyset := +\infty$.
	
	Wir setzen: $\forall x \in \mathbb{R}: -\infty < x < +\infty \equiv \infty$.
	
	\Definition[Bestimmt Divergent]
	\label{def:BesDivergent}
	Eine Folge $(a_n)$ reeller Zahlen heißt bestimmt divergent gegen $+\infty$, falls gilt: \(\forall K \in \mathbb{R} \exists n_0 = n_0(K) \in \mathbb{N} \forall n \ge n_0: a_n \ge K\)
	
	Eine Folge $(a_n)$ heißt bestimmt divergent gegen $-\infty$, falls $(-a_n)$ bestimmt divergent gegen $+\infty$ ist.
	
	\Satz
	Die Folge $(a_n)$ sei \hyperref[def:BesDivergent]{bestimmt divergent} gegen $\pm \infty$ Dann gilt:
	\begin{tasks}[counter-format=(tsk[r]), label-width=4ex]
		\task $a_n \ne 0$ für alle bis auf endlich viele $n \in \mathbb{N}$
		\task $\lim\limits_{n \rightarrow \infty} \frac{1}{a_n} = 0$
	\end{tasks}
	
	\Satz
	Es sei $(a_n)_{n \in \mathbb{N}}$ eine Nullfolge mit $a_n > 0$ für alle bis auf endlich viele $n \in \mathbb{N}$. Dann divergiert die Folge $(\frac{1}{a_n})$ bestimmt gegen $+\infty$. 
	
	Analog für $a_n < 0$ gegen $-\infty$
	
	\Satz[Charakterisierung von limsup und liminf]
	Es sei $(a_n)$ eine Folge reeller Zahlen 
	
	\begin{enumerate}
		\item Es gilt $\limsup a_n = a \in \mathbb{R}$ genau dann, wenn für alle $\epsilon > 0$ gilt
		\begin{tasks}[counter-format=(tsk[r]), label-width=4ex]
			\task $a_n < a + \epsilon$ für alle bis auf endlich viele $n \in \mathbb{N}$
			\task $a_n > a - \epsilon$ für unendlich viele $n \in \mathbb{N}$
		\end{tasks} 
		Heißt das Supremum $a_n$ ist nur dann kleiner als der Grenzwert, wenn man etwas unendlich kleines an den Grenzwert addiert. Es ist größer als der Grenzwert sobald man etwas vom Grenzwert abzieht.
		\item Es gilt $\liminf a_n = a \in \mathbb{R}$ genau dann, wenn für alle $\epsilon < 0$ gilt:
		\begin{tasks}[counter-format=(tsk[r]), label-width=4ex]
			\task $a_n > a - \epsilon$ für alle bis auf endlich viele $n \in \mathbb{N}$
			\task $a_n < a + \epsilon$ für unendlich viele $n \in \mathbb{N}$
		\end{tasks} 
		Ähnliche Erklärung wie oben auch.
		\item $(a_n)$ ist genau dann Konvergent, wenn $\liminf\limits_{n \rightarrow \infty} a_n = \limsup\limits_{n \rightarrow \infty} a_n \in \mathbb{R}$. In diesem Fall gilt $\lim\limits_{n \rightarrow \infty} a_n = \liminf\limits_{n \rightarrow \infty} a_n = \limsup\limits_{n \rightarrow \infty} a_n$
	\end{enumerate}
	\newpage
	\section{Stetige Funktionen und die Topologie des \(\mathbb{R}^n\)}
	\Definition
	Es sei $D \subset \mathbb{R}^n$ eine nichtleere Teilmenge. Eine reelwertige bzw komplexwertige Funktion auf D ist eine Abbildung $f: D \rightarrow \mathbb{R}$ bzw $f: D \rightarrow \mathbb{C}$ 
	
	Für zwei Funktionen $f, g: D \rightarrow \mathbb{R}$ (oder $\mathbb{C}$) definieren wir 
	
	\begin{tabular}{l l}
		(f + g)(x) := f(x) + g(x) & $\forall x \in D$ \\
		(f - g)(x) := f(x) - g(x) & $\forall x \in D$ \\
		(f * g)(x) := f(x) * g(x) & $\forall x \in D$ \\
	\end{tabular}
	
	und falls gilt: $g(x) \ne 0 \forall x \in D$, $\frac{f}{g}(x) := \frac{f(x)}{g(x)}$
	
	\Definition[Endlicher Abstand]
	Der endliche Abstand zweier Punkte $x = (x_1, ..., x_n)$ und $y = (y_1, ..., y_n)$ des $\mathbb{R}^n$ ist 
	\[|x - y| := \sqrt{\sum_{i = 1}^{n}(x_i - y_i)^2} = ((x_1 - y_1)^2 + (x_2 - y_2)^2 + ... + (x_n - x_n)^2)^{\frac{1}{2}}\]
	
	
	\subsection{Stetigkeit}
	\label{def:Stetig}
	\Definition[$\epsilon - \delta-$Kriterium]
	Eine Funktion $f: D \rightarrow \mathbb{R}$ (oder $\mathbb{C}$) heißt \underline{stetig in $x_0 \in D$}, falls gilt:
	\[\forall \epsilon > 0 \exists \delta > 0 \forall x_0 \in D: |x - x_0| < \delta \rightarrow |f(x) - f(x_0)| < \epsilon\] 
	f heißt \underline{stetig} in D, falls f in jedem Punkt von D stetig ist. 
	\newline{}
	
	Heißt soviel wie für jeden Abstand den 2 Folgeglieder [f(x) nach f(y)] von einander haben muss es einen Abstand zwischen x und y geben. Was nicht erfüllt werden kann, wenn die Funktion an $x_0$/einer Stelle springt.
	
	\Satz[Folgenkriterium]
	$f: D \rightarrow \mathbb{C}$ ist genau dann stetig in $x^* \in D$, wenn für jede Folge $(x_n)_{n \in \mathbb{N}}$ in D mit $\lim\limits_{n \rightarrow \infty} x_n = x^*$ gilt $\lim\limits_{n \rightarrow \infty} f(x_n) = f(x^*)$
	$\Leftrightarrow \lim\limits_{n \rightarrow \infty} f(x_n) = f(\lim\limits_{n \rightarrow \infty} x_n)$ 
	
	\Satz[Stetige Funktionen verknüpft sind Stetig]
	Es seien $f, g \in D \rightarrow \mathbb{C}$ stetig in $x^* \in D$ Dann sind $f \pm g, f * g$ und falls $g(x) \ne 0 \forall x \in D \frac{f}{g}$ stetig in $x^*$.
	
	\Satz[Komposition stetiger Funktionen sind Stetig]
	Es sei $f: D \rightarrow \mathbb{C}$ und $g: E \rightarrow \mathbb{C}$ mit $f(D) \subset E$ gegeben. Ist f stetig in $x^* \in D$ und g stetig in $y^* := f(x^*) \in E$, so ist auch $g \circ f$ stetig in $x^*$ 
	
	\begin{tabular}{l l l}
		$D$ & $\xrightarrow{f} f(D) \in E$ & $\xrightarrow{g} \mathbb{C}$ \\
		$x^*$ & $\rightarrow f(x^*) = y^*$ & $\rightarrow g(y^*)$
	\end{tabular}
	
	\Definition[Lipschitz-Stetig]
	Eine Funktion $f: D \rightarrow \mathbb{C}$ heißt Lipschitz-Stetig mit Lipschitz-Konstante $L \ge 0$, falls gilt: $\forall x, y \in D: |f(x) - f(y)| \le L|x - y|$
	
	\subsection{Offen und abgeschlossene Mengen}
	\label{def:Br}
	Sei $a \in \mathbb{R}^n$ und $r > 0$. Dann ist $B_r(a) := \{x \in\mathbb{R}^n: |x - a| \le r\}$ der (abgeschlossene) Ball um a mit Radius r.
	
	Um jede Menge im $\mathbb{R}^n$ kann man einen Kreis/Ball ziehen mit Radius r um alle Elemente der Menge M einzuschließen. 
	
	\Definition[Randpunkt von M]
	Es sei $M \subset \mathbb{R}^n$ eine Menge. Ein Punkt $p \in \mathbb{R}^n$ heißt Randpunkt von M, falls gilt:
	\[\forall r > 0: \hyperref[def:Br]{B_r}(p) \cap M \ne \emptyset \ne \hyperref[def:Br]{B_r}(p) \cap (\mathbb{R}^n \backslash  M)\]
	
	Ein Punkt p heißt Randpunkt, wenn man Kreise/Bälle um den Punkt ziehen kann, dass Teil des Kreises/Balls in der Menge liegt
	
	\Definition[Offene Mengen]
	\label{def:Offen}
	\begin{tasks}
		\task Eine Teilmenge $M \subset \mathbb{R}^n$ heißt offen, falls $M \cap \delta M = \emptyset$ gilt.
		\task M heißt abgeschlossen, falls $\delta M \subset M$ gilt.
	\end{tasks}
	Wobei $\delta M$ den Rand von M beschreibt $\Rightarrow$ Eine Menge ist offen, wenn M vereinigt mit dem Rand leer ist und abgeschlossen, falls der Rand eine Teilmenge von M ist
	
	\Lemma[Offen]
	$M \subset \mathbb{R}^n$ ist genau dann \hyperref[def:Offen]{offen}, wenn gilt:
	\[\forall p \in M \exists r > 0: \hyperref[def:Br]{B_r}(p) \subset M\]
	Heißt M ist offen, wenn für jeden Punkt aus der Menge einen Kreis/Ball ziehen kann mit einem Radius größer 0, der ganz in der Menge ist - Ginge bei abgeschlossenen Mengen nicht, da wenn man einen Punkt am Rand wählt gibt es kein r für den der ganze Kreis/Ball in der Menge ist
	
	\Korollar[Offene Mengen bestehen aus Offenen Bällen]
	Jede \hyperref[def:Offen]{offene} Menge $M \subset \mathbb{R}^n$ ist eine Vereinigung \hyperref[def:Offen]{offener} Bälle. 
	\[\forall p \in M \exists r_p > 0: \mathring{B}_{r_p}(p) \subset M\]
	\[\Rightarrow M = \bigcup\limits_{p \in M} \mathring{B}_{r_p}(p) =: RHS\]
	Heißt um jeden Punkt zieht man ein Kreis dessen Radius größer 0 ist, nimmt dann alle Elemente in den Kreisen und steckt sie in die Menge M
	
	\Lemma[Abgeschlossen]
	$M \subset \mathbb{R}^n$ ist genau dann abgeschlossen, wenn gilt: Für alle Folgen $(a_n)_{n \in \mathbb{N}}$ mit $a_n \in M \forall n \in \mathbb{N}; \lim\limits_{n \rightarrow \infty} a_n \rightarrow a$ gilt mit $a \in M$ \newline
	Heißt, die Menge ist abgeschlossen wenn es keine Folge gibt, dessen Grenzwert außerhalb der Menge liegt
	
	\Satz
	\begin{tasks}
		\task Beliebige Vereinigungen \hyperref[def:Offen]{offener} Mengen sind \hyperref[def:Offen]{offen}. \newline
		\begin{tabular}{l p{10cm}}
			\underline{Formulierung 1:} & Es sei $\mathcal{U} \subset P(\mathbb{R}^n)$ mit $\forall U \in \mathcal{U}: U$ ist \hyperref[def:Offen]{offen} \newline
			$\Rightarrow \bigcup \mathcal{U} := \{x \in \mathbb{R}^n | \exists U \in \mathcal{U}: x \in U\} \subset \mathbb{R}^n$ ist \hyperref[def:Offen]{offen} \\
			\underline{Formulierung 2:} & Es sei I eine (Index-)Menge $\forall i \in I$ sei $U_i \subset \mathbb{R}^n$ \hyperref[def:Offen]{offen} \newline
			$\Rightarrow \bigcup\limits_{i \in I} U_i \subset \mathbb{R}^n$ ist \hyperref[def:Offen]{offen}
		\end{tabular}
		\task Belibige Durchschnitte abgeschlossener Mengen sind abgeschlossen \newline
		\begin{tabular}{l p{10cm}}
			\underline{Formulierung 1:} & Es sei $\mathcal{A} \subset P(\mathbb{R}^n): \forall A \in \mathcal{A}: A$ ist abgeschlossen \newline
			$\Rightarrow \bigcap \mathcal{A} := \{x \in \mathbb{R}^n | \forall A \in \mathcal{A}: x \in A\}$ ist abgeschlossen \\
			\underline{Formulierung 2:} & Es sei I eine (Index-)Menge $\forall i \in I$ sei $A_i \subset \mathbb{R}^n$ abgeschlossen \newline
			$\Rightarrow \bigcap\limits_{i \in I} A_i \subset \mathbb{R}^n$ ist abgeschlossen.
		\end{tabular}
	\end{tasks}
	
	\Definition[Innen und Abschluss]
	Das Innere $\mathring{M}$ einer Menge $M \subset \mathbb{R}^n$ ist $\mathring{M} := M \backslash \delta M$.\newline
	Der Abschluss $\overline{M} \text{ ist } \overline{M} := M \cup \delta M$. \newline
	Heißt das Innere einer Menge M wird beschrieben als M ohne dem Rand. Der Abschluss einer Menge ist M vereinigt mit dem Rand
	
	\Lemma
	\begin{enumerate}
		\item $\mathring{M}$ ist die größte \hyperref[def:Offen]{offene} Teilmenge, die in M enthalten ist. 
		\[M \backslash \delta M = \mathring{M} = \bigcup \{U | U \subset M \text{ und U offen}\}\]
		Dadurch, dass $\mathring{M}$ die Menge M ohne dem Rand ist, ist es die größte offene Teilmenge von M
		\item $\overline{M}$ ist die kleinste abgeschlossene Teilmenge, die M enthält.
		\[M \cup \delta M = \overline{M} = \bigcap \{A | A \supset M \text{ und A abgeschlossen}\}\]
		Dadurch, dass $\overline{M}$ die Menge M mit dem Rand ist, ist es die kleinste abgeschlossene Teilmenge von M
	\end{enumerate}
	
	\Korollar
	Es sei M eine Menge. Dann gilt: 
	\begin{tasks}
		\task M \hyperref[def:Offen]{offen} $\Leftrightarrow M = \mathring{M}$
		\task M abgeschlossen $\Leftrightarrow M = \overline{M}$
	\end{tasks}
	
	\Definition[Topologie des $\mathbb{R}^n$]
	Die Topologie des $\mathbb{R}^n$ ist das Mengensystem
	\[\mathcal{T} := \{U \subset \mathbb{R}^n | \text{U offen}\}\]
	die Menge der \hyperref[def:Offen]{offenen} Mengen
	
	\subsection{Kompaktheit und die Existenz von Extrema}
	\Definition[Beschränkt und Kompakt]
	\begin{enumerate}
		\item Ein $M \in \mathbb{R}^n$ heißt beschränkt, wenn gilt: $\exists s > 0 : M \subset B_s(0)$ \newline
		D.h. $\forall x \in M : |x| \le s$.
		\item Eine Menge $M \subset \mathbb{R}^n$ heißt kompakt, falls jede Folge in M eine konvergente Teilfolge mit Limes in M besitzt \newline
		$\forall (x_n)_{n \in \mathbb{N}}$ mit $x_n \in M \forall n \in \mathbb{N} \exists$ Teilfolge $(x_{n_k})_{k \in \mathbb{N}}$ mit $ \lim\limits_{k \rightarrow \infty} x_{n_l} = x \in M$
	\end{enumerate}
	
	\Satz[Heine-Borel]
	Eine Menge $M \subset \mathbb{R}^n$ ist genau dann kompakt wenn sie beschränkt und abgeschlossen ist.
	
	\Satz
	Sei $\emptyset \ne K \subset \mathbb{R}$ kompakt. Dann existiert max K und min K.
	
	\Satz
	Das stetige Bild einer Kompakten Menge ist Kompakt 
	
	D.h. Es sei $f: K \rightarrow \mathbb{R} (\text{oder } \mathbb{C})$ eine \hyperref[def:Stetig]{stetige} Funktion und $K \in \mathbb{R}^n$ kompakt. Dann ist $f(K) \subset \mathbb{R} (\text{oder } \mathbb{C})$ kompakt. 
	
	\Korollar[Existenz von Extrema]
	Es sei $\emptyset \ne K \subset \mathbb{R}$ kompakt und $f: K \rightarrow \mathbb{R}$ \hyperref[def:Stetig]{stetig}. Dann nimmt f ein Maximum und ein Minimum an. D.h \newline
	$\exists x_{min}, x_{max} \in K: \min f := f(x_{min}) \le f(x) \le f(x_{max}) =: \max f \forall x \in K$
	
	\subsection{Zusammenhang und Zwischenwertsatz}
	\Definition[Zusammenhängend]
	\label{def:Zusammenhangend}
	Eine Menge $M \subset \mathbb{R}^n$ heißt zusammenhängend, falls es keine \hyperref[def:Offen]{offenen} Mengen $U_1, U_2 \subset \mathbb{R}^n$ gibt mit:
	
	\begin{tasks}[counter-format=(tsk[r]), label-width=4ex]
		\task $M \cap U_1 \ne \emptyset \ne M \cap U_2$
		\task $M = (M \cap U_1) \cup (M \cap U_2)$
		\task $(M \cap U_1) \cap (M \cap U_2) \ne \emptyset$
	\end{tasks}
	
	\Definition[Wegzusammenhängend]
	\label{def:Wegzusammenhangend}
	Eine Teilmenge $W \subset \mathbb{R}^n$ heißt wegzusammenhängend, falls es für alle $x, y \in W$ eine stetige	Abbildung $\gamma : [0, 1] \rightarrow \mathbb{R}^n$ gibt mit $\gamma([0, 1]) \subset W$ und $\gamma(0) = x$ sowie $\gamma(1) = y$. Die Abbildung $\gamma$ nennt man in dieser Situation einen Weg von x nach y in W.
	
	\Satz
	Es sei $\emptyset \ne M \subset \mathbb{R}^n$. Dann gilt: 
	\[M \text{ ist ein Intervall} \Leftrightarrow \forall a, b \in M \text{ mit } a < b : [a, b] \subset M\]
	
	\Satz
	Die \hyperref[def:Zusammenhangend]{Zusammenhängenden} Teilmengen von $\mathbb{R}$ sind genau die Intervalle: \newline
	$M \subset \mathbb{R} \text{ \hyperref[def:Zusammenhangend]{Zusammenhängend} } \Leftrightarrow$ M ist ein Intervall
	
	\Satz[Offen und Abgeschlossen] 
	Eine Menge $D \subset \mathbb{R}^n$ sei gegeben. Dann heißt $M \subset D \text{ offen in D}$, falls es eine \hyperref[def:Offen]{offene} Menge $U \subset \mathbb{R}^n$ mit $M = D \cap U$ gilt. 
	
	Analog mit M ist abgeschlossen in D, falls eine abgeschlossene Menge $A \subset \mathbb{R}^n$ existiert mit $M = D \cap A$
	
	\Lemma
	Die Teilmenge $M \subset D$ ist genau dann \hyperref[def:Offen]{offen} in D, wenn gilt:
	\[\forall x \in M \exists r > 0 : \hyperref[def:Br]{B_r}(x) \cap D \subset M\]
	
	
	\Lemma[stetig = Urbild offener Mengen sind offen]
	\label{lem:Urbild}
	Es sei $D \subset \mathbb{R}^n$ und $f: D \rightarrow \mathbb{R}$ gegeben. 
	
	f ist \hyperref[def:Stetig]{stetig} $\Leftrightarrow \forall U \in \mathbb{R}$ \hyperref[def:Offen]{offen} : $f^{-1} (U) \subset D$ ist \hyperref[def:Offen]{offen} in D. 
	
	\Lemma[Stetige Bilder zusammenhängender Mengen sind zusammenhängend]
	Sei $f: D \rightarrow \mathbb{R}$ \hyperref[def:Stetig]{stetig} und $D \subset \mathbb{R}^n$ \hyperref[def:Zusammenhangend]{Zusammenhängend}
	
	$\Rightarrow f(D) \subset \mathbb{R}$ ist \hyperref[def:Zusammenhangend]{Zusammenhängend}. 
	
	\Satz[Zwischenwertsatz]
	Es sei $f: [a, b] \rightarrow \mathbb{R}$ eine \hyperref[def:Stetig]{stetige} Funktion. Dann nimmt f alle Werte zwischen f(a) und f(b) an. \newline
	D.h. $f([a, b]) \supset [f(a), f(b)]$ falls gilt $f(a) \le f(b)$\newline
	$f([a, b]) \supset [f(b), f(a)]$ falls gilt $f(b) \le f(a)$
	
	\Korollar
	Jedes reelle Polynom ungeraden Grades hat eine Nullstelle 
	
	D.h. $f(x) = a_n * x^n + ... + a_1 * x + a_0$ mit $a_0, ..., a_n \in \mathbb{R}$ und $a_n \ne 0$ und n ungerade, dann $\exists x_0 \in \mathbb{R} : f(x_0) = 0$
	
	\Definition[(Streng-) Monoton Steigend]
	\label{def:Monoton}
	$f: [a, b] \rightarrow \mathbb{R}$ heißt (streng) monoton steigend, falls gilt: 
	\[\forall x, y \in [a, b]: x \le y \Rightarrow f(x) \le f(y) \text{ bzw für streng } x < y \Rightarrow f(x) < f(y)\]
	
	Analog für (streng) monoton fallend
	
	\Satz
	Es sei $f: [a, b] \rightarrow \mathbb{R}$ \hyperref[def:Stetig]{stetig} und \hyperref[def:Monoton]{streng monoton steigend}. Dann ist $f: [a, b] \rightarrow [f(x), f(y)]$ bijektiv und $f^{-1}$ ist ebenfalls \hyperref[def:Stetig]{stetig} und \hyperref[def:Monoton]{streng monoton wachsend}. 
	\label{satz:BijektivEigenschaften}
	
	Analog für \hyperref[def:Monoton]{streng monoton fallend}.
	\newpage
	\section{Differenzialrechnung}
	Es sei $D \subset \mathbb{R}^n$ \hyperref[def:Offen]{offen} und $x_0 \in D$ für eine Funktion $f: D \backslash \{x_0\} \rightarrow \mathbb{R}$ schreiben wir: $\lim\limits_{x \rightarrow x_0} f(x) = c \in \mathbb{R}$, falls gilt 
	\[\forall \epsilon > 0 \exists \delta > 0 \forall x \in D \backslash \{x_0\} : |x - x_0| < \delta \Rightarrow |f(x) - c| < \epsilon \]
	Dies ist äquivalent zu:
	\[\forall (x_n)_{n \in \mathbb{N}} \in D \backslash \{x_0\} : x_n \xrightarrow{n \rightarrow \infty} x_0 \Rightarrow f(x_n) \xrightarrow{n \rightarrow \infty} c\]
	
	\subsection{Differenzeierbarkeit}
	\label{def:Differenzierbar}
	\Definition[Differenzierbar]
	Sei $I \subset \mathbb{R}$ ein Intervall und $f: I \rightarrow \mathbb{R}$ eine Funktion. Dann heißt f Differenzierbar in $x_0 \in I$, falls $\lim\limits_{x \rightarrow x_0} \frac{f(x) - f(x_0)}{x - x_0}$ existiert. 
	
	Dann heißt $f'(x_0) := \frac{df}{dx}(x_0) := \frac{d}{dx} f(x_0) := \lim\limits_{x \rightarrow x_0} \frac{f(x) - f(x_0)}{x - x_0}$ Die Ableitung von f in $x_0$. f heißt Differenzierbar auf I, falls f in jedem Punkt von I Differenzierbar ist. Die Ableitung von f ist die Funktion:
	
	\begin{align*}
	f': & I \rightarrow \mathbb{R} \\
	& x \rightarrow f'(x)
	\end{align*}
	
	\Definition[Tangente]
	Tangente an (Graphen von) f in $x_0$ ist $x \rightarrow f(x_0) + f'(x_0)(x - x_0)$
	
	\Lemma
	$f: I \rightarrow \mathbb{R}$ ist genau dann \hyperref[def:Differenzierbar]{Differenzierbar} in $x_0 \in I$, falls es eine Funktion $\Delta: I \rightarrow \mathbb{R}$ gibt mit
	\begin{tasks}
		\task $\Delta \text{ stetig in } x_0$
		\task $f(x) = f(x_0) + (x - x_0) * \Delta(x)$ $\forall x \in I$
	\end{tasks}
	
	\Satz[Differenzierbar $\Rightarrow$ stetig]
	Es sei $f: I \rightarrow \mathbb{R}$ \hyperref[def:Differenzierbar]{Differenzierbar} in $x_0 \in I$, dann ist f \hyperref[def:Stetig]{stetig} in $x_0$
	
	\subsection{Differentationsregeln}
	Im folgenden seien $f, g: I \rightarrow \mathbb{R}$ \hyperref[def:Differenzierbar]{Differenzierbar} in $x_0$:\newline
	$f(x) = f(x_0) + (x - x_0) \Delta_f (x)$\newline
	$g(x) = g(x_0) + (x - x_0) \Delta_g (x)$
	
	mit $\Delta_f, \Delta_g$ \hyperref[def:Stetig]{stetig} in $x_0$
	
	\Satz
	$\lambda f, \lambda \in \mathbb{R}$, und $f \pm g$ sind \hyperref[def:Differenzierbar]{Differenzierbar} in $x_0$ mit \newline
	$(\lambda f)'(x_0) = \lambda f'(x_0)$ \newline
	$(f \pm g)' (x_0) = f'(x_0) \pm g'(x_0)$
	
	\Satz[Produktregel/Leibniz-Regel]
	$f * g$ sind \hyperref[def:Differenzierbar]{Differenzierbar} in $x_0$ mit \newline
	$(f * g)'(x_0) = f'(x_0)g(x_0) + f(x_0)g'(x_0)$ 
	
	\Satz[Quotientenkriterium]
	Sei $g(x_0) \ne 0$. Dann ist $g(x) \ne 0$ für alle x in der Nähe von $x_0$ und $\frac{f(x)}{g(x)}$ ist \hyperref[def:Differenzierbar]{Differenzierbar} mit: \newline
	$(\frac{f(x)}{g(x)})'(x_0) = \frac{f'(x_0)g(x_0) - f(x_0)g'(x_0)}{(g(x_0))^2}$
	
	\Satz[Kettenregel]
	Es seien $f: I \rightarrow \mathbb{R} \text{ und } g: J \rightarrow \mathbb{R}$ Funktionen auf den Intervallen $I, J \subset R$ mit $f(I) \subset J$. Falls f \hyperref[def:Differenzierbar]{Differenzierbar} in $x_0$ und g \hyperref[def:Differenzierbar]{Differenzierbar} in $y_0$ mit $y_0 := f(x_0)$ so ist $g \circ f$ \hyperref[def:Differenzierbar]{Differenzierbar} in $x_0$ mit: \newline
	$(g \circ f)'(x_0) = g'(f(x_0)) * f'(x_0)$.
	
	\Satz
	Es sei $f: [a, b] \rightarrow \mathbb{R}$ \hyperref[def:Monoton]{streng monoton}, \hyperref[def:Stetig]{stetig} und in $x_0 \in [a, b]$ \hyperref[def:Differenzierbar]{Differenzierbar} mit $f(x_0) \ne 0$ Dann ist die nach \hyperref[satz:BijektivEigenschaften]{Satz 4.12} wohldefinierte Umkehrabbildung $f^{-1}$ in $y_0 := f(x_0)$ \hyperref[def:Differenzierbar]{Differenzierbar} mit: \newline
	$(f^{-1})'(y_0) = \frac{1}{f'(x_0)}$
	
	\subsection{Mittelwertsatz und die Existenz von Extrema}
	$I \subset \mathbb{R}$ Intervall 
	\begin{align*}
	\rightsquigarrow & C^0(I) := \{f: I \rightarrow \mathbb{R} | f \text{ stetig}\} \\
	& C^1(I) := \{f: I \rightarrow \mathbb{R} | f \text{ ist stetig Differenzierbar}\}
	\end{align*}
	Dabei heißt f \hyperref[def:Stetig]{stetig} \hyperref[def:Differenzierbar]{Differenzierbar}, falls f \hyperref[def:Differenzierbar]{Differenzierbar} und $f': I \rightarrow \mathbb{R}$ \hyperref[def:Stetig]{stetig} ist. \newline
	falls f' \hyperref[def:Differenzierbar]{Differenzierbar}, so heißt f 2-fach \hyperref[def:Differenzierbar]{Differenzierbar}. \newline
	Wir schreiben $f^{(2)}(x) := f''(x) := (f')'(x)$ 2-te Ableitung. \newline
	Analog: $f^{(k + 1)}(x) := (f^{(k)})'(x)$ falls diese Ableitung existiert. \newline
	f heißt k-fach \hyperref[def:Stetig]{stetig} \hyperref[def:Differenzierbar]{Differenzierbar}, falls $f^{(k)}: I \rightarrow \mathbb{R}$ existiert und \hyperref[def:Stetig]{stetig} ist. \newline
	$\rightsquigarrow C^k(I) := \{f: I \rightarrow \mathbb{R} | f \text{ k-fach stetig Differenzierbar}\}$
\end{document}