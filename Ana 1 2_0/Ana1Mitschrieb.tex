\documentclass[12pt,a4paper]{article} % using article ensures it starts at 1 and does not have odd numberings for section
\usepackage{graphicx}
\usepackage[utf8]{inputenc} % this way umlaute are included from the get go
\usepackage[ngerman]{babel} % german spell check
\usepackage{datetime}

\usepackage{amsmath} % this package is one option for math lines

\usepackage{hyperref} % these two lines are so that the table of content is clickable
\usepackage{amssymb} % package for Natural Number sign etc
\hypersetup{linktoc=all}

\newcounter{Definition}[section]
\newcounter{Lemma}[section]
\newcounter{Korollar}[section]
\newcounter{Satz}[section]

\newcommand{\Definition}{
	\stepcounter{Definition}
	\underline{Definition \theDefinition:}
}

\newcommand{\Lemma}{
	\stepcounter{Lemma}
	\underline{Lemma \theLemma:}
}

\newcommand{\Korollar}{
	\stepcounter{Korollar}
	\underline{Korollar \theKorollar:}
}

\newcommand{\Satz}{
	\stepcounter{Satz}
	\underline{Satz \theSatz:}
}

\newdateformat{gerDate}{\THEDAY \space \monthname[\THEMONTH], \THEYEAR} % german date format

\begin{document}
	\begin{titlepage} % good tital page template, only needs the class X notes part to be changed for each new class
	\centering
	\includegraphics[width=0.40\textwidth]{UniLogo}\par\vspace{1cm}
		{\scshape\LARGE Universität Heidelberg \par}
		\vspace{1cm}
		{\Huge\bfseries Analysis 1 \par}
		\vspace{1cm}
		{\LARGE\bfseries Prof. Dr. Peter Albers \par}
		\vspace{1cm}
		{\huge Wintersemester 17/18 \par}
		\vspace{2cm}
		{\Large\itshape by Charles Barbret \par}
		
		\vfill

		% Bottom of the page
		{\large \gerDate\today\par}
	\end{titlepage}

\tableofcontents % creats a table of contents, ensured already that it is clickable
\newpage % starts the actual document on a new page so there is no weird colision of text and toc


\section*{Ablauf der Vorlesung}
\subsection{Moodle}
Modle passwort: Ableitung
\subsection{Übungsbetrieb}
Donnerstags kommen die neuen Zettel \newline
Zettel sollen in Zweiergruppen abgegeben werden \footnote{ist ja knuffig} \newline
Abgabeschluss ist Donnerstags 09:15 im Mathematikon bei den Briefkästen \newline
Das erste Blatt wird spätestenz am 19.10.2017 veröffentlicht 

\subsection{Plenarübung}
Donnerstags 16:00 bis 18:00 im KIP HS1 \newline
Erste Plenarübung findet bereits am 19.10.2017

\subsection{Klausur}
50\% der Gesamtpunkte der Aufgabenblätter und einmal vorgerechnet haben.

Klausurtermine: 23.02.2018 09:00 bis 13:00 Uhr und 9.4.2018 09:00 bis 13:00 Uhr

Nicht erscheinen bei der ersten Klausur ohne Abmeldung gilt als 5.0, man kann dann aber immer noch in die Nachklausur.

\section{Grundlagen}
\subsection{Mengen und Aussagen}

\Definition \footnote{Cantor 1895}
Eine \underline{Menge} ist eine Zusammenfassung von wohldefinierten und wohl unterschiedenen Objekten zu einem Ganzen. Die Objekte heißen \underline{Elemente} der Menge. \newline
\underline{wohlbestimmt}: Von jedem Objekt steht fest, ob es Element der Menge ist oder nicht \newline
\underline{wohl unterschieden}: Jedes Objekt kommt höchstens einmal in der Menge vor \newline
\begin{tabular}{l l}
Beschreibung der Menge & a) Durch Aufzählung $\mathbb{N}: = \{1, 2, 3, ...\} $\\
& \parbox[t]{8cm}{b) Durch Angabe einer charakteristischen Eigenschaft in Form einer Aussage, d.h. eines Satzes, von dem eindeutig feststeht, ob er wahr oder falsch ist. Der Wahrheitsgehalt muss zum gegebenen Zeitpunkt nicht bekannt sein.
	Bsp A(u):= u ist eine Primzahl (D.h. $u \ge 2$)} \\
& c) Durch Beschreibung der Elemente 
\end{tabular}
\Definition Es sein M und N Mengen
[...]
\newline
\underline{Bemerkung}
"oder" bedeutet das einschließliche oder, also nicht "{}entweder \dots oder \dots" $\rightarrow$ Wahrheitstabellen \newline
Seien A und B Aussagen. Wir leiten ab \footnote{w = wahr, f = falsch}

\begin{tabular}{c | c | c | c | c}
	A & B & A und B & A oder B & "{}Entweder A oder B" \\ \hline
	w & w & w & w & f \\
	w & f & f & w & w \\
	f & w & f & w & w \\
	f & f & f & f & f
\end{tabular}

\underline{Implikation} zwischen Aussagen

\begin{tabular}{l l}
	A $\Rightarrow$ B & steht für: A impliziert B \\
	 & Falls A gilt, dann gilt auch B \\
	 & A ist \underline{hinreichende} Bedingung für B \\
	 & B ist \underline{notwendige} Bedingung für A
\end{tabular}

Formal definieren wir A $\Rightarrow$ B ist wahr, falls $\neg$ A oder B

D.h.
\begin{tabular}{c | c | c}
	A & B & A $\Rightarrow$ B \\ \hline
	w & w & w \\
	w & f & f \\
	f & w & w \\
	f & f & w
\end{tabular}

Außerdem kürzen wir ab:

A $\Leftrightarrow$ B steht für A $\Rightarrow$ B $\land$ A $\Leftarrow$ B

Beispiel:

[\dots]

\underline{Bemerkung}:\newline
Für alle Mengen M gilt $\emptyset \subset M$ \newline
$\forall M: \emptyset \subset M$.

\newpage

\section{Reelle Zahlen}

\newpage

\section{Folgen}
%scheint als ob 2 sätze übersprungen wurden im skript
\stepcounter{Satz}
\stepcounter{Satz}

\Definition Eine \underline{Folge komplexer Zahlen} ist eine Abbildung 

\begin{align*}
	a: & \mathbb{N} \rightarrow \mathbb{C} \\
	& n \rightarrow a(n) = a_n
\end{align*}

\underline{Notation} \(a = (a_n)_{n \in \mathbb{N}}\).

Analog: \(a_{n_0}, a_{n_0 + 1}, a_{n_0 + 2}, ... = (a_n)_{n \ge n_0}\)

Heißt soviel wie, da die Folge von einem Term in Abhängigkeit von n steht, dass man eine Bijektion mit den Natürlichen Zahlen bilden kann.

\subsection{Konvergent und Grenzwert}

\Definition 1) Eine Folge \(a_n)_{n \in \mathbb{N}}\) heißt \underline{konvergent}, falls $a \in \mathbb{C}$ mit filgender Eigenschaft existiert:

\label{formel:Konvergenz}
\[\forall \epsilon > 0 \exists n_0 = n_0(\epsilon) \in \mathbb{N} \forall n \ge n_0: |a_n - a| < \epsilon\]

Dann heißt a \underline{Grenzwert} oder \underline{Limes} von $(a_n)$.

\begin{align*}
	Schreibweise: &a_n \rightarrow a, a_n \xrightarrow{n \rightarrow \infty} a \\
	&\lim a_n = a, \lim\limits_{n \rightarrow \infty} a_n = a \\
\end{align*}

2) Hat eine Folge keinen Grenzwert, heißt sie \underline{divergent}.

3) Gilt $a_n \rightarrow 0$, so heißt $(a_n)$ \underline{Nullfolge}.
\newline

\underline{WICHTIG} Die Aussagen "für alle bis auf endlich viele Folgenglieder" und "für unendlich viele Folgenglieder" sind nicht äquivalent.
\newline {}

\Lemma a) Der Grenzwert einer konvergenten Folge ist eindeutig bestimmt.

b) Jede konvergente Folge ist beschränkt:

\label{formel:KonvergentBeschrnkt}
\[\exists s \in \mathbb{R} \forall n \in \mathbb{N}: |a_n| \le s \]
\newline {}

\Satz a) Für $a \in \mathbb{R}^+$ gilt: $\lim\limits_{n \rightarrow \infty} \sqrt[n]{a} = 1$.

\label{satz:nthRoot}
b) $\lim\limits_{n \rightarrow \infty} \sqrt[n]{n} = 1$

\subsection{Rechenregeln für Grenzwerte}

\Lemma Es seien $(a_n)$ und $(b_n)$ \underline{konvergente} Folgen mit $a_n \rightarrow a$ und $b_n \rightarrow b$. Dann sind $(\lambda a_n)_{n \in \mathbb{N}}, \lambda \in \mathbb{C}, (a_n + b_n)_{n \in \mathbb{N}}, (a_n * b_n)_{n \in \mathbb{N}}$ und, falls $b \ne 0: (\frac{a_n}{b_n})_{n \in \mathbb{N}}$ konvergent.

Heißt soviel wie, sollten die Folgen jeweils konvergieren, kann man die Grenzwerte miteinander verrechnen.
\newline {}

\Lemma Es seien $(a_n)$ und $(b_n)$ konvergente Folgen. Es gelte $a_n \le b_n$ für unendlich viele $n \in \mathbb{N}$ 

Dann gilt: $\lim a_n \le \lim b_n$.

Hiermit können wir Folgen abschätzen, was manchmal ganz hilfreich sein kann.
\newline {}

\Lemma Sandwich-Lemma
\label{lem:Sandwich}

Es seien $(a_n), (b_n), (c_n)$ reelle Folgen mit:

i) $(a_n)$ und $(c_n)$ sind konvergent mit
\[\lim a_n = \lim c_n\]

ii) Für alle bis auf endlich viele $n \in \mathbb{N}$ gilt
\[a_n \le b_n \le c_n\]

Dann ist $(b_n)$ konvergent mit
\[\lim a_n = \lim b_n = \lim c_n\]

Wenn wir eine Minorante und Majorante finden können, mit dem gleichen Grenzwert, so konvergiert die Folge gegen den selben Grenzwert

\subsection{Häufungspunkt, Cauchy-Folge, Teilfolgen}

\Definition Eine Folge $(a_n)_{n \in \mathbb{N}}$ reeller Zahlen heißt \underline{monoton wachsend}, falls gilt:
\[\forall n \in \mathbb{N}: a_{n + 1} \ge a_n\]

Monoton Fallend analog.
\newline {}

\Satz Jede beschränkte monotone Folge $(a_n)_{n \in \mathbb{N}}$ ist konvergent mit 

$\lim\limits_{n \rightarrow \infty} a_n = \left\lbrace 
	\begin{array}{l l}
		sup \{a_n | n \in \mathbb{N}\} & \text{falls monoton wachsend} \\
		inf \{a_n | n \in \mathbb{N}\} & \text{sonst}
	\end{array}
\right.$

Monoton + Beschränkt = Konvergent
\newline {}

\Definition Sei $(a_n)$ eine Folge. Dann heißt $a \in \mathbb{C}$ \underline{Häufungspunkt} der Folge, falls gilt:

$\forall \epsilon > 0$ gilt $|a_n - a| < \epsilon$ für unterschiedlich viele $n \in \mathbb{N}$

Heißt soviel wie wenn \hyperref[def:Teilfolge]{Teilfolgen} konvergieren, sind die Grenzwerte der Teilfolgen die Häufungspunkte der Folge.
\newline {}

\label{def:Teilfolge}
\Definition Sei $(a_n)_{n \in \mathbb{N}}$ eine Folge komplexer Zahlen und $(n_l)_{l \in \mathbb{N}}$ eine Folge natürlicher Zahlen mit \(\forall l \in \mathbb{N}: n_{l + 1} > n_l\). Dann heißt $(a_{n_l})_{l \in \mathbb{N}}$ eine \underline{Teilfolge} von $(a_n)_{n \in \mathbb{N}}$
\newline {}

\Lemma $a \in \mathbb{C}$ ist genau dann ein Häufungspunkt der Folge $(a_n)_{n \in \mathbb{N}}$, wenn es eine konvergente Teilfolge $(a_{n_l})_{l \in \mathbb{N}}$ gibt mit $\lim\limits_{n \rightarrow \infty} a_{n_l} = a$.

Also genau was ich gemeint hatte.
\newline {}

\Satz $\mathbb{Q}$ liegt dicht in $\mathbb{R}$
\newline {}

\Lemma Jede Folge reeller Zahlen besitzt eine monotone Teilfolge.

Heißt, bei jeder Folge reeller Zahlen kann man sich vereinzelnd Elemente raus suchen, welche eine monotone Teilfolge bilden.
\newline {}

\Satz (Bolzano-Weierstraß für $\mathbb{R}$)

Jede beschränkte Folge reeller Zahlen besitzt eine konvergente Teilfolge

Gilt natürlich auch in komplexen Zahlen.
\newline {}

\label{def:Cauchy}
\Definition Eine Folge $(a_n)_{n \in \mathbb{N}}$ komplexer Zahlen heißt \underline{Cauchy-Folge}, falls gilt:
\[\forall\epsilon > 0 \exists n_0 = n_0(\epsilon) \in \mathbb{N} \forall m, n \ge n_0: |a_m - a_n| < \epsilon\]
\newline {}

\Satz (Cauchy Kriterium)

Die Folge $(a_n)$ ist genau dann konvergent, wenn $(a_n)$ eine Cauchy-Folge ist.

\underline{Bemerkung} Es gibt Cauchy-Folgen $(a_n)$ mit $a_n \in \mathbb{Q} \forall n \in \mathbb{N}$ die nicht gegen eine rationale Zahl konvergieren.

\subsection{Limes superior und Limes inferior}

Es sei $(a_n)_{n \in \mathbb{N}}$ eine beschränkte Folge, d.h. $\exists s \in \mathbb{R} \forall n \in \mathbb{R}: |a_n| \le s$.

\begin{align*}
	\text{\underline{Beobachtung} Sei } & B_n := sup\{a_k  \ge n\} \\
	& b_n := inf\{a_k  \ge n\}
\end{align*}

Dann ist die Folge $(B_n)_{n \in \mathbb{N}}$ monoton fallend und $(b_n)_{n \in \mathbb{N}}$ monoton wachsend.

Außerdem gilt: $\forall n \in \mathbb{N}: b_n \le B_n$

TEST
\end{document}