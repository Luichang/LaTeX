\documentclass[12pt,a4paper]{article} % using article ensures it starts at 1 and does not have odd numberings for section
\usepackage{graphicx}
\usepackage[utf8]{inputenc} % this way umlaute are included from the get go
\usepackage[ngerman]{babel} % german spell check
\usepackage{datetime}

\usepackage{breqn} % this package is one option for math lines

\usepackage{hyperref} % these two lines are so that the table of content is clickable
\hypersetup{linktoc=all}

\usepackage{amssymb} % package for Natural Number sign etc

\usepackage{soul} % this package is for strike outs

\newcommand{\itab}[1]{\hspace{0em}\rlap{#1}}
\newcommand{\tab}[1]{\hspace{.2\textwidth}\rlap{#1}} % own tab function

\newdateformat{gerDate}{\THEDAY \space \monthname[\THEMONTH], \THEYEAR} % german date format

\begin{document}
	\begin{titlepage} % good tital page template, only needs the class X notes part to be changed for each new class
	\centering
	\includegraphics[width=0.40\textwidth]{UniLogo}\par\vspace{1cm}
		{\scshape\LARGE Universität Heidelberg \par}
		\vspace{1cm}
		{\huge\bfseries Numerik Zusammenfassung \par}
		\vspace{2cm}
		{\Large\itshape by Charles Barbret \par}
		
		\vfill

		% Bottom of the page
		{\large \gerDate\today\par}
	\end{titlepage}

	\tableofcontents % creats a table of contents, ensured already that it is clickable
	\newpage % starts the actual document on a new page so there is no weird colision of text and toc
	
	
	\section{Fehleranalyse}
	\subsection[Zahlendarstellung und Rundungsfehler]{Zahlendarstellung und Rundungsfehler Schema für Gleitkomma Zahl}
	\begin{displaymath}
		x = \pm m * b^{\pm e}
	\end{displaymath}
	Basis $b \in \mathbb{N}$ \space b $\geq 2 \space $ \footnote{Beispiel: $ 2^e$ oder $10^e $}
	\newline
	Mantisse $ m = m_1 b^{-1} + m_2 b^{-2} + ... \in \mathbb{R} $ \footnote{ Beispiel: $ m_1 = 3, m_2 = 1, m_3 = 4 \Rightarrow m = 314 $ normal kommt nach $ m_1 $ ein Komma, also m = 3,14}
	\newline
	Exponent e = $e_1 b^{1} + e_2 b^{2} + ... \in \mathbb{N}_0 $
	\newline
	$ \forall m_i \in m $ und $ \forall e_i \in e $ gilt $e_i , m_i \in $ \{0, ..., b - 1\}
	\newline
	Sollte b = 10 sein befinden wir uns im Dezimalsystem
	\newline \indent 
	$\Rightarrow$ Es gibt keine Ziffer > 9 = 10 - 1 $\Leftrightarrow$ b - 1 $q.e.d$
	\newline
	Jede Ziffer ($m_i , e_i , b, \pm, \pm$) braucht man eine Speicherzelle, wobei b im Computer bereits eingespeichert ist
	\newline \indent 
	$\Rightarrow $ muss nicht explizit angegeben werden
	\newline
	X wird gespeichert als: ($\pm$)[$m_1 , \dots, m_r $]($\pm$)[$e_{s - 1} , \dots, e_0 $]
	\newline
	r + 1 Einträge um m zu speichern ($\pm$ ist ein Eintrag)
	\newline
	s + 1 Einträge um s zu speichern ($\pm$ ist ein Eintrag)
	\newline
	$X_{max/min} = \pm (1 - b^-r) * b^{b^s - 1}$ \footnote{1 oder 0 für erste Speicherzelle, sonst nur 1en}
	\newline
	$X_{posmin/negmax} = \pm b^{-b^s}$ \footnote{1 oder 0 für erste Speicherzelle, eine 1, sonst nur 0en}
	\newline
	\subsubsection{Rundungsoperation}
	$\|x - rd(x)\|$ = $min_{y \in A}\|x - y\|$ \footnote{Wobei diese Operation x als nächst darstellbare Zahl zurück gibt}
	\newline
	Dies verläuft von D $\rightarrow$ A, wobei D:=$[X_{min}, x_{negmax}]\{0\}[X_{posmin}, X_{max}]$
	ist und A die Menge der darstellbaren Zahlen \footnote{D gibt die Theoretisch minimalen Zahlen bis Theoretisch maximalen Zahlen an}
	\newline
	$rd(x) = \pm$
	$\left\{
	\begin{array}{ll}
		m_1 \dots m_{53} * 2^e & für \space m_{54} = 0 \\ % OK so there is an error here but I have no idea how to fix it
		(m_1 \dots m_{53} + 2^{-53}) * 2^e \footnote{Hier wird die Zahl um 1 incrementiert, als Rundung} & für \space m_{54} = 1
	\end{array}
	\right.$
	
	Der absolute Rundungsfehler sieht aus:
	\begin{displaymath}
		|x - rd(x)| \le \frac{1}{2}b^{-s}b^e
	\end{displaymath}
	TODO: insert image
	\newline
	absolut weil er noch vom Exponenten abhängt
	\newline
	Der relative Fehler:
	\begin{displaymath}
		|\frac{x - rd(x)}{x} \le \frac{1}{2} \frac{b^{-r} \st{b^e}}{|m| \st{b^e}}
	\end{displaymath}
	
	
\end{document}