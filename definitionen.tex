\documentclass[12pt,a4paper]{article} % using article ensures it starts at 1 and does not have odd numberings for section
\usepackage{graphicx}
\usepackage[utf8]{inputenc} % this way umlaute are included from the get go
\usepackage[ngerman]{babel} % german spell check
\usepackage{datetime}

%\usepackage{breqn} % this package is one option for math lines
\usepackage{mathtools} % this package contains math functions and a kind of table generator (align)
\usepackage{mathdots}

\usepackage{hyperref} % these two lines are so that the table of content is clickable
\hypersetup{linktoc=all}

\usepackage{amssymb} % package for Natural Number sign etc

\usepackage[makeroom]{cancel}

%added commands:
\newcommand*\conj[1]{\overline{#1}}
\newcommand*\mean[1]{\bar{#1}}
\newcommand*\tab[1][1cm]{\hspace*{#1}}
\newcommand*\rfrac[2]{{}^{#1}\!/_{#2}} % Diagonal slash fraction

%added commands:
%\newcommand*\conj[1]{\overline{#1}}
%\newcommand*\mean[1]{\bar{#1}}
%\newcommand*\tab[1][1cm]{\hspace*{#1}}
%\newcommand*\rfrac[2]{{}^{#1}\!/_{#2}} % Diagonal slash fraction

\begin{document}
	\subsection{Norm}
		Definitheit: $||x|| = 0 \Rightarrow x = 0$
		
		absolute Homogenität: $||\alpha x|| = |\alpha| * ||x||$
		
		Dreiecksungleichung: $||x + y|| \le ||x|| + ||y||$
	
	\subsection{Skalarprodukt}
	
		\[
			\left.
				\begin{array}{l}
					<x + y, z> = <x, z> + <y, z> \\
					<x, y + z> = <x, y> + <x, z> \\
					<\lambda x, y> = \lambda <x, y> \\
					<x, \lambda y> = \lambda <x, y>
				\end{array}
			\right \} \textnormal{Linearität}
		\]
		
		\begin{math}
			<x, y> = <y, x>
		\end{math}
		
		\[
			\left.
				\begin{array}{l}
					<x, x> \ge 0 \\
					<x, x> = 0 \Rightarrow x = 0
				\end{array}
			\right \} positiv Definitheit
		\]
	
	\section{Symmetrische, positiv definite Matrix}
	
	positiv definit: $x^t Ax > 0$ (beliebige Matrix)
	
	alle EW $>$ 0 (symmetrische Matrix)
	
	\subsection{Natürliche Matrixnorm}
	
	$||A||_\infty := \max\limits_{x \ne 0} \frac{||Ax||_\infty}{||x||_\infty} = \max\limits_{||x|| = 1}||Ax||_\infty$
	
	$||A|| = 0 \Rightarrow A = 0 $
	
	$||\lambda A|| = |\lambda|*||A|| $
	
	$||A+B|| \le ||A|| + ||B||$
	
	$||A*B|| \le ||A|| * ||B||$
	
	\subsection{Zeilensummennorm}
	= natürliche Matrixnorm
	
	$||A||_\infty = \max\limits_{||x||_\infty = 1} ||Ax||_\infty = \max\limits_{i = 1, ..., m} \sum\limits_{j = 1}^{n} |a_{ij}|$
	
	\subsection{Spaltensummennorm}
	
	$||A||_1 := \max\limits_{x \ne 0} \frac{||Ax||_1}{||x||_1} = \max\limits_{||x||_1 = 1} ||Ax||_1 = \max\limits_{j = 1, ..., n} \sum\limits_{i = 1}^{m}|a_{ij}|$
	
	\subsection{Spektralnorm}
	
	\begin{equation*}
	\begin{split}
	A||_2 &:= \max\limits_{||x||_2 = 1} ||Ax||_2 \\
	&= \max\limits_{x \ne 0} \frac{||Ax||_2}{||x||_2} \\
	&= \max\limits_{||x||_2 = 1} <Ax, Ax> \\
	&= \max\limits_{||x||_2 = 1} <A^tAx, x> \\
	&= max{\sqrt{|\lambda |}, \lambda * EW von A^tA}
	\end{split}
	\end{equation*}
	
	\subsection{Konditionszahl einer Matrix A}
	
	$cond(A) = ||A||*||A^{-1}||$
	
	\subsection{Sonderfall symmetrisch, positiv definite Matrix}
	
	$cond(A) = \frac{ \lambda_{max}}{ \lambda_{min}}$
	
	\subsection{Gleitkommazahl (normalisiert)}
	
	$b \in \mathbb{N}, b \ge 2, x \in \mathbb{R}$
	
	$x = \pm m * b^{\pm e}$
	
	Mantisse: $m = m_1b^{-1} + m_2b^{-2} + ... \in \mathbb{R}$
	
	Exponent: $e = e_{s-1}b^{s-1} + ... + e_0b^0 \in \mathbb{N}$
	
	für $x \ne 0$ eindeutig
	
	\subsection{Gleitkommagitter}
	
	$A = A(b, r, s)$ größte Darstellbare Zahl: $(1 - b^{-r})*b^{b^s-1}$
	
	mit b als Basis, r als Mantissenlänge, s als Exponentenlänge
	
	\subsection{Maschienengenauigkeit eps}
	
	$eps = \frac{1}{2}b^{-r + 1}, IEEE: eps = \frac{1}{2} * 2^{-52} \approx 10^{-16}$
	
	\subsection{Rundungsfehler}
	
	$absolut: |x - rd(x)| \le \frac{1}{2}b^{-r}b^e$
	
	$relativ: |\frac{x - rd(x)}{x}| \le \frac{1}{2}b^{-r+1} = eps$
	
	\subsection{Fehler I}
	
	f$ \in C^{n+1}$[a, b], $\forall$ x $\in$ [a, b] $\exists \xi_x \in$ ($\overline{x_0, ..., x_n, x}$), wobei das Intervall das kleinst mögliche Intervall, das alle $x_i$ enthällt, s.d.
	
	$f(x) - p(x) = \frac{f^{(n+1)}(\xi x)}{(n+1)!} \prod\limits_{j = 0}^{n}(x - x_j)$
	
	\subsection{numerische Aufgabe}
	
	$x_j \in \mathbb{R} \text{ mit }f(x_1, ..., x_m) \Rightarrow y_i = f_i(x_j)$
	
	fehlerhafte Eingangsgrößen $x_i + \Delta y_i$
	
	$|\Delta y_i|$ ist der absolute Fehler, $|\frac{\Delta y_i}{y_i}|$ ist der relative Fehler
	
	\subsection{Konditionszahl (relativ)}
	
	$k_{ij}(x) = \frac{\partial f_i}{\partial x_i}(x) \frac{\Delta x_j}{x_j}$
	
	$\frac{\Delta y_i}{y_i} = \sum\limits_{j = 1}^{m}k_{ij}(x)\frac{\Delta x_j}{x_j}$
	
	$|k_{ij}(x)| >> 1 \Rightarrow$ schlecht konditioniert
	
	$|k_{ij}(x)| << 1 \Rightarrow$ gut konditioniert, ohne Fehlerverstärkung
	
	$|k_{ij}(x)| > 1 \Rightarrow$ Fehlerverstärkung
	
	$|k_{ij}(x)| < 1 \Rightarrow$ Fehlerdämpfung
	
	\subsection{stabiler Algorithmus}
	
	akkumulierte Fehler der Rechnung (Rundungsfehler, Auswertungsfehler, etc.) übersteigen den unvermeidbaren Problemfehler der Konditionierung der Aufgabe nicht. Aka Trotz Ungenauigkeiten bei den Eingabe Variablen erhalten wir fast sehr genaue Ergebnisse.
	
	\section{Auslöschung}
	
	Verlust von Genauigkeit bei der Subtraktion von Zahlen mit gleichem Vorzeichen
	
	\section{Horner-Schema*}
	
	Das sogenannte "Horner-Schema"
	
	\begin{tabular}{l l l}
		$b_n = a_n$ &, k = n - 1, ..., 0 & $b_k = a_k + \xi b_{k + 1} $
	\end{tabular}
	liefert den Funktionswert $p(\xi) = b_0$ des Polynoms
			
	$p(x) = a_0 + x(... + x(a_{n-1} + a_nx)...)$
	
	\subsection{Code}
	
	def horner(Ac, Ax, n, x): 
	
	y = 0.0
	
	for i in reversed range(n):
	
	\tab y = y * (x - Ax[i]) + Ac[i]
	
	return y
	
	Ac: Vektor mit Koeffizienten, ist ein np Array
	
	Ax: Stützstellen, ist ein np Array
	
	n: Anzahl der Stützstellen, ist ein int
	
	x: Auswertungspunkt, ist ein double
	
	Immer Horner-Schema zur Auswertung von Polynomen verwenden.
	
	\subsection{Interpolation}
	
	Zuordnung von $g \in P$ zu f durch Fixieren von Funktionswerten
	
	$g(x_i) = y_i = f(x_i), i = 0, ..., n$
	
	\subsection{Approximation}
	
	$g \in P$ beste Darstellung, z.B. 
	
	$\max\limits_{a \le x \le b}|f(x) - g(x)| minimal$
	
	$(\int\limits_{a}^{b}|f(x) - g(x)|^2dx)^{\frac{1}{2}} minimal$
	
	\subsection{Lagransche Darstellung}
	
	$p(x) = \sum\limits_{i = 0}^{n}y_iL_i^{(n)}(x) \in P_n$ mit $p(x_j) = y_j$
	
	Nachteil: Bei Hinzunahme von $(x_{n+1}, y_{n+1})$ ändert sich das Basispolynom komplett
	
	\subsubsection{Newton-Polynome}
	
	Bei Hinzunahme von $(x_{n + 1}, y_{n + 1})$ muss nur eine neue Rechnung durchgeführt werden, und nicht das gesamte Polynom neu berechnet werden
	
	\subsection{Dividierte Differenzen*}
	
	$y[x_i, ..., x_{k + 1}] = \frac{y[x_{i + 1}, ..., x_{k + 1}] - y[x_i, ..., x_{i + k - 1}]}{x_{i + k} - x_i}$ mit k = 1, ..., j und i = k - j
	
	für beliebige [?] $\sigma:{0, ..., n} \rightarrow {0, ..., n}$ gilt $y[\tilde{x_0}, ..., \tilde{x_n}] = y[x_0, ..., x_n]$
	
	\subsection{Extrapolationsfehler}
	
	a(n) habe die Entwickling:
	
	$a(h) = a_0 + \sum\limits_{j = 1}^{n}a_jh^{jq} + a_{n + 1}(h)h^{(n+1)q}$ \tab mit q $>$ 0,
	Koeffizienten $a_j$ 
	
	und $a_{n + 1}(h) = a_{n + 1} + a(1[?????])$
	
	$(h_k)_{k \in \mathbb{N}}$ erfülle:
	
	$0 \le \frac{h_{k + 1}}{h_k} \le p < 1$ ($\Rightarrow h_k$ positiv monoton fallend)
	
	Dann gilt für $p_1^{(k)} \in P_n$ (in $h^q$) durch ($h_k^q, a(h_k)$), ..., ($h_{k +n}^q, a(h_{k + 1})$)
	
	$a(0) - p_n^{(k)}(0) = O(h_k^{(n + 1)q})$ \tab ($k \rightarrow \infty$)
	
	\subsection{Kubischer Spline}
	
	$s_n: [a, b] \rightarrow \mathbb{R}$ kubischer Spline bezüglich $a = x_0 < x_1 < ... < x_n = b$, wenn gillt
	
	1. $s_n \in C^2[a, b]$
	
	2. $S_n|_{I_i} \in P_3$, i = 1, ..., n
	
	natürlicher Spline:
	
	3. $s_n''(a) = s_n''(b) = 0$
	
	\section{Gauß-Approximation}
	
	$<f, g>:= \int\limits_{a}^{b}f(t)\conj{g(t)}dt$ \tab $||f|| = \sqrt{<f, f>}$
	
	H Prähilbertraum, $\delta \subset H$ endlich Dimensional
	
	$\exists f \in H$ eindeutig bestimmte "beste Approximation" $g \in S$
	
	\tab $||f - g|| = \min\limits_{\varphi \in S}||f - \varphi||$
	
	bes. einfache Lösung, wenn \{$\varphi_1, ..., \varphi_n$\} eine ONB ist, d.h.
	
	$(\varphi_i, \varphi_j) = \delta_{i, j} \Rightarrow \alpha_i = <f, \varphi_i>$ \tab i = 1, ..., n
	
	\tab $\Rightarrow g = \sum\limits_{i = 1}^{n}<f, \varphi_i>\varphi_i$ ist beste Approximation
	
	\section{Gram-Schmidt-Algorithmus}
	
	$w_1 := \frac{v_1}{||v_1||}$ \space $\tilde{w_k}:=v_k - \sum\limits_{i = 1}^{k - 1}\gamma<v_k, w_i>w_i$, \space $w_k := \frac{\tilde{w_k}}{||\tilde{w_k}||}$
	
	\subsection{Code}
	
	n = size(v, 1)
	
	k = size(v, 2)
	
	u = np.zeros(n, k)
	
	u[:, 1] = v[:, 1]/sqrt(v[:, 1] * v[:, 1])
	
	for i in range(2, k):
	
	\tab u[:, i] = v[:, i]
	
	\tab for j in range(1, i - 1):
	
	\tab \tab u[:, i] = u[:, i] - (u[:, i] * u[:, j]) / (u[:, j] * u[:, j]) * u[:, j]
	
	\tab u[:, i] = u[:, i] / sqrt(u[:, i] * u[:, i])
	
	\subsection{Darstellung des Interpolationsfehler}
	
	Sei $f \in C^{n + 1}[a, b]$. Dann gibt es zu jedem $x \in [a, b]$ ein $\xi_x \in(x_0, ..., x_n, x)$, so dass gilt:
	
	$f(x) - p(x) = \frac{f^{(n + 1)}(\xi_x)}{(n + 1)!}\prod\limits_{j = 0}^n(x - x_j)$ 
	
	\section{Interpolatorische Quadraturformeln}
	
	$I(f) = \int\limits_{a}^{b}f(x)dx \approx I^{(n)}(f) = \sum\limits_{i = 1}^{n}\alpha_if(x_i)$
	
	Stützstellen a $\le a_0 < x_1 < ... < x_n \le b$ und Gewichte $\alpha_i \in \mathbb{R}$
	
	\subsection{Newton-Cotes-Formel*}
	
	äquidistante Stützstellen
	
	\subsection{Abgeschlossene Formeln}
	
	$H = \frac{b - a}{n}, x_i = a + iH, a = x_0, b = x_n$
	
	Trapezregel: $I^{(1)}(f) = \frac{b - a}{2}[f(a) + f(b)]$
	
	Simpsonregel: $I^{(2)}(f) = \frac{b - a}{6}[f(a) + 4f(\frac{a + b}{2}) + f(b)]$
	
	$\rfrac{3}{8}-Regel: I^{(3)}(f) = \frac{b - a}{8}[f(a) + 3f(a + H) + 3f(b - h) + f(b)]$
	
	\subsection{Offene Formeln}
	
	($H = \frac{b - a}{n + 2}, x_i = a + (i + 1)H, a < x_0, x_n < b$)
	
	$I^{(0)}(f) = (b - a)f(\frac{a + b}{2})$ \tab Mittelpunktregel
	
	$I^{(1)}(f) = \frac{(b - a)}{2}(f(a + H) + f(b - H))$
	
	$I^{(2)}(f) = \frac{(b-  a)}{3}(2f(a + H) - f(\frac{a + b}{2}) + 2f(b - H))$
	
	\subsection{Code}
	
	\subsection{Abhilfe: Summierte Quadraturformeln *}
	
	$I-n^{(n)}(f) = \sum\limits_{i = 1}^{N - 1}I_{[x_i, x_i + 1]}^{(n)}(f)$ \tab $h = \frac{b - a}{N}, x_i = a + iH$
	
	\subsection{Gauß-Quadratur}
	
	$\exists!$ interpolierte Quadraturformel [?] (n + 1) paarweise verschiedene Stützstellen auf [-1, b] mit Ordnung 2n + 2. Stützstellen = Nullstellen
	
	$\alpha_i = \int\limits_{-1}^1\prod\limits_{j = 0, j \ne i}(\frac{x - \lambda_j}{\lambda_i - \lambda_j})^2 dx > 0$\footnote{Positivität der Gewichte}, \tab i = 0, ..., n
	
	$f\in C^{2n + 2}([-1, 1]) Restglied:$
	
	$R^{(n)} = \frac{f^{(2n + 2)}(\xi)}{(2n + 2)!}\int\limits_{-1}^1\prod\limits_{j = 0}^n(x - \lambda_j)^2dx$, \tab $\xi \in (-1 , 1)$
	
	\subsection{Kongergenz der Gauß-Quadraturen}
	
	Sei $I^{(n)}(f)$ die (n + 1) punktige [?] Gauß-Formel zu $I(f) = \int\limits_{-1}^1f(x)dx \forall f \in C[-1, 1]: I^{(n)}(f) \xrightarrow{n \rightarrow \infty} I(f)$
	
	\section{Legendre Polynome *}
	
	Die Legendre Polynome $L_n \in P_n$ bzw, ihre Vielfachen $p_n$ lassen sich auf $[-1, 1]$ in der Form $p_n(x) = \frac{n!}{(2n)!}\frac{d^n}{dx^n}(x^2 - 1)^n$ schreiben und genügen der rekursiven Bezeichnung
	
	$p_0(x) \equiv 1$, $p_1(x) \equiv x$
	
	$p_{n + 1}(x) = xp_n(x) - \frac{n^2}{4n^2 - 1}p_{n - 1}(x)$ für $n \ge 1$
	
	\section{Ordnung von Quadraturformeln *}
	
	Eine Quadraturformel $I^{(n)}(\cdot)$ wird "(mindestens) von der Ordnung m" genannt, wenn durch sie wenigstens alle Polynome aus $p_{m - 1}$ exakt integriert werden
	Die interpolatorische Quadraturformel $I^{(n)}(\cdot)$ zu n + 1-Stützstellen sind also mindestens von der Ordnung n + 1
	
	\subsection{Störungssatz}
	
	$A \in \mathbb{K}^{n x n}$ regulär mit $||\delta A|| \le \frac{1}{||A^{-1}||}$, dann gilt für die
	
	\subsection{gestörte Matrix}
	
	$\tilde{A} = A + \delta A$ ist regulär
	
	Für den relativen Fehler der Lösung gilt mit Konditionszahl von A:
	
	\tab $cond(A) = ||A|| * ||A^{-1}||$
	
	die Ungleichung:
	
	\tab $\frac{||\delta_x||}{||x||} \le \frac{cond(A)}{1 - cond(A) \frac{||\delta A||}{||A||}}\left[ \frac{||\delta b||}{||b||} + \frac{||\delta A||}{||A||}\right] $
	
	\subsection{Gewichtetes Skalarprodukt}
	
	$<f, g>_\omega = \int\limits_a^bf(x)g(x)\omega(x)dx$, \tab $\omega(x)\ge 0, x \in (a, b)$
	
	\subsection{Kongergenz der Gauß-Quadraturen}
	
	Sei $I^{(n)}(f)$ die (n + 1) punktige [?] Gauß-Formel zu $I(f) = \int\limits_{-1}^1f(x)dx \forall f \in C[-1, 1]: I^{(n)}(f) \xrightarrow{n \rightarrow \infty} I(f)$
	
	\subsection{Störungssatz}
	
	$\frac{||\delta_x||}{||x||} \le \frac{cond(A)}{1 - cond(A) \frac{||\delta A||}{||A||}}\left[ \frac{||\delta b||}{||b||} + \frac{||\delta A||}{||A||}\right] $
	
	\subsection{Aufwend Vorwärts/Rückwärts einsetzen}
	
	$\sum\limits_{j = 1}^nj = \frac{(n + 1)n}{2} = \frac{n^2}{2} + O(n)$
	
	\subsection{Cholesky-Zerlegung}
	
	Jede symmetrische positiv definite Matrix A hat eine sogenannte Cholesky - Zerlegung:
	
	$A = LDL^t = \tilde{L}\tilde{L}^t$, $\tilde{L} := LD^{\frac{1}{2}}$
	
	\subsection{Aufwand der Cholesky-Zerlegung}
	
	$N_{Cholesky}(n) = \frac{n^3}{6} + O(n^2)$ \tab (billiger als A = LR)
	
	\subsection{Least-Squares-Lösung}
	
	Es existiert immer eine "Lösung" $\bar{x} \in \mathbb{R}^n$ mit kleinsten Fehlerquadraten
	[Editor Note: Least Squares hat den Namen, dass man das A von $A \in \mathbb{R}^{m \times n} Ax = b$ quadriert, um ein $\tilde{A} = AA^t$ hat, von dem man eine Lösung $\tilde{x}$ bestimmen kann]
	
	\subsection{Aufwand QR-Zerlegung}
	
	Doppelter Aufwand für QR wie für LR
	
	$N_{QR}(n) = \frac{2}{3}n^3 + O(n^2)$
	
	\subsection{Householder Transformation}
	
	$v \in \mathbb{K}$, $||v||_2 = 1$
	
	$H = E_n - 2 \frac{vv^t}{v^tv}$
	
	Spiegelung eines Vektors an einer Hyperebene durch Null im euklidischen Raum $H_{\overrightarrow{x}} = x - 2vv^tx$
	
	\begin{equation*}
		v\conj{v}^t =
		\begin{pmatrix}
			v_1\conj{v}_1 & \dotsc & v_1\conj{v}_n \\
			\vdots & \ddots & \vdots \\
			v_n\conj{v}_1 & \dotsc & v_n\conj{v}_n
		\end{pmatrix}
	\end{equation*}
	
	\subsection{Eigenschaften von Householder}
	
	Symmetrisch: $H = H^t$
	
	Orthogonal: $H^tH = E_n$
	
	Involutorisch: $h^2 = E_n$
	
	EW: -1 einfach, 1 (n - 1)-fach
	
	Mantisse-Vektor-Multiplikationen mit Householder sind schnell berechenbar
	
	\subsection{Householder-Verfahren}
	
	Gruppe numerischer Verfahren zur Bestimmung von Nullstellen einer skalaren, reellen Funktion
	
	Zur Berechnung der QR-Zerlegung
	
	$S_tA^{(t - 1)} = A^{(t)}$, $S_t$ ist Householder-Transformation
	
	\subsection{Ergebnis}
	
	QR-Zerlegung von A (aber \underline{nicht}eindeutig!)
	
	\subsection{Code Intervallschachtelung/Bisektionsverfahren}
	
	for k = 0, 1, ... do:
	
	\tab x[k] = 0.5(a[k] + b[k])
	
	\tab if f(a[k])f(x[k]) < 0:
	
	\tab \tab a[k + 1] = a[k]
	
	\tab \tab b[k + 1] = x[k]
	
	\tab else:
	
	\tab \tab a[k + 1] = x[k]
	
	\tab \tab b[k + 1] = b[k]
	
	\tab if abs(b[k + 1] - a[k + 1]) \textless TOL abs(a[k + 1]):
	
	\tab \tab Ende Lösung: 0.5(b[k + 1] + a[k + 1])
	
	\section{Konvergenz interativer Methoden}
	
	\subsection{Konvergenzordnung}
	
	$\alpha > 1$
	
	$|x_{k + 1} - x_*| \le C|x_k - x_*|^\alpha$ \tab k = 0, 1, ...
	
	\subsection{Lineare Konvergenz}
	
	$\alpha = 1 \Rightarrow C lineare Kontraktionsrate$
	
	\subsection{Superlineare Konvergenz}
	
	$|x_{k + 1} - x_k| \le C_k |x_k - x_*|$ \tab mit $c_k \rightarrow 0$
	
	\subsection{Quadratische Konvergenz}
	
	$|x_{k + 1} - x_*| \le C |x_k - x_*|$ \tab also $\alpha = 2$
	
	Konvergenz der Ordnung $\alpha$ (quadratische Konvergenz, kubische, etc.) $\Rightarrow$ superlineare Konvergenz $\Rightarrow$ lineare Konvergenz
	
	\section{Newton Verfahren im \texorpdfstring{$\mathbb{R}^n$}{R pow(n)}}
	
	TODO
	
	\section{Newton-Kantorisch}
	
	\subsection{Voraussetzungen}
	
	$f: x \subset \mathbb{R}^n \rightarrow \mathbb{R}^n$ mit f' L-stetig, d.h.
	
	1. $|| j_f(x) - j_f(y)|| \le L||x - y||$
	
	2. $||j_f^{-1}(x)|| \le \beta$
	
	3. $x_0 \in D_f(x)$
	
	4. $q := \frac{1}{2} \alpha \beta L$ mit $\alpha = ||j_f^{-1}(x_0)f(x_0)||$ auf der Nieveaumenge $D(x) = {y \in D | ||f(x) || \le ||x||}$
	
	$\Rightarrow$ Dann konvergiert $(x_k)$ quadratisch gegen Nullstelle $z \in D$ von f
	
	\subsection{Fehlerabschätzung}
	
	$||x_k - x_*|| \le \frac{\alpha}{1 - q}q^{(2^k - 1)}$ \tab , $k \ge 1$
	
	\section{Sukzessive Approximation *}
	
	$x_{t + 1} = x_t + C^{-1} f(x_t)$ \tab C reguläre Matrix $\in \mathbb{R}^{n \times n}$
	
	$x_t \xrightarrow[t \rightarrow \infty]{} z \Rightarrow z = z + C^{-1}f(z)$ bzw $f(z) = 0 \Rightarrow$ z Fixpunkt der Abbildung $g(x) := x + C^{-1}f(x)$
	
	\subsection{Fixpunktiteration}
	
	für sehr große n ist Gauß-Elimination zu speicherintensiv
	
	$Ax = b \Leftrightarrow a_{j, j} * x_j + \sum\limits_{k = 1}^na_{j, k}x_k = b_j$
	
	\section{Jacobi *, Gauß-Seidel *, SOR *}
	
	\subsection{Jacobi-Verfahren}
	
	$B = J = -D^{-1}(L + R)$ mit A = D + L + R
	
	\subsection{Gauß-Seidel}
	
	$B = H_1 = -(D + L)^{-1}R$
	
	\subsection{SOR}
	
	$H_w = (D + wL)^{-1}((1 - w)D - wR)$
	
	(für w = 1: Gauß-Seidel)
	
	\subsection{Konvergenz}
	
	für positiv definiti symetrische Matrizen:
	
	(I) starke/strikte Diagonaldominanz $\Rightarrow$ J/G-S konvergiert
	
	(II) diagonaldominanz + ireduzibel\footnote{Knotenmodell} $\Rightarrow$ J/G-S konvergiert
	
	\begin{tabular}{l l l}
		(III) Beachte EW von: & $J = -D^{-1}(L + R)$ & J konvergiert für $spr(J) < 1$ \\
		& $H = -(D + L)^{-1}R$ & G-S konvergiert für $spr(H) < 1$
	\end{tabular}
	
	\subsection{Fehler verringern}
	
	Wie viele Iterationsschritte t bis Fehler um $10^k$ verbessert ist?
	
	$t \ge -\frac{k}{\log_{10}f}$ \tab f = spr(B)\section{Jacobi *, Gauß-Seidel *, SOR *}
	
	\subsection{Jacobi-Verfahren}
	
	$B = J = -D^{-1}(L + R)$ mit A = D + L + R
	
	\subsection{Gauß-Seidel}
	
	$B = H_1 = -(D + L)^{-1}R$
	
	\subsection{SOR}
	
	$H_w = (D + wL)^{-1}((1 - w)D - wR)$
	
	(für w = 1: Gauß-Seidel)
	
	\subsection{Konvergenz}
	
	für positiv definiti symetrische Matrizen:
	
	(I) starke/strikte Diagonaldominanz $\Rightarrow$ J/G-S konvergiert
	
	(II) diagonaldominanz + ireduzibel $\Rightarrow$ J/G-S konvergiert
	
	\begin{tabular}{l l l}
		(III) Beachte EW von: & $J = -D^{-1}(L + R)$ & J konvergiert für $spr(J) < 1$ \\
		& $H = -(D + L)^{-1}R$ & G-S konvergiert für $spr(H) < 1$
	\end{tabular}
	
	\subsection{Fehler verringern}
	
	Wie viele Iterationsschritte t bis Fehler um $10^k$ verbessert ist?
	
	$t \ge -\frac{k}{\log_{10}f}$ \tab f = spr(B)
	
	\section{Allgemeines Abstiegsverfahren}
	
	TODO
	
	\section{Gradientenverfahren}
	
	TODO
	
	\section{CG-Verfahren}
	
	TODO
	
	\section{Vorkonditionierung}
	
	TODO
	
	\section{Satz von Gerschgorin}
	
	Alle EW von $A \ in \mathbb{K}^{n \times n}$ liegen in den Vereinigung der sogenannten Gerschgorin-Kreise
	
	$k_j = {\zeta \in \mathbb{C} | |\zeta * a_{j, j}| \le \sum\limits_{k = 1, k \ne j}^n|a_jk|}$
	
	$U = \bigcup\limits_{i = 1}^mk_{j, i}$, $V = \bigcup\limits_{j = 1}^nk_j \text{\textbackslash} U$
	
	Sind $U \cup V = \emptyset \Rightarrow$ m EW in U, n - m in V
	
	\section{Stabilitätsatz}
	
	$A \in \mathbb{K}^{n \times n}$ mit n linear unabhängig EV, $B \in \mathbb{K}^{n \times n}$
	
	$\Rightarrow \forall \lambda_B$ EW von B $\exists \lambda_A$ EW von A:
	
	$|\lambda_A - \lambda_B| \le cond_2(W) ||A - B||_2$
	
	$W = (w_1, ..., w_n) \in \mathbb{K}^{n \times n}$ EV von A
	
	\section{Potenzmethode *, inverse Iteration *}
	
	TODO
	
	\section{Hessenberg-Normalform}
	
	TODO
	
	\section{QR-Verfahren}
	
	TODO
\end{document}