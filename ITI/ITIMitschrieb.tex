\documentclass[12pt,a4paper]{article} % using article ensures it starts at 1 and does not have odd numberings for section
\usepackage{graphicx}
\usepackage[utf8]{inputenc} % this way umlaute are included from the get go
\usepackage[ngerman]{babel} % german spell check
\usepackage{datetime}

\usepackage{breqn} % this package is one option for math lines

\usepackage{hyperref} % these two lines are so that the table of content is clickable
\hypersetup{linktoc=all}

\newdateformat{gerDate}{\THEDAY \space \monthname[\THEMONTH], \THEYEAR} % german date format

\begin{document}
	\begin{titlepage} % good tital page template, only needs the class X notes part to be changed for each new class
	\centering
	\includegraphics[width=0.40\textwidth]{UniLogo}\par\vspace{1cm}
		{\scshape\LARGE Universität Heidelberg \par}
		\vspace{1cm}
		{\huge\bfseries Einführung in die Technische Informatik \par}
		\vspace{1cm}
		{\Large\bfseries Prof. Dr. Ulrich Brüning \par}
		\vspace{1cm}
		{\huge Wintersemester 17/18 \par}
		\vspace{2cm}
		{\Large\itshape von Charles Barbret \par}
		
		\vfill

		% Bottom of the page
		{\large \gerDate\today\par}
	\end{titlepage}

\tableofcontents % creats a table of contents, ensured already that it is clickable
\newpage % starts the actual document on a new page so there is no weird colision of text and toc


\section{Organisatorisches}
\subsection{Moodle}
Passwort: EinfTI17

\subsection{Wo wir hin wollen}
Wie ein Conputer aufgebaut ist, von der Hardware.

Das Skript wird im Moodle zur Verfügung gestellt werden

\subsection{Übungsgruppen}
Die Ersten 3 Wochen findet keine Übungsgruppe statt, die ersten zwei Wochen werden Zettel zur Bearbeitung freigegeben werden. 

Um zur Klausur zugelassen werden muss man mehr als 50\% der Punkte erreicht werden, wobei man maximal 2 mal bei der Übungsgruppe fehlen darf. Es müssen ebenfalls 80\% der Zettel abgegeben werden. 

\subsection{Klausur}
Erste Klausur ist 2 Wochen nach der letzten Vorlesung. (Mittwoch 21.2.2018)

\newpage
\section{Vorlesung 2}
\subsection{Information und ihre Übertragung}
Eine Nachricht ist eine Zeichenfolge, wobei ein Zeichen aus einer endlichen Menge von unterscheidbaren Elementen stammt. 

Um zu vermitteln, dass nichts gesendet wird, wird etwas versendet, was mitteilt, dass nichts gesendet wird.




\end{document}