\documentclass[12pt,a4paper]{article} % using article ensures it starts at 1 and does not have odd numberings for section
\usepackage{graphicx}
\usepackage[T1]{fontenc}
\usepackage[utf8]{inputenc} % this way umlaute are included from the get go
\usepackage[ngerman]{babel} % german spell check
\usepackage{lmodern}
\usepackage{datetime}

\usepackage{amsmath} % this package is one option for math lines
\usepackage{tasks}

\usepackage{hyperref} % these two lines are so that the table of content is clickable
\usepackage{amssymb} % package for Natural Number sign etc
\hypersetup{linktoc=all}

\begin{document}
	\begin{description}
		\item[Aussagenlogik (AL)]:
		\begin{itemize}
			\item Syntax vs. Semantik
			\item syntaktische und semantische Grundbegriffe der AL (Junktoren, Formeln, Belegungen, Wahrheitsfunktionen (= Boolesche Funktionen))
			\item zentrale logische Begriffe (Erfüllbarkeit, Allgemeingültigkeit, Fol-gerungs- und Äquivalenzbegriff) und Zusammenhänge zwischen diesen Begriffen
			\item Zusammenhang Formeln - Boolesche Funktionen: von einer Formel definierte Boolesche Funktion, Verfahren zur Bestimmung der disjunktiven Normalform (DNF) und konjunktiven Normalform (KNF), aussagenlogische vs. Boolesche Formeln, Basen der Booleschen Funktionen.
		\end{itemize}
		\item[Kalküle]:
		\begin{itemize}
			\item Idee, Anforderungen an und Bestandteile von Kalkülen, Beweise und Beweisbarkeit (aus T), elementare Eigenschaften von Beweisen und Beweisbarkeit, Korrektheit und Vollständigkeit logischer Kalküle, Korrektheitslemma
			\item Korrektheit (mit Beweisidee) und Vollständigkeit des Shoenfield-Kalküls für AL
			\item Kompaktheitssatz für AL. 
		\end{itemize}
		\item[Syntax und Semantik der Prädikatenlogik erster Stufe]: 		
		\begin{itemize}
			\item syntaktische und semantische Grundbegriffe der PL (Strukturen und deren Signatur, Sprachen und deren Signatur
			\item Terme, Formeln und Sätze und deren Interpretation in Strukturen
			\item Theorien, Modellbegriff (Modelle und Modellklassen))
			\item zentrale logische Begriffe (Erfüllbarkeit (von Formeln und Formelmengen), Allgemeingültigkeit, Folgerungs- und Äquivalenzbegriff, Tautologien = aussagenlogisch gültige Formeln) und Zusammenhänge zwischen diesen Begriffen.
		\end{itemize}
		\item[Der Shoenfield-Kalkül für PL]:
		\begin{itemize}
			\item Korrektheitssatz: Beweisidee und Folgerungen (Konsistenzlemma)
			\item Vollständigkeitssatz: was versteht man unter einer zulässigen Regel bzw.
			einem zulässigen Axiom?, was besagt das Deduktionstheorem?, Erfüllbarkeitslemma (EL),
			wie folgt der Vollständigkeitssatz aus dem EL?, Beweisidee des EL (was ist eine
			Termstruktur?, Lemma und Satz über Termstrukturen, was ist eine Henkin-Theorie?, Aussage der Sätze von Lindenbaum und über Henkin-Erweiterungen)
			\item Kompaktheitssatz (für Folgerungsbegriff und für die Erfüllbarkeit - mit Beweis).
		\end{itemize}
		\item[Definierbarkeit in PL1]:
		\begin{itemize}
			\item Elementare und $\Delta$-elementare Strukturklassen - Eigenschaften
			hiervon
			\item Beispiele hierzu (Mengen unterschiedlicher Mächtigkeiten, Ordnungen, Gruppen
			und Körper; Arithmetik)
			\item Isomorphie und elementare Äquivalenz
			\item negative Definierbarkeitsergebnisse: z.B. Nicht-$\Delta$-Ele-mentarität der endlichen Strukturen (Satz über die Existenz unendlicher Modelle), Nicht-Elementarität der unendlichen Strukturen, Nicht-$\Delta$-Elementari-tät der Wohlordnungen, sowie der Satz von Skolem (jeweils mit Beweisideen)
			\item die Prädikatenlogik 2. Stufe
			\item Definierbarkeit in PL2 (Endlichkeit und Arithmetik - Beweisidee!)
			\item warum gibt es keinen Kalkül für PL2 (Begründung)?
		\end{itemize}
	\end{description}
	
	\newpage
	\tableofcontents
	\newpage
	
	\section{Aussagenlogik}
	\subsection{*Sprache der Aussagenlogik}
	\label{ALSprache}
	Besteht aus 3 Bausteinen:
	
	\begin{tabular}{l l}
		1) & Aussagenvariablen: $A_0, A_1, ...$ \\
		2) & \begin{tabular}{l l}
			1-Stellig: & $\neg$ \\
			2-Stellig: & $\land, \lor, \rightarrow \leftrightarrow$
		\end{tabular} \\
		3) & Klammern
	\end{tabular}

	Das Alphabet $A_{AL}$ ist die Menge der Symbole
	
	$A^*$ ist die Menge der Wörter (wobei jedes Wort endlich viele Symbole hat) und das leere Wort $\lambda$
	
	\subsection{*Formeln}
	\label{Formel}
	(F1) Alle $A_{n_{> 0}}$ sind Formeln \newline
	(F2) $\varphi$ al. Formel $\Rightarrow \neg \varphi$ al. Formel \newline
	(F3) $\varphi_1, \varphi_2$ al. Formeln $\Rightarrow (\varphi_1 \bigodot \varphi_2)$ mit $\bigodot \in \{\land, \lor, \rightarrow\leftrightarrow\}$ al. Formel
	
	\begin{tabular}{l p{10cm}}
		Länge von $\varphi$: & $l(\varphi)$ := Anzahl der Zeichen \\
		& $lz(\varphi)$ := Anzahl der Junktoren \\
		Anderes zu $\varphi$ & $\rho(\varphi)$ := Schachtelungstiefe der Junktoren \\
		& $V(\varphi)$ := Anzahl der Aussagenvariablen \\
		& $TF(\varphi)$ := Anzahl der Teilformeln \\
		& F(V) := $\{\varphi: V(\varphi)\subseteq V\}$ Menge der al. Formeln, die Variablen aus V enthalten
	\end{tabular}
	
	\subsection{*Prinzip der syntaktischen Induktion}
	\begin{tabular}{l l}
		(i) & E Eigenschaft trifft auf jede Aussagenvariable A zu \\
		& Heißt wir Zeigen es gilt für Aussagenvariablen \\
		(ii) & Trifft E auf al. \hyperref[Formel]{Formel} $\varphi$ zu $\Rightarrow$ E trifft auf $\neg \varphi$ zu \\
		& E gilt für die \hyperref[Formel]{Formel} und dessen Negation \\
		(iii) & E trifft auf $\varphi_1, \varphi_2$ zu $\Rightarrow$ E trifft auf $(\varphi_1 \bigodot \varphi_2)$ mit $\bigodot \in \{\land, \lor, \rightarrow\leftrightarrow\}$ zu \\
		& Gilt E für $\varphi_1, \varphi_2$ so gilt es auch für deren Verknüpfungen
	\end{tabular}
	
	\subsection{*!Belegung}
	\label{Belegung}
	Die Belegung B der Variablenmenge V ist Abbildung $B: V \rightarrow \{0, 1\}$
	
	B(V) Menge aller Belegungen von V
	
	$|B(V)| = 2^{|V|}$
	
	\subsection{*Bewertung}
	Die zur Belegung $B: V \rightarrow \{0, 1\}$ gehörende Bewertung $\widehat{B}$ 
	
	1. $\varphi \equiv A : \widehat{B}(A) := B(A)$ \newline
	2. $\varphi \equiv \neg \psi : \widehat{B}(\neg \psi):= f_{\neg}(\widehat{B}(\psi))$ \newline
	3. $\varphi \equiv (\varphi_1 \bigodot \varphi_2) := f_{\widehat{B}}(\widehat{B}(\varphi_1), \widehat{B}(\varphi_2))$
	
	\subsection{*!Koinzidenzlemma}
	Seien $B_i : V_i \rightarrow \{0, 1\}, i = 0, 1$, \hyperref[Belegung]{Belegung}, sei $\varphi$ eine al. \hyperref[Formel]{Formel} deren Aussagenvariable in $V_0$ und $V_1$ liegen und stimmen $B_0$ und $B_1$ auf den in $\varphi$ vorkommenden Variablen überein (Was soviel heißt wie jede Variable aus $V_0$ wird durch $B_0$ einen Wert zugewiesen, und jede Variable aus $V_1$ wird durch $B_1$ einen Wert zugewiesen. Jetzt müssen die Variablen in $\varphi$ vorkommen in $V_0$ und $V_1$ vorkommen und deren \hyperref[Belegung]{Belegungen} müssen gleich sein.)
	
	$\Rightarrow \widehat{B}_0(\varphi) = \widehat{B}_1(\varphi)$
	
	\subsection{*Zentrale semantische Begriffe}
	\begin{tabular}{l l}
		$\varphi$ allgemeingültig/$ag[\varphi]$ & $\Leftrightarrow \forall$ \hyperref[Belegung]{Belegungen} B: $V(\varphi) \rightarrow \{0, 1\}$ gilt $B(\varphi) = 1$ \\
		$\varphi$ erfüllbar/$erfb[\varphi]$ & $\Leftrightarrow \exists$ \hyperref[Belegung]{Belegungen} B: $V(\varphi) \rightarrow \{0, 1\}$ mit $B(\varphi) = 1$ \\
		$\varphi$ kontradiktorisch/$kd[\varphi]$ & $\Leftrightarrow \forall$ \hyperref[Belegung]{Belegungen} B: $V(\varphi) \rightarrow \{0, 1\}$ gilt $B(\varphi) = 0$ 
	\end{tabular}
	
	\subsection{*Boolesche Gesetze}
	\begin{tabular}{l l l}
		1. & $A \land B$ äq $B \land A$ und $A \lor B$ äq $B \lor A$ & Kommutativgesetz \\
		2. & $A \land(B \land C)$ äq $(A \land B) \land C$ & Assoziativgesetz \\
		& $A \lor (B \lor C)$ äq $(A \lor B) \lor C$ & \\
		3. & $A \lor A$ äq $A$ äq $A \land A$ & Idempotenz \\
		4. & $A \land (B \lor C)$ äq $(A \land B) \lor (A \land C)$ & Distributivgesetz \\
		& $A \lor (B \land C)$ äq $(A \lor B) \land (A \lor C)$ & \\
		5. & $A \land (A \lor B)$ äq $A$ äq $A \lor (A \land B)$ & Absorptionsgesetz \\
		6. & $\neg (A \land B)$ äq $\neg A \lor \neg B$ & De Morgan \\
		& $\neg (A \lor B)$ äq $\neg A \land \neg B$ &
	\end{tabular}

	\subsection{!Einsetzungsregel}
	Ist al. \hyperref[Formel]{Formel} $\varphi$ allgemeingültig, dann auch $\varphi[\psi/A]$, die aus $\varphi$ durch ersetzen aller vorkommenden A in $\varphi$ durch $\psi$
	
	\subsection{!Ersetzungsregel}
	$\varphi \leftrightarrow \psi$ allgemeingültig $\Rightarrow \chi \leftrightarrow \chi (\psi/\varphi)$ allgemeingültig
	
	\subsection{Substitutionsregel}
	$sub(\chi, \varphi, \psi)$ Menge aller Varianten $\chi(\psi/\varphi)$
	
	Ist $\varphi \equiv \chi \Rightarrow sub(\chi, \varphi, \psi) = \{\chi, \psi\}$
	
	\begin{tabular}{l l l}
		sonst & $\chi \equiv A$ & $sub(\chi, \varphi, \psi) = \{\chi\}$ ($\varphi$ keine Teilformel von $\chi$) \\
		& $\chi \equiv \neg \chi_1$ & $sub(\chi, \varphi, \psi) = \{\neg \chi_1^*: \chi_1^* \in sub(\chi_1, \varphi, \psi)\}$ \\
		& $\chi \equiv \chi_1 \bigodot \chi_2$ & $sub(\chi, \varphi, \psi) = \{\chi_1^* \bigodot \chi_2^*: \chi_i \in sub(\chi_1, \varphi, \psi)\}$
	\end{tabular}
	
	\subsection{*!Erfüllbarkeit von Formelmengen}
	\label{Erfullbar}
	$T \ne \emptyset$ nicht leere Menge al. \hyperref[Formel]{Formeln} mit Variablenmenge V(T)
	
	(i) \hyperref[Belegung]{Belegung} B: $V(T) \rightarrow\{0, 1\}$ macht T wahr ($B \vDash T$), falls B alle $\varphi \in T$ wahr macht, d.h. $\forall \varphi \in T: B(\varphi) = 1$
	
	(ii) T erfüllbar $(erfb[T]) \Leftrightarrow \exists B$ von V(T), T wahrmacht \newline
	$\Rightarrow erfb[T] \Leftrightarrow \exists B \in B(V(T)): B \vDash T$ \newline
	$\Leftrightarrow \exists B \in B(V(T)) \forall \varphi \in T: B(\varphi) = 1$ \newline
	$\Leftrightarrow T = \{\varphi_1, ..., \varphi_n\}: erfb[T] \Leftrightarrow erfb[\varphi_1 \land ... \land \varphi_n]$
	
	\subsection{*!Semantischer Folgerungsbegriff}
	Eine al. \hyperref[Formel]{Formel} $\varphi$ folgt aus einer Menge T von al. \hyperref[Formel]{Formeln} $(T \hyperref[Erfullbar]{\vDash} \varphi) \Leftrightarrow$ jede \hyperref[Belegung]{Belegung} $B \in V(T) \cup V(\varphi)$, die T wahr macht, macht auch $\varphi$ wahr, d.h. $\forall B \in B(V(T) \cup V(\varphi) [B \hyperref[Erfullbar]{\vDash} T \Rightarrow B \hyperref[Erfullbar]{\vDash} \varphi]$ 
	
	\subsection{Find A Stupid Name}
	$T = \{\varphi\}$ Einelementig: $\{\varphi\} \hyperref[Erfullbar]{\vDash} \psi \Leftrightarrow \varphi$ impl. $\psi$ \newline
	$\{\varphi_1, ..., \varphi_n\} \hyperref[Erfullbar]{\vDash} \psi$ kürzer geschrieben: \newline $\varphi_1, ..., \varphi_n \hyperref[Erfullbar]{\vDash} \psi: \varphi_1, ..., \varphi_n \hyperref[Erfullbar]{\vDash} \psi \Leftrightarrow \varphi_1 \land ... \land \varphi_n$ impl. $\psi$
	
	\subsection[Monotonie]{Monotonie weil keine Besseren Namen}
	Der semantische Folgerungsbegriff ist monoton, d.h. \newline
	$T \subseteq T'\text{ und } T \hyperref[Erfullbar]{\vDash} \varphi \Rightarrow T' \hyperref[Erfullbar]{\vDash} \varphi$
	
	\subsection{*!Folgerung vs Erfüllbarkeit}
	(i) $T \hyperref[Erfullbar]{\vDash} \varphi \Leftrightarrow$ nicht $erfb[T \cup \{\neg \varphi\}]$ \newline
	(ii) $T \nvDash \varphi \Leftrightarrow erfb[T \cup \{\neg \varphi\}]$
	
	\subsection{*Erfüllbarkeit vs. Semantische Folgerung}
	Für $T \ne \emptyset$ sind äquivalent:
	
	\begin{tabular}{l l}
		(i) & $erfb[T]$ \\
		(ii) & $\nexists \varphi[T \hyperref[Erfullbar]{\vDash} \varphi \text{ und } T \hyperref[Erfullbar]{\vDash} \neg \varphi]$ \\
		(iii) & $\exists \varphi [T \nvDash \varphi]$
	\end{tabular}

	\subsection{get Fucked}
	$\varphi$ al. \hyperref[Formel]{Formel} mit $V(\varphi) \subseteq \{A_0, ..., A_{n - 1}\}$
	
	Die von $\varphi$ dargestellte n-Stellige Boolesche Funktion $f_{\varphi, n}$ ist definiert durch $f_{\varphi, n}(i_0, ..., i_{n-1}) = \widehat{B}_{i_0, ..., i_{n - 1}}(\varphi)$
	
	mit $B_{i_0, ..., i_{n - 1}}: \{A_0, ..., A_{n - 1}\} \rightarrow \{0, 1\}, B_{i_0, ..., i_{n - 1}}(A_j) = i_j$ $(j = 0, ..., n - 1)$
	
	\subsection{*DNF}
	\label{DNF}
	Eine Boolesche \hyperref[Formel]{Formel} $\varphi$ ist in disjunktiver Normalform (DNF), wenn
	$\varphi$ die endliche Disjunktion von $\land$-Klauseln ist: $\varphi \equiv \kappa_1 \lor ... \lor \kappa_m (m \ge 1)$
	
	Allgemeines Verfahren um DNF anzugeben: Tabelle Aufstellen, alle Terme $x_0$ bis $x_n$ ver-und-en, wobei wenn der Term $x_i$ 0 ist wird die Negation verwendet wird
	
	Beispiel:
	
	\begin{tabular}{l | l | l}
		$x_0$ & $x_1$ & \hyperref[Formel]{Formel} \\ \hline
		0 & 0 & $\neg x_0 \land \neg x_1$ \\
		0 & 1 & $\neg x_0 \land x_1$ \\
		1 & 0 & $x_0 \land \neg x_1$ \\
		1 & 1 & $x_0 \land x_1$ \\
	\end{tabular}
	
	\subsection{*KNF}
	Boolesche \hyperref[Formel]{Formel} $\varphi$ ist in KNF, wenn $\varphi$ endliche Konjunktion von $\lor$-Klauseln, $\varphi \equiv S_1 \land ... \land S_m$ mit $m \ge 1$
	
	Im großen und ganzen nur cooles Trivia wissen, man braucht fast immer die \hyperref[DNF]{DNF}
	
	\subsection{*Darstellungssatz}
	Jede n-stellige Boolesche Funktion kann effektiv als eine \hyperref[Formel]{Formel} $\varphi$ in \hyperref[DNF]{DNF} dargestellt werden. 
	
	Heißt man hat eine Funktion, nachdem man die Variablen eingesetzt hat, findet man eine \hyperref[Formel]{Formel} $\varphi$ in \hyperref[DNF]{DNF} sodass $f_{\varphi, n}(i_0, ..., i_{n-1}) = \widehat{B}_{i_0, ..., i_{n - 1}}(\varphi)$
	
	Analog KNF

	\subsection{Folgerung aus Darstellungssatz}
	Basis der Booleschen Funktionen: \newline
	Menge $\{f_1, ..., f_k\}$ von Booleschen Funktionen, sodass sich jede Boolesche Funktion über $
	f_1, ..., f_k$ definieren lässt. \newline
	Basis M ist minimal $\Leftrightarrow$ keine echte Teilmenge von M eine Basis ist
	
	\subsection{1. Basissatz}
	Die Booleschen Funktionen $\{\neg, \land, \lor\}$ bilden eine Basis der Booleschen Funktionen
	
	\subsection{2. Basissatz}
	Folgende Mengen sind Basen der Booleschen Funktionen: \newline
	\begin{tabular}{l l}
		i) & $\{\neg, \lor\}$ \\
		ii) & $\{\neg, \land\}$ \\
		iii) & \{NOR\} \\
		iv) & \{NAND\}
	\end{tabular}
	
	\subsection{Normalformsatz}
	Zu jeder al. \hyperref[Formel]{Formel} $\varphi$ kann man eine äquivalente \hyperref[Formel]{Formel} $\varphi_{DNF}$ in \hyperref[DNF]{DNF} , sodass: $V(\varphi) = V(\varphi_{DNF})$
	
	\subsection{*Kalkül $\mathcal{K}$}
	\begin{itemize}
		\item \hyperref[ALSprache]{Sprache} von $\mathcal{K}$, vom Alphabet von festgelegt 
		\item Menge der \hyperref[Formel]{Formeln} von $\mathcal{K}$, Teilmenge der Wörter über Alphabet von$\mathcal{K}$
		\item Menge der Axiome von $\mathcal{K}$, Teilmenge der Menge der \hyperref[Formel]{Formeln} von $\mathcal{K}$
		\item Menge der Regeln der Gestalt \newline
		(R) $\frac{\varphi_1, ..., \varphi_n}{\varphi} n \ge 1, \varphi_1, ..., \varphi_n$ \hyperref[Formel]{Formeln} von $\mathcal{K}$\newline
		$\varphi_1, ..., \varphi_n$ Prämissen, $\varphi$ Konklusion
	\end{itemize}

	\subsection{*Beweise und Beweisbarkeit}
	Sei $\mathcal{K}$ ein Kalkül aus T Menge von ($\mathcal{K}$-)\hyperref[Formel]{Formeln} ein ($\mathcal{K}$-)Beweis der ($\mathcal{K}$-)\hyperref[Formel]{Formel} $\varphi$ aus T ist endliche Folge $\psi_1, ..., \psi_n$ von ($\mathcal{K}$-)\hyperref[Formel]{Formeln}, sodass gilt:
	\begin{itemize}
		\item $\varphi \equiv \psi_n$
		\item $\forall \psi_m, 1 \le m \le n:$ \begin{itemize}
		\item ist ($\mathcal{K}$-)Axiom oder
		\item Eine \hyperref[Formel]{Formel} aus der Formelmenge T oder
		\item Konklusion einer ($\mathcal{K}$-)Regel R, deren Prämissen in $\{\psi_1, ..., \psi_{m - 1}\}$ liegen
		\end{itemize}
		\item n ist die Länge des Beweises $\psi_1, ..., \psi_n$
	\end{itemize}

	\subsection{*Beweisbar}
	\label{Beweisbar}
	Eine ($\mathcal{K}$-) \hyperref[Formel]{Formel} $\varphi$ is ($\mathcal{K}$-) beweisbar aus T, wenn es einen ($\mathcal{K}$-) Beweis von $\varphi$ aus T gibt.
	
	NB: Jede \hyperref[Formel]{Formel} $\varphi \in T$ ist ein ($\mathcal{K}$-) Beweis (der Länge 1) aus T und damit
	($\mathcal{K}$-) beweisbar aus T.
	$\hyperref[Erfullbar]{\vDash}$ = beweisbar $\Rightarrow \varphi$ ist ($\mathcal{K}$-) beweisbar aus T $\Leftrightarrow T \hyperref[Erfullbar]{\vDash} K \varphi$
	
	Außerdem gilt:
	
	\begin{itemize}
		\item Monotonie: Falls $T \subseteq T'$ und $T \hyperref[Beweisbar]{\vdash} \varphi \Rightarrow T' \hyperref[Beweisbar]{\vdash} \varphi$
		\item Transitivität: Gelte $T \hyperref[Beweisbar]{\vdash} \varphi$ und $T' \hyperref[Beweisbar]{\vdash} \psi \forall \psi \in T \Rightarrow T' \hyperref[Beweisbar]{\vdash} \varphi$ 
		\item Endlichkeit: Falls $T \hyperref[Beweisbar]{\vdash} \varphi \Rightarrow \exists$ endliche Teilmenge $T_0$ von T mit $T_0 \hyperref[Beweisbar]{\vdash} \varphi$
	\end{itemize}

	\subsection{*Konsistent}
	\label{ALKonsistent}
	Kalkül $\mathcal{K}$ ist Konsistent, falls es eine $\mathcal{K}$-\hyperref[Formel]{Formel} $\psi \text{ mit } \nvdash \psi$ gibt
	
	\subsection{*Kalkül der Aussagenlogik}
	\begin{itemize}
		\item \hyperref[ALSprache]{Sprache} basiert auf Alphabet $\mathcal{A} = \{A_0, A_1, ..., j_1, ..., j_k, (, )\}$ Junktoren $j_1, .., j_k$ Basis der Booleschen Funktionen
		\item (F1) Aussagenvariable $A_i (i \ge 0)$ ist eine \hyperref[Formel]{Formel} \newline
		(F2) * 1-Stelliger Junktor, $\varphi$ \hyperref[Formel]{Formel} $\Rightarrow *\varphi$ ist eine \hyperref[Formel]{Formel} \newline
		(F3) * 2-Stelliger Junktor, $\varphi_1, \varphi_2$ \hyperref[Formel]{Formeln} $\Rightarrow (\varphi_1 * \varphi_2)$ ist eine \hyperref[Formel]{Formel}
	\end{itemize}

	\subsection{*Korrekt und Vollständig}
	\label{ALVollstandig}
	$\mathcal{K}$ Kalkül der Aussagenlogik
	\begin{itemize}
		\item $\mathcal{K}$ ist korrekt bezüglich der Allgemeingültigkeit, falls jede $\mathcal{K}$-beweisbare \hyperref[Formel]{Formel} $\varphi$ allgemeingültig ist, d.h. $\hyperref[Beweisbar]{\vdash_{\mathcal{K}}} \varphi \Rightarrow \hyperref[Erfullbar]{\vDash} \varphi \forall \mathcal{K}$-\hyperref[Formel]{Formeln} $\varphi$ 
		\item $\mathcal{K}$ ist korrekt bezüglich Folgerungen, falls $T \hyperref[Beweisbar]{\vdash_{\mathcal{K}}} \varphi \Rightarrow T \hyperref[Erfullbar]{\vDash} \varphi$ für alle $\mathcal{K}$-Formelmengen T und $\mathcal{K}$-\hyperref[Formel]{Formeln} $\varphi$
		\item $\mathcal{K}$ ist vollständig bezüglich der Allgemeingültigkeit, falls $\hyperref[Erfullbar]{\vDash} \varphi \Rightarrow \hyperref[Beweisbar]{\vdash_{\mathcal{K}}} \varphi$ für alle $\mathcal{K}$-\hyperref[Formel]{Formeln} $\varphi$
		\item $\mathcal{K}$ ist vollständig bezüglich Folgerungen, falls $\hyperref[Erfullbar]{\vDash} \varphi \Rightarrow \hyperref[Beweisbar]{\vdash_{\mathcal{K}}} \varphi$ für alle $\mathcal{K}$-Formelmengen T und $\mathcal{K}$-\hyperref[Formel]{Formeln} $\varphi$
		\item $\mathcal{K}$ ist adäquat, wenn $\mathcal{K}$ korrekt und vollständig ist, d.h. $\hyperref[Beweisbar]{\vdash_{\mathcal{K}}} \varphi \Leftrightarrow T \hyperref[Erfullbar]{\vDash} \varphi$
	\end{itemize}

	\subsection{*Korrektheitslemma}
	Sei $\mathcal{K}$ ein Kalkül der Aussagenlogik, dessen Axiome allgemeingültig sind und dessen Regeln korrekt bezüglich Folgerungen sind $\Rightarrow \mathcal{K}$ Korrekt bezüglich Folgerungen
	
	\subsection{Ein/Ersetzungsregel}
	Sei R korrekt bezüglich Folgerungen $\Rightarrow$ R korrekt bezüglich Allgemeingültigkeit
	
	\begin{tabular}{l l}
		Einsetzungsregel: & (EIN) $\frac{\varphi}{\varphi[\psi/x]}$ \\
		Ersetzungsregel: & (ERS) $\frac{\chi}{\chi[\varphi/\psi]}$ falls $\varphi$ äq $\psi$
	\end{tabular}

	\subsection{*Shoenfield}
	Shoenfield Kalkül S
	\begin{itemize}
		\item Basis: $\{\neg, \lor\}$, $\mathcal{A} = \{A_0, A_1, ..., \neg, \lor, (, )\}$
		\item Identitäten: \begin{tasks}[counter-format=(tsk[r]), label-width=4ex]
			\task $(\varphi \land \psi) :\equiv \neg (\neg \varphi \lor \neg \psi)$ 
			\task $(\varphi \rightarrow \psi) :\equiv (\neg \varphi \lor \psi)$
			\task $(\varphi \leftrightarrow \psi) :\equiv ((\varphi \rightarrow \psi) \land (\psi \rightarrow \varphi))$
		\end{tasks}
		\item Axiom: (Ax) $\neg \varphi \lor \varphi (\equiv \varphi \rightarrow \varphi)$
		\item Regeln: \begin{description}
			\item[Expansion (E)] $\frac{\psi}{\varphi \lor \psi}$
			\item[Assoziativität (A)] $\frac{\varphi \lor (\psi \lor \delta)}{(\varphi \lor \psi) \lor \delta}$
			\item [Kürzung (Kü)] $\frac{\varphi \lor \varphi}{\varphi}$
			\item [Schnitt (S)] $\frac{\varphi \lor \psi, \neg \varphi \lor \delta}{\psi \lor \delta}$
		\end{description}
	\end{itemize}
	
	\subsection{*Korrektheitssatz}
	Shoenfield-Kalkül S ist korrekt (bezüglich Folgerungen): $T \hyperref[Beweisbar]{\vdash_S} \varphi \Rightarrow T \hyperref[Erfullbar]{\vDash} \varphi$
	
	\subsection{*Vollständigkeitssatz}
	Shoenfield-Kalkül S ist \hyperref[ALVollstandig]{vollständig} (bezüglich Folgerungen): $T \hyperref[Erfullbar]{\vDash} \varphi \Rightarrow T \hyperref[Beweisbar]{\vdash_S} \varphi$
	
	\subsection{*Zulässige Regeln}
	\label{Zulassig}
	$\mathcal{K}$, $\mathcal{K}$' Kalküle mit identischen Formelmengen, $\mathcal{K}$' heißt Erweiterung von $\mathcal{K}$ ($\mathcal{K} \subseteq \mathcal{K}$', wenn alle Axiome und Regeln von $\mathcal{K}$ Axiome und Regeln von $\mathcal{K}$' sind \newline
	Eine Erweiterung $\mathcal{K}$' ist konservativ $\Leftrightarrow \forall T \hyperref[Beweisbar]{\vdash_{\mathcal{K}'}} \varphi \Rightarrow T \hyperref[Beweisbar]{\vdash_{\mathcal{K}}} \varphi$
	
	Für $\mathcal{K} \subseteq \mathcal{K}': T \hyperref[Beweisbar]{\vdash_{\mathcal{K}}} \varphi \Rightarrow T \hyperref[Beweisbar]{\vdash_{\mathcal{K}'}} \varphi$
	
	Für $\mathcal{K} \subseteq_{\text{konservativ}} \mathcal{K}': T \hyperref[Beweisbar]{\vdash_{\mathcal{K}}} \varphi \Leftrightarrow T \hyperref[Beweisbar]{\vdash_{\mathcal{K}}} \varphi$
	
	\subsection{Regel Erweiterung}
	Eine Regel R $\frac{\varphi_1, ..., \varphi_n}{\varphi}$ ist \hyperref[Zulassig]{zulässig} in dem Kalkül $\mathcal{K}$ (ableitbar in $\mathcal{K}$), falls $\varphi_1, ..., \varphi_n \hyperref[Beweisbar]{\vdash_{\mathcal{K}}} \varphi$ \newline
	Eine \hyperref[Formel]{Formel} $\varphi$ ist ein \hyperref[Zulassig]{zulässiges Axiom} von $\mathcal{K}$ falls $\hyperref[Beweisbar]{\vdash_{\mathcal{K}}} \varphi$
	
	\subsection{Zulässige Erweiterung}
	Eine Erweiterung $\mathcal{K}$' von $\mathcal{K}$ ist eine zulässige Erweiterung, von $\mathcal{K}$ wenn jedes Axiom und jede Regel von $\mathcal{K}$' in $\mathcal{K}$ \hyperref[Zulassig]{zulässig} ist
	
	\subsection{Satz über zulässige Erweiterungen}
	$\mathcal{K}$' zulässige Erweiterung von $\mathcal{K} \Rightarrow \mathcal{K}$' ist eine konservative Erweiterung d.h. für jede Formelmenge T und \hyperref[Formel]{Formel} $\varphi: T \hyperref[Beweisbar]{\vdash_{\mathcal{K}}} \varphi \Leftrightarrow \varphi: T \hyperref[Beweisbar]{\vdash_{\mathcal{K}'}} \varphi$
	
	\subsection{Zulässige Regeln}
	\begin{description}
		\item[Kommutativität (Ko)] $\frac{\varphi \lor \psi}{\varphi \lor \psi}$
		\item[Modus Ponens (M)] $\frac{\varphi, \varphi \rightarrow \psi}{\psi}$
		\item[Verallgemeinerter Expansion (VE)] $\frac{\varphi_{i_1} \lor ... \lor \varphi_{i_m}}{\varphi_1 \lor ... \lor \varphi_n}; m, n \ge 1; \varphi_{i_1}, ..., \varphi_{i_m} \in \{\varphi_1 \lor ... \lor \varphi_n\}$
		\item[Negationsregeln:] \begin{description}
			\item[(N1)] $\frac{\varphi \lor \psi}{\neg \neg \varphi \lor \psi}$
			\item[(N2)] $\frac{\neg \neg \varphi \lor \psi}{\varphi \lor \psi}$
			\item[(N3)] $\frac{\varphi \rightarrow \delta, \psi \rightarrow \delta}{(\varphi \lor \psi) \rightarrow \delta}$
		\end{description}
	\end{description}
	
	\subsection{Tautologiesatz}
	$\hyperref[Erfullbar]{\vDash} \varphi \Rightarrow \hyperref[Beweisbar]{\vdash} \varphi$
	
	\subsection{*Vollständigkeitssatz für endlich T}
	Für endliche T gilt: $T \hyperref[Erfullbar]{\vDash} \varphi \Rightarrow T \hyperref[Beweisbar]{\vdash} \varphi$
	
	\subsection{Deductionstheorem}
	$T \hyperref[Beweisbar]{\vdash} \varphi \rightarrow \psi \Leftrightarrow T \cup \varphi \hyperref[Beweisbar]{\vdash} \psi$
	
	\subsection{Formelmenge Konsistent}
	Eine Formelmenge T ist \hyperref[ALKonsistent]{konsistent}, falls \hyperref[Formel]{Formel} $\varphi$ ex. mit: $T \nvdash \varphi$ Andernfalls ist T inkonsistent
	
	\subsection{Formelmenge Vollständig}
	Eine Formelmenge T ist vollständig, falls für jede \hyperref[Formel]{Formel} $\varphi$ gilt: $T \hyperref[Beweisbar]{\vdash} \varphi$ oder $T \hyperref[Beweisbar]{\vdash} \neg \varphi$
	
	\subsection{Charakterisierungslemma}
	T \hyperref[ALKonsistent]{konsistent} $\Leftrightarrow$ Es gibt kein $\varphi$ mit $T \hyperref[Beweisbar]{\vdash} \varphi$ und $T \hyperref[Beweisbar]{\vdash} \neg \varphi$
	
	\subsection{Endlichkeitslemma für Konsistenz}
	Eine Formelmenge T ist genau dann \hyperref[ALKonsistent]{konsistent}, wenn jede endliche Teilmenge $T_0$ von T \hyperref[ALKonsistent]{konsistent} ist
	
	\subsection{Lemma über Beweisbarkeit und Konsistenz}
	$T \hyperref[Beweisbar]{\vdash} \varphi \Leftrightarrow T \cup \{\neg \varphi\}$
	
	\subsection{*Konsistenzlemma}
	Jede \hyperref[Erfullbar]{erfüllbare} Formelmenge T ist \hyperref[ALKonsistent]{konsistent}
	
	\subsection{*Erfüllbarkeitslemma}
	Jede \hyperref[ALKonsistent]{konsistente} Formelmenge T ist \hyperref[Erfullbar]{erfüllbare}
	
	\subsection{Adäquatheitssatz für den Folgerungsbegriff}
	Für jede Formelmenge T und jede \hyperref[Formel]{Formel} $\varphi$ gilt: $T \hyperref[Erfullbar]{\vDash} \varphi \Leftrightarrow T \hyperref[Beweisbar]{\vdash} \varphi$
	
	\subsection{Adäquatheitssatz für den Erfüllbarkeitsbegriff}
	Für jede Formelmenge T gilt: T \hyperref[Erfullbar]{erfüllbar} $\Leftrightarrow$ T \hyperref[ALKonsistent]{konsistent} 
	
	\subsection{*Kompaktheitssatz/Endlichkeitssatz für den Folgerungsbegriff}
	Eine \hyperref[Formel]{Formel} $\varphi$ folgt genau dann aus einer Formelmenge T, wenn es eine endliche Teilmenge $T_0$ von T gibt, aus der $\varphi$ folgt: $T \hyperref[Erfullbar]{\vDash} \varphi \Leftrightarrow$ Es gibt $T_0 \subseteq T$ endlich: $T_0 \hyperref[Erfullbar]{\vDash} \varphi$
	
	\subsection{*Kompaktheitssatz/Endlichkeitssatz für den Erfüllbarkeitsbegriff}
	erfb[T] $\Leftrightarrow$ Für alle $T_0 \subseteq T$ endlich: erfb[$T_0$]
	
	\newpage
	\section{Prädikatenlogik}
	\subsection{*Mathematische Struktur}
	\label{Struktur}
	Struktur $\mathcal{A}$ ist ein 4-Tupel $\mathcal{A} = (A; (R_i^{\mathcal{A}} | i \in I); (f_j^{\mathcal{A}} | j \in J); (c_k^{\mathcal{A}} | k \in K))$ \newline 
	Mit I, J, K beliebige Mengen für die gelten:
	\begin{itemize}
		\item A nichtleer : Universum/Träger/Individuenbereich der Struktur $\mathcal{A}$
		\item $\forall i \in I: R_i^{\mathcal{A}}$ $n_i$-stellige Relation auf A: Grundrelationen auf $\mathcal{A}$
		\item $\forall j \in J: f_j^{\mathcal{A}}$ $m_j$-stellige Funktion auf A; $f_j^{\mathcal{A}}: A^{m_j} \rightarrow A$: Grundfunktionen 
		\item $\forall k \in K: c_k^{\mathcal{A}}$ Element von A, welche die Konstanten von A bilden
	\end{itemize}

	\subsection{*Signatur}
	\label{Signatur}
	\hyperref[Struktur]{Struktur} $\mathcal{A} = (A; (R_i^{\mathcal{A}} | i \in I); (f_j^{\mathcal{A}} | j \in J); (c_k^{\mathcal{A}} | k \in K))$ ist vom Typ / besitzt die Signatur: $\sigma (\mathcal{A}) = ((n_i | i \in I); (m_j | j \in J); K)$ \newline
	falls $R_i^{\mathcal{A}}$ $n_i$-stellig, $f_j^{\mathcal{A}}$ $m_j$-stellig\newline 
	Sofern die \hyperref[Struktur]{Struktur} keine Relation/Funktion hat, kennzeichnet man das in der Signatur mit einem -
	
	\subsection{*Sprache}
	\label{PLSprache}
	Eine Sprache besteht aus Zeichen die man verwenden kann/muss wie: 1)
	Aussagenvariablen 2) Junktoren (wie $\neg, \lor$) 3) Existenzquantor 4) Gleichzeichen 5) Komma und Klammersymbole.
	
	\subsection{*Terme: Syntax}
	Sei $\mathcal{L} = \mathcal{L}(\sigma)$ mit $\sigma = ((n_i | i \in I); (m_j | j \in J); K)$. Menge der ($\mathcal{L}$-)Therme ist Induktiv definiert durch:
	\begin{description}
		\item[(T1)] Jede Variable $v_n$, jede Konstante $c_k$ ist ein Term
		\item[(T2)] $t_1, ..., t_m \Rightarrow f_j(t_1, ..., t_m)$ ist ein Term
	\end{description}
	
	\subsection{*Terme: Semantik}
	Für konstanten $\mathcal{L}$-Term t ist $t^{\mathcal{A}} \in A$ durch Ind(t) definiert:
	\begin{enumerate}
		\item $(c_k)^{\mathcal{A}} := c_k^{\mathcal{A}}$
		\item $(f_j(t_1, ..., t_{m_j}))^{\mathcal{A}} := f_j(t_1, ..., t_{m_j})^{\mathcal{A}}$
	\end{enumerate}

	\subsection{(Variablen-)Belegung in $\mathcal{A}$}
	Sei $V = \{x_1, ..., x_n\}$ Menge von Variablen und $\mathcal{A}$ \hyperref[Struktur]{$\mathcal{L}$-Struktur}. \newline
	Eine (Variablen-)\hyperref[Belegung]{Belegung} B von V in $\mathcal{A}$ ist eine Abbildung $B: V \rightarrow \mathcal{A}$ 
	
	\subsection{Wert von t}
	Sei $t \equiv t(\overrightarrow{x}) \equiv t(x_1, ..., x_n) \mathcal{L}$-Term, in dem höchstens Variablen $x_1, ..., x_n$ vorkommen, $B: \{x_1, ..., x_n\} \rightarrow \mathcal{A}$ \hyperref[Belegung]{Belegung} der Variablen in \hyperref[Struktur]{$\mathcal{L}$-Struktur $\mathcal{A}$}, Wert $t_B^{\mathcal{A}} \in A$ von t in $\mathcal{A}$ bezüglich \hyperref[Belegung]{Belegung} B ist durch Ind(t) definiert: 
	\begin{enumerate}
		\item $(x_i)_B^{\mathcal{A}} := B(x_i), (c_k)_B^{\mathcal{A}} := {c_k}^{\mathcal{A}}$
		\item $(f_j(t_1, ..., t_{m_j}))_B^{\mathcal{A}} := f_j^{\mathcal{A}}((t_1)_B^{\mathcal{A}}, ..., (t_{m_j})_B^{\mathcal{A}})$
	\end{enumerate}

	\subsection{*Koinzidenzlemma}
	\hyperref[Struktur]{$\mathcal{A} \mathcal{L}$-Struktur}, t $\mathcal{L}$-Term, $V = \{x_1, ..., x_n\}, V' = \{x_1', ..., x_n'\}$ Variablenmenge mit $V(t)\subseteq V, V'$ und B, B' \hyperref[Belegung]{Belegungen} in $\mathcal{A}$ sodass $B \upharpoonright V(t) = B' \upharpoonright V(t) \Rightarrow t_B^{\mathcal{A}} = t_{B'}^{\mathcal{A}}$
	
	\subsection{*Formeln und Sätze: Syntax}
	Die Menge der ($\mathcal{L}$-)\hyperref[Formel]{Formeln} ist definiert durch:
	
	\begin{tabular}{l l}
		(F1) (Gleichheitsformel) & \begin{tabular}{l l}
			a) & $t_1, t_2$ Terme $\Rightarrow t_1 = t_2$ ist eine \hyperref[Formel]{Formel} \\
			b) & $t_1, ..., t_{n_i}$ Terme $\Rightarrow R_i(t_1, ..., t_{n_i})$ ist eine \hyperref[Formel]{Formel}
		\end{tabular} \\
		(F2) (Negationsformel) & $\varphi$ \hyperref[Formel]{Formel} $\Rightarrow \neg \varphi$ \hyperref[Formel]{Formel} \\
		(F3) (Disjunktionen) & $\varphi_1, \varphi_2$ \hyperref[Formel]{Formeln} $\Rightarrow (\varphi_1 \lor \varphi_2)$ \hyperref[Formel]{Formel} \\
		(F4) (Existenzformel) & $\varphi$ \hyperref[Formel]{Formel}, x Variable $\Rightarrow \exists x \varphi$ \hyperref[Formel]{Formel}
	\end{tabular}

	\subsection{($\mathcal{L}$-)Satz}
	\label{LSatz}
	Kommt in ($\mathcal{L}$-)\hyperref[Formel]{Formel} keine Variable frei vor ($FV(\varphi) = \emptyset$), dann ist $\varphi$ ein ($\mathcal{L}$-)Satz
	
	\subsection{*Formeln und Sätze: Semantik}
	\hyperref[Struktur]{$\mathcal{A} \mathcal{L}$-Struktur}, $\varphi \equiv \varphi(x_1, ..., x_n) \mathcal{L}$-\hyperref[Formel]{Formel} mit $FV(\varphi) \subseteq \{x_1, ..., x_n\}$, B \hyperref[Belegung]{Belegung} von $\{x_1, ..., x_n\}$ \newline
	$\Rightarrow$ Wahrheitswert $W_B^{\mathcal{A}}(\varphi) \in \{0, 1\}$ von $\varphi$ in $\mathcal{A}$ bezüglich B durch $Ind(\varphi)$ definiert:
	\begin{enumerate}
		\item $W_B^{\mathcal{A}}(t_1 = t_2) = 1$ gdw $(t_1)_B^{\mathcal{A}} = (t_2)_B^{\mathcal{A}}$
		\item $W_B^{\mathcal{A}}(R_i(t_1, ..., t_{n_i})) = 1$ gdw $((t_1)_B^{\mathcal{A}}, ..., (t_{n_i})_B^{\mathcal{A}}) \in R_i^{\mathcal{A}}$
		\item $W_B^{\mathcal{A}}(\neg \psi) = 1 \Leftrightarrow W_B^{\mathcal{A}}(\psi) = 0$
		\item $W_B^{\mathcal{A}}(\varphi_1 \lor \varphi_2) = 1$ gdw $W_B^{\mathcal{A}}(\varphi_1) = 1$ oder $W_B^{\mathcal{A}}(\varphi_2) = 1$
		\item $W_B^{\mathcal{A}}(\exists y \psi) = 1$ gdw $\exists B'$ von $\{x_1, ..., x_n, y\}$, die mit B auf $\{x_1, ..., x_n\} \backslash \{y\}$ übereinstimmt und $W_B^{\mathcal{A}}(\psi) = 1$
	\end{enumerate}

	\subsection{*Koinzidenzlemma (für Formeln)}
	\hyperref[Struktur]{$\mathcal{A}$ eine $\mathcal{L}$-Struktur}, $\varphi \mathcal{L}$-\hyperref[Formel]{Formel}, $V = \{x_1, ..., x_m\}, V' = \{x_1', ..., x_m'\}$ Variablenmengen mit $FV(\varphi) \subseteq V, V'$ und B, B' \hyperref[Belegung]{Belegungen} in $\mathcal{A}$ s.d. $B \upharpoonright FV(\varphi) = B' \upharpoonleft FV(\varphi) \Rightarrow W_B^{\mathcal{A}}(\varphi) = W_{B'}^{\mathcal{A}}(\varphi)$ 
	
	\subsection{*Satz ist in Struktur wahr}
	\hyperref[LSatz]{$\mathcal{L}$-Satz} $\sigma$ ist in \hyperref[Struktur]{$\mathcal{L}$-Struktur $\mathcal{A}$} wahr, wenn $W_B^{\mathcal{A}}(\sigma) = 1$ für die leere Variablenmenge gilt.
	
	Man sagt für $A \hyperref[Erfullbar]{\vDash} \sigma$, A ist Modell von sigma.
	
	\subsection{*Formel ist in Struktur wahr}
	$\mathcal{L}$-\hyperref[Formel]{Formel} $\varphi$ ist in \hyperref[Struktur]{$\mathcal{L}$-Struktur $\mathcal{A}$} wahr, wenn $W_B^{\mathcal{A}}(\varphi) = 1$ für alle \hyperref[Belegung]{Variablenbelegungen} B von $FV(\varphi)$ gilt
	
	Man sagt für $A \hyperref[Erfullbar]{\vDash} \varphi$, A ist Modell von phi.
	
	\subsection{*Allabschluss mit Formeln}
	Der Allabschluss $\forall \varphi$ einer \hyperref[Formel]{Formel} $\varphi$, mit freien Variablen $x_1, ..., x_n$ ist ein Satz $\forall \varphi := \forall x_1, ..., \forall x_n \varphi$, wobei Variablen $x_1, ..., x_n$ geordnet bezüglich Aufzählung
	
	\subsection{*Relation mit Formeln}
	$\varphi \equiv \varphi(x_1, ..., x_n) \mathcal{L}$-\hyperref[Formel]{Formel} mit $FV(\varphi) \subseteq \{x_1, ..., x_n\}$. Die von $\varphi$ auf \hyperref[Struktur]{$\mathcal{L}$-Struktur $\mathcal{A}$} definierte n-stellige Relation $R_{\varphi}^{\mathcal{A}}$ ist bestimmt durch: $(a_1, ..., a_n) \in R_{\varphi}^{\mathcal{A}} \Leftrightarrow \mathcal{A} \hyperref[Erfullbar]{\vDash} \varphi[a_1, ..., a_n]$
	
	\subsection{*Zentrale semantische Konzepte: Allgemeingültigkeit}
	$\mathcal{L}$-\hyperref[Formel]{Formel} $\varphi$ ist (logisch) wahr oder allgemeingültig, wenn alle \hyperref[Struktur]{$\mathcal{L}$-Strukturen} Modell von $\varphi$ sind: Für alle \hyperref[Struktur]{$\mathcal{L}$-Strukturen $\mathcal{A}$}
	
	\subsection{*Zentrale semantische Konzepte: Erfüllbarkeit}
	\begin{tasks}
		\task ($\mathcal{L}$-)\hyperref[Formel]{Formel} $\varphi$ ist \hyperref[Erfullbar]{erfüllbar}, wenn $\varphi$ ein Modell besitzt: Es gibt eine \hyperref[Struktur]{$\mathcal{L}$-Struktur $\mathcal{A}$} mit $\mathcal{A} \hyperref[Erfullbar]{\vDash} \varphi$. Andernfalls ist $\varphi$ unerfüllbar.
		\task Menge $\Phi$ von $\mathcal{L}$-\hyperref[Formel]{Formeln} ist \hyperref[Erfullbar]{erfüllbar}, wenn es eine \hyperref[Struktur]{$\mathcal{L}$-Struktur $\mathcal{A}$} gibt, die Modell aller \hyperref[Formel]{Formeln} in $\Phi$ ist
	\end{tasks}

	\subsection{*Zentrale semantische Konzepte: Folgerungsbegriff}
	$\mathcal{L}$-\hyperref[Formel]{Formel} $\varphi$ folgt aus $\mathcal{L}$-\hyperref[Formel]{Formel} $\psi (\psi \hyperref[Erfullbar]{\vDash} \varphi)$, wenn jedes Modell von $\psi$ auch Modell von $\varphi$ ist.
	
	Für alle \hyperref[Struktur]{$\mathcal{L}$-Strukturen $\mathcal{A}$}: $\mathcal{A} \psi \Rightarrow \mathcal{A} \hyperref[Erfullbar]{\vDash} \varphi$
	
	$\varphi$ und $\psi$ sind äquivalent ($\varphi$ äq $\psi$), wenn $\varphi$ und $\psi$ die selben Modelle haben
	
	\subsection{Formel folgt aus Menge}
	$\mathcal{L}$-\hyperref[Formel]{Formel} $\varphi$ folgt aus Menge $\Phi$ von ($\mathcal{L}$-)\hyperref[Formel]{Formeln} ($\Phi \hyperref[Erfullbar]{\vDash} \varphi$), wenn jedes Modell von $\Phi$ auch Modell von $\varphi$ ist:
	
	Für alle \hyperref[Struktur]{$\mathcal{L}$-Strukturen $\mathcal{A}$}: $\mathcal{A} \hyperref[Erfullbar]{\vDash} \Phi \Rightarrow \mathcal{A} \hyperref[Erfullbar]{\vDash} \varphi$ 
	
	\subsection{Monotonie des Folgerungsbegriffs}
	$\Phi \subseteq \Psi$ und $\Phi \hyperref[Erfullbar]{\vDash} \varphi \Rightarrow \Psi \hyperref[Erfullbar]{\vDash} \varphi$
	
	\subsection{Verträglichkeit von $\vDash$ und $\rightarrow$}
	\begin{tabular}{l l}
		$\varphi_1, ..., \varphi_n \hyperref[Erfullbar]{\vDash} \sigma$ & $\Leftrightarrow \varphi_1 \land ... \land \varphi_n \hyperref[Erfullbar]{\vDash} \sigma$ \\
		& $\Leftrightarrow \hyperref[Erfullbar]{\vDash} (\varphi_1 \land ... \land \varphi_n) \rightarrow \sigma$
	\end{tabular}

	\subsection{Erfüllbare Formelmenge}
	$\mathcal{L}$-Formelmenge $\Phi$ ist \hyperref[Erfullbar]{erfüllbar} $\Leftrightarrow$ Es gibt keinen \hyperref[LSatz]{$\mathcal{L}$-Satz} $\sigma$ mit: $\Phi \hyperref[Erfullbar]{\vDash} \sigma$ und $\Phi \hyperref[Erfullbar]{\vDash} \neg \sigma$
	
	\subsection{Zusammenhang zwischen Folgerungs- und Erfüllbarkeitsbegriff}
	Für jede $\mathcal{L}$-Formelmenge $\Phi$ und jeden \hyperref[LSatz]{$\mathcal{L}$-Satz} $\sigma$ gilt: $\Phi \hyperref[Erfullbar]{\vDash} \sigma \Leftrightarrow \Phi \cup \{\neg \sigma\}$ unerfüllbar
	
	\subsection{*Tautologie}
	\hyperref[Formel]{Formel} $\varphi$ ist eine Tautologie (aussagenlogisch gültig, $\hyperref[Erfullbar]{\vDash_{AL}} \varphi$), falls $B(\varphi) = 1$ für alle al. \hyperref[Belegung]{Belegungen} B
	
	Jede Tautologie ist allgemeingültig: $\hyperref[Erfullbar]{\vDash_{AL}} \varphi \rightarrow \hyperref[Erfullbar]{\vDash} \varphi$
	
	\subsection{Aussagenlogische Folgerung}
	Formen $\varphi$ ist aussagenlogische Folgerung aus \hyperref[Formel]{Formeln} \newline 
	$\varphi_1, ..., \varphi_n (\varphi_1, ..., \varphi_n \hyperref[Erfullbar]{\vDash_{AL}} \varphi)$, falls für alle al. \hyperref[Belegung]{Belegungen} B gilt: \newline
	$B(\varphi_1) = ... = B(\varphi_n) = 1 \Rightarrow B(\varphi) = 1$
	
	\subsection{Lemma über al Folgerungen}
	$\varphi_1, ..., \varphi_n \hyperref[Erfullbar]{\vDash_{AL}} \varphi \Rightarrow \varphi_1, ..., \varphi_n \hyperref[Erfullbar]{\vDash} \varphi$
	
	\subsection{Substitution}
	\label{Substituierbar}
	$\varphi[t/x]$: alle freien vorkommen von x durch Term t ersetzen
	
	\subsection{!Substituirbarkeitsbedingung}
	Term t heißt in \hyperref[Formel]{Formel} $\varphi$ für Variable x \hyperref[Substituierbar]{substituierbar}, wenn keine in t vorkommende Variable $y \ne x$ in $\varphi$ gebunden vorkommt
	
	\subsection{Substitutionslemma}
	Sei Term t für Variable x in \hyperref[Formel]{Formel} $\varphi$ \hyperref[Substituierbar]{substituierbar} 
	
	$\Rightarrow \varphi[t/x] \rightarrow \exists x \varphi$ allgemeingültig
	
	\section{Ein Adäquater Kalkül der Prädikatenlogik}
	
	\subsection{Axiome}
	Substitutionsaxiome:
	
	(S1) $\varphi[t/x] \rightarrow \exists x \varphi$
	\newline	
	Gleichheitsaxiome:
	
	\begin{description}
		\item[(G1)] x = x
		\item[(G2)] $x_1 = y_1 \land ... \land x_{m_j} = y_{m_j} \rightarrow f(x_1, ..., x_{m_j}) = f(y_1, ..., y_{m_j})$
		\item[(G3)] $x_1 = y_1 \land ... \land x_{n_i} = y_{n_i} \rightarrow R(x_1, ..., x_{n_i}) = R(y_1, ..., y_{n_i})$
		\item[(G4)] $x_1 = y_1 \land x_2 = y_2 \land x_1 = x_2 \rightarrow y_1 = y_2$
	\end{description} 
	$\exists$-Einführungsregeln:
	
	($\exists$1) $\frac{\varphi \rightarrow \psi}{\exists x \varphi \rightarrow \psi}$ 
	
	Falls x in $\psi$ nicht frei vorkommt. 
	
	\subsection{Tautologiesatz}
	\begin{tasks}[counter-format=(tsk[r]), label-width=4ex]
		\task $\vDash_{AL} \varphi \Rightarrow \hyperref[Beweisbar]{\vdash} \varphi$ 
		\task $\varphi_1, ..., \varphi_n \vDash_{AL} \varphi \Rightarrow \varphi_1, ..., \varphi_n \hyperref[Beweisbar]{\vdash} \varphi$
	\end{tasks}

	\subsection{Aussagenlogische Schlüsse}
	(AL) $\psi_1, ..., \psi_n \vDash_{AL} \varphi$ ($n \ne 0$) $\Rightarrow \frac{\psi_1, ..., \psi_n}{\varphi}$ in $S_{PL}$ für n = 0 ist (AL) das Axiomenschema (Hierdurch bestimmt man die Regel $\varphi$)\newline
	(AL) $\varphi$ (falls $\vDash_{AL} \varphi$) (Hierdurch bestimmt man das Axiom $\varphi$)
	
	\subsection{Generalisierung und Distribution}
	\begin{description}
		\item[($\forall$1)] $\frac{\varphi \rightarrow \psi}{\varphi \rightarrow \forall x \psi}$
		\item[($\forall$2)] $\frac{\varphi}{\forall x \varphi}$
		\item[($D_{\exists}$)] $\frac{\varphi \rightarrow \psi}{\exists x \varphi \rightarrow \exists x \psi}$
		\item[($D_{\forall}$)] $\frac{\varphi \rightarrow \psi}{\forall x \varphi \rightarrow \forall x \psi}$
	\end{description}

	\subsection{Ersetzung und Substitution I}
	\begin{description}
		\item[(E)] $\frac{\psi_1 \leftrightarrow \psi_1', ..., \psi_n \leftrightarrow \psi_n'}{\varphi \leftrightarrow \varphi'}$ falls $\varphi'$ aus $\varphi$ durch Ersetzen einzelner Vorkommen der Teilformeln $\psi_i$ durch $\psi_i'$ entsteht (wobei die ersetzten Teilformeln nicht ineinander liegen)
		\item[(S2)] $\frac{\varphi}{\varphi[t_1/x_1, ..., t_n/x_n]}$ falls t für $x_i$ Substituirbar in p(S)
	\end{description}

	\subsection{Umbenennung Gebundener Variablen}
	\begin{description}
		\item[(U)] $\varphi \leftrightarrow \varphi^*$ \newline
		Falls $\varphi^*$ aus $\varphi$ durch Umbenennung gebundener Variablen entsteht. Hierbei darf bei Ersetzung einer Teilformel $\exists x \psi$ durch $\exists y \psi[y/x]$ die Variable y nicht in $\psi$ vorkommen.
	\end{description}

	\subsection{Substitution II}
	\begin{description}
		\item[($S2_{\exists}$)] $\varphi[t_1/x_1, ..., t_n/x_n] \rightarrow \exists x_1 ... \exists x_n \varphi$ (Falls SB erfüllt)
		\item[($S2_{\forall}$)] $\forall x_1 ... \forall x_n \varphi \rightarrow \varphi[t_1/x_1, ..., t_n/x_n]$ (Falls SB erfüllt)
		\item[($\forall 3_1$)] $\frac{\varphi}{\forall \varphi}$
		\item[($\forall 3_2$)] $\frac{\forall \varphi}{\varphi}$
	\end{description}

	\subsection{Gleichheit}
	\begin{description}
		\item[(G5)] s = t $\rightarrow$ t = s
		\item[(G6)] $t_1 = t_1' \land ... \land t_n = t_n' \rightarrow s = s'$ \newline
		falls s' aus s durch Ersetzen einiger (oder auch aller) Vorkommenden der Terme $t_i$ durch entsprechende Terme $t_i'$ entsteht
		\item[(G7)] $t_1 = t_1' \land ... \land t_n = t_n' \rightarrow (\varphi[t_1/x_1, ..., t_n/x_n] \leftrightarrow \varphi[t_1'/x_1, ..., t_n'/x_n])$ \newline 
		falls die Terme $t_i$ und $t_i'$ für $x_i$ in $\varphi$ Substituirbar sind
	\end{description}

	\subsection{*Deduktionstheorem}
	Sei $\Phi$ Menge von Formeln, $\psi$ Formel $\sigma$ Satz. Dann gilt: 
	
	$\Phi \hyperref[Beweisbar]{\vdash} \sigma \rightarrow \psi \Leftrightarrow \Phi \cup \{\sigma\} \hyperref[Beweisbar]{\vdash} \psi$
	
	\subsection{Korollar zum Deduktionstheorem}
	Sei $\Phi$ Menge von Formeln, $\psi$ Formel, $\sigma_1, ..., \sigma_n$ Sätze. Dann gilt:
	
	$\Phi \hyperref[Beweisbar]{\vdash} \sigma_1 \land ... \land \sigma_n \rightarrow \psi \Leftrightarrow \Phi \cup \{\sigma_1, ..., \sigma_n\} \hyperref[Beweisbar]{\vdash} \psi$
	
	\subsection{Theorie}
	\label{Theorie}
	Eine ($\mathcal{L}$-)Theorie T ist ein Paar T = ($\mathcal{L}, \Sigma$), wobei:
	\begin{itemize}
		\item $\mathcal{L}$ eine \hyperref[PLSprache]{Sprache} der Prädikatenlogik: Sprache der Theorie T
		\item $\Sigma$ eine Menge von $\mathcal{L}$-Sätzen ist: Axiom von T
	\end{itemize}
	T endlich $\Leftrightarrow \Sigma$ endlich
	
	\subsection{Modellklasse}
	Die Modellklasse Mod(T) einer \hyperref[Theorie]{$\mathcal{L}$-Theorie} $T = (\mathcal{L}, \Sigma)$ ist die Menge aller \hyperref[Struktur]{$\mathcal{L}$-Strukturen}, die Modell der Axiomenmenge $\Sigma$ von T sind: $Mod(T) = \{\mathcal{A} : \mathcal{A} \text{ ist eine } \hyperref[Struktur]{\mathcal{L} \text{-Struktur und } \mathcal{A}} \hyperref[Erfullbar]{\vDash} \Sigma\}$
	\begin{itemize}
		\item $\mathcal{A}$ Modell von T: $\mathcal{A} \hyperref[Erfullbar]{\vDash} T$
		\item T erfüllbar $\Leftrightarrow \Sigma$ erfüllbar, also $Mod(T \ne \emptyset)$
	\end{itemize}
	
	\subsection{Syntaktischer Deduktiver Abschluss}
	Der (syntaktische) deduktive Abschluss von $T = (\mathcal{L}, \Sigma)$ ist: \newline 
	$C_{\vdash}(T) = \{\sigma : \sigma \text{ist ein } \hyperref[LSatz]{\mathcal{L} \text{-Satz }} \text{ und } T \hyperref[Beweisbar]{\vdash} \sigma\}$
	
	Semantischer Abschluss: \newline
	$C_{\vDash}(T) = \{\sigma : \sigma \text{ist ein } \hyperref[LSatz]{\mathcal{L} \text{-Satz}} \text{ und } T \hyperref[Erfullbar]{\vDash} \sigma\}$
	
	\subsection{Äquivalente L-Theorien}
	Zwei \hyperref[Theorie]{$\mathcal{L}$-Theorien $T = (\mathcal{L}, \Sigma), T' = (\mathcal{L}, \Sigma')$} sind äquivalent $\Leftrightarrow C_{\vdash}(T) = C_{\vdash}(T')$
	
	\subsection{Erweiterung und Konservative Erweiterung}
	\label{Erweiterung}
	\hyperref[Theorie]{$\mathcal{L}$'-Theorien $T' = (\mathcal{L}', \Sigma')$} ist eine Erweiterung der \hyperref[Theorie]{$\mathcal{L}$-Theorie $T = (\mathcal{L}, \Sigma) (T \subseteq T')$} falls:
	\begin{tasks}[counter-format=(tsk[r]), label-width=4ex]
		\task $\mathcal{L}$' Erweiterung von $\mathcal{L}$
		\task für jede $\mathcal{L}$-Formel $\varphi$ gilt: $T \hyperref[Beweisbar]{\vdash} \varphi \Rightarrow T' \hyperref[Beweisbar]{\vdash} \varphi$ \newline
		Gilt außerdem für jede $\mathcal{L}$-Formel $\varphi$: $T \hyperref[Beweisbar]{\vdash} \varphi \Leftrightarrow T' \hyperref[Beweisbar]{\vdash} \varphi$ so heißt T' eine konservative Erweiterung
	\end{tasks}

	\subsection{Sprachliche Erweiterung}
	\label{SprachlicheErweiterung}
	\hyperref[Theorie]{Theorie $T' = (\mathcal{L}', \Sigma')$} heißt sprachliche Erweiterung der \hyperref[Theorie]{Theorie $T = (\mathcal{L}, \Sigma)$}, wenn $\mathcal{L} \subseteq \mathcal{L}'$ und $\Sigma = \Sigma'$
	
	Außerdem gilt dann noch, dass T' eine \hyperref[Erweiterung]{konservative Erweiterung} von T ist, da: Für alle $\mathcal{L}$-Formeln $\varphi$ gilt: $T' \hyperref[Beweisbar]{\vdash} \varphi \Rightarrow T \hyperref[Beweisbar]{\vdash} \varphi$
	
	\subsection{Zusätzliches zu Theorien}
	\begin{enumerate}
		\item Sei $T = (\mathcal{L}, \Sigma)$ eine \hyperref[Theorie]{$\mathcal{L}$-Theorie}, $\mathcal{L}$' eine \hyperref[Erweiterung]{Erweiterung} von $\mathcal{L}$ um die Konstante c, $T' = (\mathcal{L}', \Sigma)$ rein \hyperref[SprachlicheErweiterung]{sprachliche Erweiterung} von T auf $\mathcal{L}$'. Dann gilt für jede $\mathcal{L}$-Formel $\varphi$: $T' \hyperref[Beweisbar]{\vdash} \varphi[c/x] \Leftrightarrow T \hyperref[Beweisbar]{\vdash} \forall x \varphi (\Leftrightarrow T' \hyperref[Beweisbar]{\vdash} \forall x \varphi)$
		\item Sei $\Sigma$ Menge von $\mathcal{L}_0$-Sätzen, $\varphi \mathcal{L}_0$-Formel, c konstante von $\mathcal{L}_0$, die weder in $\varphi$ noch in $\Sigma$ vorkommt. Dann gilt: $\Sigma \hyperref[Beweisbar]{\vdash} \forall x \varphi$
	\end{enumerate}
	
	\subsection{Konsistent/Widerspruchsfrei}
	\label{PLKonsistent}
	Eine \hyperref[Theorie]{Theorie $T = (\mathcal{L}, \Sigma)$} ist \hyperref[PLKonsistent]{konsistent} oder widerspruchsfrei, falls es einen
	\hyperref[LSatz]{$\mathcal{L}$-Satz} $\sigma$ gibt mit $T \nvdash \sigma$
	Man nennt eine \hyperref[Theorie]{Theorie} \hyperref[PLKonsistent]{konsistent}, wenn man mindestens einen Satz
	nicht beweisen kann - man kann nicht alles beweisen, sonst würde etwas 
	vorkommen wie $T \hyperref[Beweisbar]{\vdash} \sigma \land T \hyperref[Beweisbar]{\vdash} \neg \sigma$.
	
	\subsection{Charakterisierungslemma für Konsistenz (LCK)}
	\hyperref[Theorie]{Theorie $T = (\mathcal{L}, \Sigma)$} ist \hyperref[PLKonsistent]{konsistent} $\Leftrightarrow \nexists$ \hyperref[LSatz]{$\mathcal{L}$-Satz} $\sigma : T \hyperref[Beweisbar]{\vdash} \sigma$ und $T \hyperref[Beweisbar]{\vdash} \neg \sigma$
	
	\subsection{Lemma über Zusammenhang zwischen Beweisbarkeit und Konsistenz (LBK)}
	\begin{tasks}[counter-format=(tsk[r]), label-width=4ex]
		\task $T \hyperref[Beweisbar]{\vdash} \varphi \Leftrightarrow T \cup \{\neg \forall \varphi\}$ \hyperref[PLKonsistent]{inkonsistent}
		\task $T \nvdash \varphi \Leftrightarrow T \cup \{\neg \forall \varphi\}$ \hyperref[PLKonsistent]{konsistent}
	\end{tasks}
	für Satz $\sigma$ ist Allabschluss $\forall \sigma = \sigma \Rightarrow$ für Sätze gilt: $T \hyperref[Beweisbar]{\vdash} \sigma \Leftrightarrow T \cup \{\neg \sigma\}$ \hyperref[PLKonsistent]{inkonsistent}
	
	\subsection{*Erfüllbarkeitslemma (Modellexistenzsatz, EL)}
	Jede \hyperref[PLKonsistent]{konsistente} \hyperref[Theorie]{Theorie T} ist erfüllbar (d.h. besitzt ein Modell)
	
	\subsection{*Vollständigkeitssatz (VS)}
	\label{TheorieVollstandig}
	Eine \hyperref[Theorie]{Theorie} ist Vollständig wenn gilt: $T \hyperref[Erfullbar]{\vDash} \sigma \Rightarrow T \hyperref[Beweisbar]{\vdash} \sigma$
	
	\subsection{Relation \textasciitilde}
	Relation \textasciitilde $\subseteq K_{\mathcal{L}} * K_{\mathcal{L}}$ ist gegeben durch: $ t \sim t' \Leftrightarrow T \hyperref[Beweisbar]{\vdash} t = t'$ \newline
	$K_{\mathcal{L}}$ = Menge aller konstanten $\mathcal{L}$-Terme 
	
	\subsection{*!Termstruktur (Termmodell)}
	Die Termstruktur (Termmodell) $\mathcal{A}_T = (A_T; (R_i^{\mathcal{A}_T}: i \in I); (f_j^{\mathcal{A}_T}: j \in J); (c_k^{\mathcal{A}_T}: k \in K))$ von T ist gegeben durch: 
	\begin{itemize}
		\item $A_T := K_T = \{\overline{t}: t \in K_{\mathcal{L}}\}$ wobei $\overline{t} = \{t' \in K_{\mathcal{L}}: t' \sim t\}$
		\item $(\overline{t_1}, ..., \overline{t_{n_i}}) \in R_i^{\mathcal{A}_T} \Leftrightarrow T \hyperref[Beweisbar]{\vdash} R_i(t_1, ..., t_{n_i})$
		\item $f_j^{\mathcal{A}_T}(\overline{t_1}, ..., \overline{t_{n_i}}) := \overline{f_j^{\mathcal{A}_T}(t_1, ..., t_{n_i})}$
		\item $c_k^{\mathcal{A}_T} := \overline{c_k}$
	\end{itemize}

	\subsection{Wohldefinierte L-Struktur}
	Falls Sprache $\mathcal{L}$ zumindest eine Konstante besitzt (d.h. $K \ne \emptyset$), so ist \hyperref[Struktur]{$\mathcal{A}_T$ eine wohldefinierte $\mathcal{L}$-Struktur} und es gilt: $\forall t \in K_{\mathcal{L}}: t^{\mathcal{A}_T} = \overline{t}$
	
	\subsection{Lemma über Termmodelle}
	Sei  $K \ne \emptyset$. Dann gilt für atomare $\mathcal{L}$-Sätze $\sigma$: $\mathcal{A}_T \hyperref[Erfullbar]{\vDash} \sigma \Leftrightarrow T \hyperref[Beweisbar]{\vdash} \sigma$
	
	\subsection{Syntaktisch Vollständige Theorie}
	\hyperref[Theorie]{$\mathcal{L}$-Theorie $T = (\mathcal{L}, \Sigma)$} ist (syntaktisch) vollständig, falls für jeden \hyperref[LSatz]{$\mathcal{L}$-Satz} $\sigma$ gilt: $T \hyperref[Beweisbar]{\vdash} \sigma$ oder $T \hyperref[Beweisbar]{\vdash} \neg \sigma$
	\begin{itemize}
		\item \hyperref[Theorie]{Theorie T} ist \hyperref[PLKonsistent]{konsistent} und vollständig, dann gilt für jeden \hyperref[LSatz]{$\mathcal{L}$-Satz} entweder $T \hyperref[Beweisbar]{\vdash} \sigma$ oder $T \hyperref[Beweisbar]{\vdash} \neg \sigma$ 
		\item \hyperref[PLKonsistent]{konsistente} vollständige \hyperref[Theorie]{Theorie} heißt maximal vollständig 
	\end{itemize}

	\subsection{*Henkin-Theorie}
	\label{Henkin}
	\hyperref[Theorie]{$\mathcal{L}$-Theorie $T = (\mathcal{L}, \Sigma)$} ist eine Henkin-Theorie, falls es für jeden $\mathcal{L}$-Existenzsatz $\exists x \varphi$ einen konstanten $\mathcal{L}$-Term t gibt mit: $T \hyperref[Beweisbar]{\vdash} \exists x \varphi \rightarrow \varphi[t/x]$
	
	\subsection{Satz über Termmodelle}
	Sei $T = (\mathcal{L}, \Sigma)$ \hyperref[PLKonsistent]{konsistent}, vollständig und eine \hyperref[Henkin]{Henkin-Theorie}. Dann gilt für alle $\mathcal{L}$-Sätze $\sigma$: $\mathcal{A}_T \hyperref[Erfullbar]{\vDash} \sigma \Leftrightarrow T \hyperref[Beweisbar]{\vdash} \sigma$ Insbesondere $\mathcal{A}_T$ ist Modell von T
	
	\subsection{Satz von Lindenbaum (PL)}
	Sei $T = (\mathcal{L}, \Sigma)$ eine \hyperref[PLKonsistent]{konsistente} \hyperref[Theorie]{Theorie}, wobei $\mathcal{L}$ abzählbar ist. \newline 
	$\Rightarrow$ Es gibt eine \hyperref[TheorieVollstandig]{vollständige} und \hyperref[PLKonsistent]{konsistente} \hyperref[Theorie]{$\mathcal{L}$-Theorie $T_V = (\mathcal{L}, \Sigma_V)$} mit $\Sigma \subseteq \Sigma_V$ \newline
	Definition von $\Sigma_V$ durch \hyperref[Erweiterung]{Erweiterungen} $\Sigma_n$ von $\Sigma$
	
	$\Sigma_0 := \Sigma$
	
	$\Sigma_{n + 1} \left\lbrace 
		\begin{array}{l l}
			\Sigma_n & \text{falls } \Sigma_n \hyperref[Beweisbar]{\vdash} \sigma_n \\
			\Sigma_n \cup \{\neg \sigma_n\} & \text{sonst}
		\end{array}
	\right.$
	
	$\Rightarrow \Sigma_V := \bigcup\limits_{n \ge 0} \Sigma_n$
	
	\subsection{*Satz über Henkin-Erweiterungen}
	\label{HenkinErweiterung}
	Sei $T = (\mathcal{L}, \Sigma)$ \hyperref[Theorie]{Theorie} über abzählbarer Sprache $\mathcal{L}$. Es gibt eine Erweiterung $T_H = (\mathcal{L}_H, \Sigma_H)$ von T mit folgenden. \newline 
	Eigenschaften:
	
	\begin{tasks}[counter-format=(tsk[r]), label-width=4ex]
		\task $T_H$ ist eine \hyperref[Henkin]{Henkin-Theorie}
		\task $T_H$ ist eine \hyperref[Erweiterung]{konservative Erweiterung} von T
		\task $\mathcal{L}_H$ ist abzählbar
	\end{tasks}

	\subsection{Henkin-Erweiterung T-H}
	1-Schritt der \hyperref[HenkinErweiterung]{Henkin-Erweiterung} $T^* = (\mathcal{L}^*, \Sigma^*)$ ist definiert durch
	\begin{itemize}
		\item $\mathcal{L}^* = \mathcal{L} \cup \{c_{\sigma} : \sigma \hyperref[LSatz]{\mathcal{L} \text{-Satz}} \text{ der Gestalt } \sigma \equiv \exists x  \varphi\}$
		\item $\Sigma^* = \Sigma \cup \{\exists x \varphi \rightarrow \varphi[c_{\exists x \varphi}/x] : \exists x \varphi \hyperref[LSatz]{\mathcal{L} \text{-Satz}}\}$ mit $c_{\exists x \varphi}$ \hyperref[Henkin]{Henkin}-Konstante und $\exists x \varphi \rightarrow \varphi[c_{\exists x \varphi}/x]$ \hyperref[Henkin]{Henkin}-Axiome 
	\end{itemize}
	$\Rightarrow$ definiere $T_n = (\mathcal{L}_n, \Sigma_n)$ durch Induktion \newline
	\begin{tabular}{l l}
		$T_0$ & $= (\mathcal{L}_0, \Sigma_0) := T$ \\
		$T_{n + 1}$ & $= (\mathcal{L}_{n + 1}, \Sigma_{n + 1}) := (T_n)^* = ((\mathcal{L}_n)^*, (\Sigma_n)^*)$ \\
		$\Rightarrow T_H$ & $= (\mathcal{L}_H, \Sigma_H)$ mit $\mathcal{L}_H := \bigcup\limits_{n \ge 0} \mathcal{L}_n$ und $\Sigma_H := \bigcup\limits_{n \ge 0} \Sigma_n$
	\end{tabular}
	
	\subsection{Adäquatheitssatz}
	$T \hyperref[Erfullbar]{\vDash} \sigma \Leftrightarrow T \hyperref[Beweisbar]{\vdash} \sigma$
	
	\subsection{Satz über Konsistenz und Erfüllbarkeit}
	T erfüllbar $\Leftrightarrow$ T \hyperref[PLKonsistent]{konsistent}
	
	\subsection{*Kompaktheitssatz(Endlichkeitssatz)}
	\begin{tasks}[counter-format=(tsk[r]), label-width=4ex]
		\task \hyperref[Theorie]{Theorie T} erfüllbar $\Leftrightarrow$ Jede endliche Teilmenge $T_0$ erfüllbar 
		\task Satz $\sigma$ folgt aus T $\Leftrightarrow \exists$ endliches $T_0$ aus der $\sigma$ folgt
	\end{tasks}

	\subsection{Satz von Löwenheim}
	$T = (\mathcal{L}, \Sigma)$ erfüllbare \hyperref[Theorie]{Theorie}, $\mathcal{L}$ abzählbar \newline
	$\Leftrightarrow$ T besitzt ein abzählbares Modell
	
	\subsection{Satz über vollständig erfüllbare Theorien}
	T vollständige erfüllbare \hyperref[Theorie]{Theorie} \newline 
	$\Rightarrow \exists$ \hyperref[Struktur]{$\mathcal{L}$-Struktur $\mathcal{A}$} mit $T = Th(\mathcal{A}) = \{\sigma : \mathcal{A} \hyperref[Erfullbar]{\vDash} \sigma\}$
	
	\subsection{Satz für endliche Sprachen}
	$\mathcal{L}$ endliche Sprache $\Rightarrow$ Menge allgemeingültiger $\mathcal{L}$-Formeln auf zählbar 
	
	\newpage
	\section{Theorien und Modelle}
	\subsection{Deduktiver Abschluss}
	Deduktiver Abschluss C(T) einer \hyperref[Theorie]{Theorie $T = (\mathcal{L}, \Sigma)$} ist Menge aller aller Folgerungen aus T: $C(T) = \{\sigma : T \hyperref[Erfullbar]{\vDash} \sigma\}$ \newline
	$T = (\mathcal{L}, \Sigma)$ deduktiv abgeschlossen, falls $\Sigma = C(T)$
	
	\subsection{Monotonie des Deduktiven Abschlusses}
	\hyperref[Theorie]{$T = (\mathcal{L}, \Sigma), T' = (\mathcal{L}, \Sigma') \mathcal{L}$-Theorien}. Dann gilt: $\Sigma \subseteq \Sigma' \Rightarrow C(T) \subseteq C(T')$
	
	\subsection{Eigenschaften über C(T)}
	\hyperref[Theorie]{$T = (\mathcal{L}, \Sigma) \mathcal{L}$-Theorie}. Dann gilt: 
	\begin{tasks}[counter-format=(tsk[r]), label-width=4ex]
		\task $\Sigma \subseteq C(T)$
		\task C(C(T)) = C(T) (deduktiver Abschluss ist deduktiv abgeschlossen)
		\task Mod(T) = Mod(C(T))
	\end{tasks}

	\subsection{Äquivalente Theorien}
	2 \hyperref[Theorie]{$\mathcal{L}$-Theorien T und T'} sind gleich/äquivalent ($T \sim T'$) $\Leftrightarrow$ C(T) = C(T') $\Leftrightarrow (\Sigma' \subseteq C(\Sigma) \text{ und } \Sigma \subseteq C(\Sigma')) \Leftrightarrow$ Mod(T) = Mod(T')
	
	\subsection{Teiltheorie}
	$T = (\mathcal{L}, \Sigma)$, \hyperref[Theorie]{$T' = (\mathcal{L}, \Sigma')$ Theorien}. T ist eine Teiltheorie von: \newline
	T' ($T \subseteq T'$) $\Leftrightarrow \Sigma \subseteq \Sigma'$
	
	\subsection{*Elementare Theorie}
	Eine Elementare Theorie $Th(\mathcal{A})$ einer \hyperref[Struktur]{Struktur $\mathcal{A}$} ist eine \hyperref[Theorie]{$\mathcal{L}$-Theorie: $Th(\mathcal{A}) = (\mathcal{L}, \Sigma)$} mit $\Sigma = \{\sigma : \mathcal{A} \hyperref[Erfullbar]{\vDash} \sigma\}$ 
	
	Zusätzlich gilt: Für jede \hyperref[Struktur]{$\mathcal{L}$-Struktur $\mathcal{A}$} ist $Th(\mathcal{A})$ erfüllbar und vollständig
	
	\subsection{*Elementar und Delta Elementar}
	\label{Elementar}
	Klasse $\kappa$ von \hyperref[Struktur]{$\mathcal{L}$-Strukturen} ist elementar $\Leftrightarrow \exists$ \hyperref[LSatz]{$\mathcal{L}$-Satz} $\sigma: \kappa = Mod(\sigma)$ \newline
	Klasse $\kappa$ heißt $\Delta$-elementar $\Leftrightarrow$ \hyperref[Theorie]{$\mathcal{L}$-Theorie T}: $\kappa = Mod(T)$
	
	\subsection{*Eigenschaften über Delta-Elementare Strukturen}
	Klasse $\kappa$ von \hyperref[Struktur]{$\mathcal{L}$-Strukturen} \hyperref[Elementar]{$\Delta$-elementar} $\Leftrightarrow \kappa$ ist Durchschnitt von \hyperref[Elementar]{elementaren Klassen} von \hyperref[Struktur]{$\mathcal{L}$-Strukturen} \newline{}
	\newline
	Familie von \hyperref[Elementar]{$\Delta$-elementaren} Strukturklassen gegen beliebige Durchschnitte abgeschlossen. Sind Klassen $\kappa_i (i \in I)$ \hyperref[Elementar]{$\Delta$-elementar} $\Rightarrow \kappa = \bigcap\limits_{i \in I} \kappa_i$ \hyperref[Elementar]{$\Delta$-elementar}. Familie \hyperref[Elementar]{$\Delta$-elementaren Klassen} von \hyperref[Struktur]{$\mathcal{L}$-Strukturen} ist nicht gegen Kompliment abgeschlossen.
	
	\subsection{*Elementare Eigenschaften}
	Familie \hyperref[Elementar]{elementarer Klassen} von \hyperref[Struktur]{$\mathcal{L}$-Strukturen} abgeschlossen gegen:
	\begin{tasks}[counter-format=(tsk[r]), label-width=4ex]
		\task Vereinigung
		\task Durchschnitt
		\task Komplement
	\end{tasks}

	\subsection{Anzahlformeln}
	$\varphi_{\ge n} :\equiv \exists x_1, ..., \exists x_n (\bigwedge\limits_{1 \le i < j \le n} x_i \ne x_j)$ \newline
	$\varphi_{\le n} :\equiv \exists x_1, ..., \exists x_n \forall x (\bigvee\limits_{1 \le i \le n} x = x_i)$ \newline
	$\varphi_{= n} :\equiv \varphi_{\ge n} \land \varphi_{\le n}$
	
	\begin{tabular}{l l}
		$\Rightarrow$ & $\mathcal{A} \hyperref[Erfullbar]{\vDash} \varphi_{\ge n} \Leftrightarrow |A| \ge n$ \\
		& $\mathcal{A} \hyperref[Erfullbar]{\vDash} \varphi_{\le n} \Leftrightarrow |A| \le n$ \\
		& $\mathcal{A} \hyperref[Erfullbar]{\vDash} \varphi_{= n} \Leftrightarrow |A| = n$
	\end{tabular}

	\begin{tabular}{l l l}
		$\Rightarrow$ Für Klassen & $\mathcal{M}_n := \{\mathcal{A} : |A| = n\}$ = Mod($\varphi_{= n}$) & ($n \ge 1$) \\
		& $\mathcal{M}_{\ge n} := \{\mathcal{A} : |A| \ge n \}$ = Mod($\varphi_{\ge n}$) & ($n \ge 1$) \\
		& $\mathcal{M}_{\le n} := \{\mathcal{A} : |A| \le n\}$ = Mod($\varphi_{\le n}$) & ($n \ge 1$) \\
		& $\mathcal{M}_{inf} := \{\mathcal{A} : A\}$ = Mod($\varphi_{\ge n} : n \ge 1$) & \\
	\end{tabular}

	Es gilt für eine $\mathcal{L}$-Sprache $n \ge 1$, dass die Klassen $\mathcal{M}_n, \mathcal{M}_{\ge n}, \mathcal{M}_{\le n}$ \hyperref[Elementar]{elementar} sind und $\mathcal{M}_{inf}$ ist \hyperref[Elementar]{$\Delta$-Elementar}
	
	\subsection{Partielle Ordnung}
	Partielle Ordnung $\mathcal{P} = (P, <^{\mathcal{P}})$ erfüllt:
	\begin{description}
		\item[$\pi_1$] $\equiv \forall x \neg(x < x)$ Irreflexibilität
		\item[$\pi_2$] $\equiv \forall x \forall y \forall z (x < y \land y < z \rightarrow x < z)$ Transitivität
		\item[$\pi_3$] $\equiv \forall x \forall y (x < y \rightarrow \neg (y < x))$ Antisymmetrie
	\end{description}
	In linearen/totalen Ordnung gilt zusätzlich:\newline
	$\pi_4 \equiv \forall x \forall y (x < y \lor x = y \lor y < x)$ Totalität
	
	\begin{tabular}{l l}
		$\Rightarrow$ & $T_{PO} = (\mathcal{L}, \{\sigma_{PO}\})$ mit $\sigma_{PO} \equiv \pi_1 \land \pi_2 \land \pi_3$. \hyperref[Theorie]{Theorie} der partiellen Ordnung \\
		& $T_{LO} = (\mathcal{L}, \{\sigma_{LO}\})$ mit $\sigma_{LO} \equiv \pi_1 \land \pi_2 \land \pi_3 \land \pi_4$. \hyperref[Theorie]{Theorie} der linearen Ordnung \\
		$\Rightarrow$ & $PO := \{\mathcal{A}: \mathcal{A} \text{ ist partielle Ordnung}\} = Mod(T_{PO}) = Mod(\sigma_{PO})$ \\
		& $LO := \{\mathcal{A}: \mathcal{A} \text{ ist lineare Ordnung}\} = Mod(T_{LO}) = Mod(\sigma_{LO})$
	\end{tabular}
	
	$\Rightarrow$ Klassen sind \hyperref[Elementar]{elementar}

	\subsection{*Gruppen- und Körperaxiome}
	\begin{tabular}{p{7cm} l}
		$\gamma_1 \equiv \forall x \forall y \forall z ((x + y) + z = x + (y + z))$ & Assoziativität \\
		$\gamma_2 \equiv \forall x (0 + x = x)$ & 0 links neutral \\
		$\gamma_3 \equiv \forall x \exists y (y + x = 0)$ \newline $T_G = (\mathcal{L}, \{\gamma_1, \gamma_2, \gamma_3\})$ Gruppentheorie \newline $\Rightarrow$ Klassen G der Gruppen \hyperref[Elementar]{elementar} & Existenz von Links inversen \\
		$\gamma_4 \equiv \forall x \forall y (x + y = y + x)$ \newline $\Rightarrow$ G Abelsch & Kommutativität \\
	\end{tabular}

	$\Rightarrow$ G ist Abelsch \newline
	Körperaxiome:
	
	Es seien $\gamma_1, ..., \gamma_4$ Gruppenaxiome
	
	\begin{tabular}{l l}
		$\gamma_1' \equiv$ & $\forall x \forall y \forall z ((x * y) * z = x * (y * z))$ \\
		$\gamma_2' \equiv$ & $\forall x (1 * x = x)$ \\
		$\gamma_3' \equiv$ & $\forall x \exists y (x \ne 0 \rightarrow x * y = 1)$ \\
		$\gamma_4' \equiv$ & $\forall x \forall y (x * y = y * x)$ \\
		$\delta \equiv$ & $\forall x \forall y \forall z (x * (y + z) = (x * y) + (x * z))$
	\end{tabular}
	
	$T_K = (\mathcal{L}, \{\gamma_1, ..., \gamma_4, \gamma_1', ..., \gamma_4', \delta\})$\newline
	$\Rightarrow$ Klasse der Körper ist elementar
	
	\subsection{Charakteristik}
	\begin{itemize}
		\item $\kappa$ hat die Charakteristik $p \ge 1 \Leftrightarrow \underbrace{1 + ... + 1}_\text{p-mal} = 0$ (endlich)
		\item unendliche Charakteristik/Charakteristik 0 $\Leftrightarrow$ keine endliche Charakteristik
	\end{itemize}
	
	\subsection{Lemma}
	\begin{tasks}[counter-format=(tsk[r]), label-width=4ex]
		\task $p \ge 1 \Rightarrow$ Klasse $K_p$ der Körper Charakteristik p ist elementar
		\task Klasse $K_0$ ist $\Delta$-Elementar 
	\end{tasks}
	
	\subsection{Isomorphismen}
	\begin{tasks}
		\task ($\mathcal{L}$)-Isomorphismus f von $\mathcal{A}$ nach $\mathcal{B} (f: \mathcal{A} \cong \mathcal{B})$ ist eine bijektive Abbildung $f: A \rightarrow B$, die mit den ausgezeichneten Relationen, Funktionen und Konstanten verträglich ist: \begin{itemize}
			\item $(a_1, ..., a_{n_i}) \in \mathcal{R}_i^{\mathcal{A}} \Leftrightarrow (f(a_1), ..., f(a_{n_i}) \in \mathcal{R}_i^{\mathcal{B}}$
			\item $f(f_i^{\mathcal{A}}(a_1, ..., a_{m_j})) = f_i^{\mathcal{B}}(f(a_1), ..., f(a_{m_j})$
			\item $f(c_k^{\mathcal{A}}) = c_k^{\mathcal{B}}$
		\end{itemize}
		\task $\mathcal{A}$ und $\mathcal{B}$ isomorph $(\mathcal{A} \cong \mathcal{B}) \Leftrightarrow \exists$ Isomorphismus von $\mathcal{A}$ nach $\mathcal{B}$
	\end{tasks}
	
	Es gilt zusätzlich: $\mathcal{A} \cong \mathcal{B} \Rightarrow \forall$ Satz $\sigma: \mathcal{A} \vDash \sigma \Leftrightarrow \mathcal{B} \vDash \sigma$ 
	
	Weiterhin: $\mathcal{A}, \mathcal{B} \mathcal{L}$-Strukturen, $f: A \rightarrow B$ Isomorphismus, $\tilde{B}: \{x_0, ..., x_n\} \rightarrow A$ Belegung. Dann gilt für jeden $\mathcal{L}$-Term $t \equiv t(x_0, ..., x_n)$, jede $\mathcal{L}$-Formel $\varphi \equiv \varphi(x_0, ..., x_n)$: \begin{itemize}
		\item $f(t_{\tilde{\mathcal{B}}}^{\mathcal{A}}) = t_{f(\tilde{\mathcal{B}})}^{\mathcal{B}}$
		\item $W_{\tilde{\mathcal{B}}}^{\mathcal{A}}(\varphi) = W_{f(\tilde{\mathcal{B}})}^{\mathcal{B}}(\varphi)$
	\end{itemize}
	
	\subsection{!(Elementar) Äquivalent}
	$\mathcal{L}$-Strukturen $\mathcal{A}, \mathcal{B}$ elementar äquivalent $(\mathcal{A} \equiv \mathcal{B}) \Leftrightarrow Th(\mathcal{A}) = Th(\mathcal{B})$ ($\forall$-Satz: $\mathcal{A} \vDash \sigma \Leftrightarrow \mathcal{B} \vDash \sigma$)
	
	Für $\mathcal{L}$-Strukturen $\mathcal{A}, \mathcal{B}$ sind äquivalent
	\begin{tasks}[counter-format=(tsk[r]), label-width=4ex]
		\task $\mathcal{A} \equiv \mathcal{B}$
		\task $B \vDash Th(\mathcal{A})$
	\end{tasks}
	
	Klasse $\kappa$ von $\mathcal{L}$-Strukturen ($\Delta$-)elementar $\Rightarrow \kappa$ gegen elementare Äquivalenz und Isomorphie abgeschlossen 
	
	Für eine $\mathcal{L}$-Struktur $\mathcal{A}$ sind folgende Aussagen äquivalent:
	\begin{tasks}[counter-format=(tsk[r]), label-width=4ex]
		\task $\{\mathcal{B}: \mathcal{A} \cong \mathcal{B}\}$ ist $\Delta$-Elementar
		\task Jede zu $\mathcal{A}$ äquivalente Struktur $\mathcal{B}$ ist zu $\mathcal{A}$ isomorph. D.h. Für alle $\mathcal{L}$-Strukturen $\mathcal{B}$ gilt: $\mathcal{A} \equiv \mathcal{B} \Rightarrow \mathcal{A} \cong \mathcal{B}$
		\task Für alle $\mathcal{L}$-Strukturen $\mathcal{B}$ gilt: $\mathcal{A} \equiv \mathcal{B} \Leftrightarrow \mathcal{A} \cong \mathcal{B}$
		\task Für alle $\mathcal{L}$-Strukturen $\mathcal{B}$ gilt: $\mathcal{A} \equiv \mathcal{B} \Leftrightarrow \mathcal{B} \vDash Th(\mathcal{A})$
	\end{tasks}

	\subsection{Kompaktheitssatz der PL1}
	\begin{description}
		\item[a.) Kompaktheitssatz für Folgerungsbegriff] T $\mathcal{L}$-Theorie, $\sigma$ $\mathcal{L}$-Satz mit $T \vDash \sigma$, dann existiert endliche Teiltheorie $T_0 \subseteq T$ mit $T_0 \vDash \sigma$ 
		\item[b.) Kompaktheitssatz für Erfüllbarkeitsbegriff] Jede endliche Teiltheorie $T_0 \subseteq T$ mit $T_0 \vDash \sigma$ 
	\end{description}

	Klasse $\kappa$ von $\mathcal{L}$-Strukturen ist elementar $\Leftrightarrow$ Klasse $\kappa$ und Komplement $\overline{\kappa} \Delta$-elementar 
	
	\subsection{Satz über Existenz unendlicher Modelle}
	$T = (\mathcal{L}, \Sigma) \mathcal{L}$-Theorie, die für jedes $n \ge 1$ ein Modell mit mindestens n Elementen besitzt. Dann besitzt T ein unendliches Modell 
	
	\begin{tasks}
		\task Klasse $M_{fin}$ ist nicht $\Delta$-Elementar
		\task Klasse $M_{inf}$ ist nicht Elementar
	\end{tasks} 

	\subsection{Ordnungen und Wohlordnungen}
	Lineare Ordnung $O = (A, <)$ ist eine Wohlordnung $\Leftrightarrow$ besitzt keine unendlich absteigende Kette, d.h. keine Individuen $a_n \in A$ mit ... $a_2 < a_1 < a_0$ (Wie die Ordnung der natürlichen Zahlen)
	
	Die Klasse der Wohlordnungen ist nicht $\Delta$-Elementar
	
	\subsection{!Körper und deren Charakteristik}
	\begin{tasks}[counter-format=(tsk[r]), label-width=4ex]
		\task Klasse $\kappa_0$ der Körper mit der Charakteristik 0 ist nicht elementar
		\task Klasse $\kappa_{fin}$ der Körper endlicher Charakteristik ist nicht $\Delta$-elementar
	\end{tasks}

	\subsection{Satz von Skolem}
	Es gibt eine $\mathcal{L}(\le; +, *; 0, 1)$-Struktur $\mathcal{N}^*$, die zur Struktur $\mathcal{N} = (\mathbb{N}; \le; +, *; 0, 1)$ der natürlichen Zahlen elementar äquivalent ist, also $\mathcal{N} \vDash Th(\mathcal{N})$ erfüllt, aber nicht zu $\mathcal{N}$ isomorph ist.
	
	Sei im Folgenden $\mathcal{A} = (A; \le^{\mathcal{A}}; +^{\mathcal{A}}, *^{\mathcal{A}}; 0^{\mathcal{A}}, 1^{\mathcal{A}})$ ein Nichtstandardmodell der Arithmetik, d.h. ein nicht zu $\mathcal{N}$ isomorphes Modell von $Th(\mathcal{N})$ 
	
	Ist $(L; \le)$ eine lineare Ordnung, dann ist $A \subseteq L$:
	\begin{itemize}
		\item Anfangsstück von L $\Leftrightarrow \forall x, y (x \in A \cap y \le x  \rightarrow y \in A)$ 
		\item Endstück von L $\Leftrightarrow \forall x, y (x \in A \cap x \le y  \rightarrow y \in A)$ 
		\item Intervall von L $\Leftrightarrow \forall x, y, z (x, y \in A \cap x \le z \le y  \rightarrow z \in A)$ 
	\end{itemize}
	Elemente einer linearen Ordnung bezeichnen wir als Punkte.
\end{document}