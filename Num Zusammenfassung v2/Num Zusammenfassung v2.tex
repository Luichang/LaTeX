\documentclass[12pt,a4paper]{article} % using article ensures it starts at 1 and does not have odd numberings for section
\usepackage{graphicx}
\usepackage[utf8]{inputenc} % this way umlaute are included from the get go
\usepackage[ngerman]{babel} % german spell check
\usepackage{datetime}

\usepackage{breqn} % this package is one option for math lines
\usepackage{mathtools}

\usepackage{hyperref} % these two lines are so that the table of content is clickable
\hypersetup{linktoc=all}

\usepackage{amssymb} % package for Natural Number sign etc

%added commands:
\newcommand*\conj[1]{\bar{#1}}
\newcommand*\mean[1]{\bar{#1}}
\newcommand*\tab[1][1cm]{\hspace*{#1}}

\begin{document}
	\tableofcontents % creats a table of contents, ensured already that it is clickable
	\newpage % starts the actual document on a new page so there is no weird colision of text and toc
	\section{Norm und Skalarprodukt}
		\subsection{Norm}
		Definitheit: $||x|| = 0 \Rightarrow x = 0$
			
		absolute Homogenität: $||\alpha x|| = |\alpha| * ||x||$
		
		Dreiecksungleichung: $||x + y|| \le ||x|| + ||y||$
		
		\subsection{Skalarprodukt}
		<x + y, z> = <x, z> + <y, z>
		
		<x, y + z> = <x, y> + <x, z>
		
		TODO: Klammer
		
		$<\lambda x, y> = \lambda <x, y>$
		
		$<x, \lambda y> = \lambda <x, y>$
		
		<x, y> = <y, x>
		
		$<x, x> \ge 0$
		
		$<x, x> = 0 \Rightarrow x = 0$
		
		
		\subsubsection{Vom Skalarprodukt induzierte Norm}
		$||x|| = \sqrt{<x, x>}$
		
		\subsubsection{Cauchy-Schwarzche Ungleichung}
		$|<x, y>| \le ||x||*||y||$
		
		\newpage
		
		\section{Symmetrische, positiv definite Matrix}
		TODO: Matrizen
		
		insbesonders: Diagonalmatrizen, Einheitsmatrizen
		
		positiv definit: $x^t Ax > 0 $(beliebige Matrix)
		
			alle EW > 0 (symmetrische Matrix)
			
			alle Haupt[TODO: ?] > 0 (symetrische Matrix)
			
			TODO: Matrix 	$\Rightarrow$ 3 Hauptminoren[?] = det(a), det(TODO: Matrix), det(TODO: Matrix)
			
			\subsection{Cholesky-Zerlegung $A = GG^t$}
			G unter der Matrix, invertierbar (symmetrische Matrix)
			
			\subsection{[?] diagonaldominant und alle $a_{ii} \ge 0$}
			(symmetrische Matrix)
			
			\subsection{Eigenwerte}
			$det(\lambda En - A) = 0$
			
			\subsection{Eigenvektor}
			$f(v) = \lambda v$
			
		\newpage
		
		\section{Matrixnormen}
		
		\subsection{Natürliche Matrixnorm}
		
		$||A||_\infty := max_{x \ne 0} \frac{||Ax||_\infty}{||x||_\infty} = max_{||x|| = 1}||Ax||_\infty$
		
		$||A|| = 0 \Rightarrow A = 0, ||\lambda A|| = |\lambda|*||A||, ||A+B|| \le ||A|| + ||B||, ||A*B|| \le ||A|| * ||B||$
		
		\subsection{Verträglichkeit}
		$||Ax|| \le ||A|| * ||x||$
		
		\subsection{Zeilensummennorm}
		= natürliche Matrixnorm
		
		$||A||_\infty = max_{||x||_\infty = 1} ||Ax||_\infty = max_{i = 1, ..., m} \sum_{j = 1}^{n} |a_{ij}|$
		
		$A = TODO: Matrix  ||A||_\infty = max{|1| + |-2| + |-3|, |2| + |3| + |-1|} = max{6, 6} = 6$
		
		\subsection{Spaltensummennorm}
		
		$||A||_1 := max_{x \ne 0} \frac{||Ax||_1}{||x||_1} = max_{||x||_1 = 1} ||Ax||_1 = max_{j = 1, ..., n} \sum_{i = 1}^{m}|a_{ij}|$
		
		$A = TODO: Matrix  ||A||_1 = max{|1| + |2|, |-2| + |3|, |-3| + |-1|} = max{3, 5, 4} = 5$
		
		$||A^t||_1 = ||A||_\infty$
		
		\subsection{Spektralnorm}
		
		$||A||_2 := max_{||x||_2 = 1} ||Ax||_2 = max_{x \ne 0} \frac{||Ax||_2}{||x||_2} = max_{||x||_2 = 1} <Ax, Ax> = max_{||x||_2 = 1} <A^tAx, x> = max{\sqrt{|\lambda |}, \lambda * EW von A^tA}$
		
		$A = TODO: Matrix  , A^tA = TODO: Matrix    det(\mu E_n - A^tA) = 0 \Leftrightarrow \mu_{1, 2} = {16, 1}$
		
		$||A||_2 = \sqrt{max(\mu_1, \mu_2)} = \sqrt{\mu_1} = \sqrt{16} = 4$
		
		\newpage
		
		\section{Spektralradius, Konditionszahl einer Matrix}
		
		\subsection{Spektralradius $\varphi$}
		
		$\varphi(A) = max:{1 \le i \le n} |\lambda_i(A)| = spr(A)$ der betragsmäßig größte Eigenwert von A
		
		$||A|| \ge |\lambda|$ (für jede Matrixnorm, die mit einer Vektornorm verträglich ist)
		
		\subsection{Konditionszahl einer Matrix A}
		
		$cond(A) = ||A||*||A^{-1}||$
		
		\subsection{Sonderfall symmetrisch, positiv definite Matrix}
		
		$cond(A) = \frac{ \lambda_{max}}{ \lambda_{min}}$
		
		\newpage
		
		\section{Ähnlichkeitstransformation, Invarianz der Eigenwerte}
		
		y = Ax
		
		$\conj{x} = Cx, \conj{y} = Cy \tab (det C \ne 0), C \in GL$ 
		
		$y = Ax \Rightarrow C^{-1} \conj{y} = AC^{-1} \conj{x} \Rightarrow \conj{y} = CAC^{-1} \conj{x} \Rightarrow \conj{y}\conj{A}\conj{x}$
		
		$\conj{A} = CAC^{-1} \Rightarrow \conj{A} \sim A$
		
		$\lambda EW, v EV zu A$
		
		$\Rightarrow Av = C^{-1}\conj{A}Cv = \lambda v$
		
		$\Rightarrow \conj{A} und A $ haben dieselben Eigenwerte, algebraisch und geometrische Vielfalten stimmen überein (Invarianz der Eigenwerte)
		
		\subsection{Reduktionsmethoden}
		
		A duch Ähnlichkeitstransformationen 
		
		$A = A^{(0)} = T_1^{-1} A^{-1}T_1 = Q ... = T_i^{-1}A^{(i)}T_i = ...$
		
		auf Form bringen, für welche EW und EV leicht zu berechnen sind (z.B. Jordan-Normalform)
		
		\newpage
		
		\section[Gleitkommazahlen]{Gleitkommazahlen, Gleitkommagitter, Maschienengenauigkeit, Rundungsfehler}
		
		\subsection{Gleitkommazahl (normalisiert)}
		
		$b \in \mathbb{N}, b \ge 2, x \in \mathbb{R}$
		
		$x = \pm m * b^{\pm e}$
		
		Mantisse: $m = m_1b^{-1} + m_2b^{-2} + ... \in \mathbb{R}$
		
		Exponent: $e = e_{s-1}b^{s-1} + ... + e_0b^0 \in \mathbb{N}$
		
		für $x \ne 0$ eindeutig
		
		\subsection{Gleitkommagitter}
		
		A = A(b\footnote{Basis}, r\footnote{Mantissenlänge}, s\footnote{Exponentenlänge}) \tab größte Darstellbare Zahl: $(1 - b^{-r})*b^{b^s-1}$
		
		$(b = 10): 0,314 * 10^1 = 3,14$
		
		$0,123 * 10^6 = 123.000$
		
		Beispiel: konvertiere von Basis 8 zu Basis 10:
		
		$x = (0,5731 * 10^5)_8 \in A(8, 5, 1)$
		
		$x = (5 * 8^{-1} + 7 * 8^{-2} + 3 * 8^{-3} + 1 * 8^{-4}) * 8^5$
		
		$x = 5 * 8^4 + 7 * 8^3 + 3 * 8^2 + 1 * 8^1 = 24.264 * 10^0$
		
		\subsection{Maschienengenauigkeit eps}
		
		$eps = \frac{1}{2}b^{-r + 1}, IEEE: eps = \frac{1}{2} * 2^{-52} \approx 10^{-16}$
		
		\subsection{Rundungsfehler}
		
		$absolut: |x - rd(x)| \le \frac{1}{2}b^{-r}b^e$
		
		$relativ: |\frac{x - rd(x)}{x}| \le \frac{1}{2}b^{-r+1} = eps$
		
		\newpage
		
		\section{Darstellung des Interpolationsfehlers}
		
		\subsection{Fehler I}
		
		$f \in C^{n+1}[a, b], \forall x \in [a, b] \exists \xi_x \in (\overline{x_0, ..., x_n, x})\footnote{kleinstes Intervall, das alle x_i enthällt}, s.d.$
		
		$f(x) - p(x) = \frac{f^{(n+1)}(\xi x)}{(n+1)!} \prod_{j = 0}^{n}(x - x_j)$
		
		\subsection{Fehler II}
		
		$f \in C^{n + 1}[a, b], \forall x \in [a, b]$ \textbackslash ${x_0, ..., x_n} gilt:$
		
		$f(x) - p(x) = f[x_0, ..., x_n, x] \prod_{j = 0}^{n}(x - x_j)$
		
		mit $f[x_i, ..., x_{i + k}] = y[x_i, ..., x_{i + k}]$
		
		und $f[x_0, ..., x_n, x] = \int_{0}^{1}\int_{0}^{t_1}...\int_{0}^{t_n}f^{n+1}(x_0 + t_1(x_1 - x_0) + ... + t_n(x_n-x_{n - 1} + t(x - x_n))dtdt_n...dt_1$ 
		
		für $x_0 = x_1 = ... = x_n: $
		
		$f[x_0, ..., x_n] = \frac{1}{n!}f^{(n)}(x_0)$
		
		$\frac{f^{(n + 1)}(\xi_x)}{(n + 1)!} \prod_{j = 0}^{n} (x-x_j) = f(x) - p(x) = f[x_0, ..., x_n, x] \prod_{j = 0}^{n}(x - x_j)$
		
		$\Rightarrow f[x_0, ..., x_n, x] = \frac{f^{(n + 1)}(\xi_x)}{(n + 1)!}$
		
		\newpage
		
		\section{Konditionierung einer numerischen Aufgabe, Konditionszahlen $k_{ij}$}
		
		\subsection{numerische Aufgabe}
		
		$x_j \in \mathbb{R} mit f(x_1, ..., x_m) \Rightarrow y_i = f_i(x_j)$
		
		fehlerhafte Eingangsgrößen $x_i + \Delta y_i$
		
		$|\Delta y_i|$ ist der absolute Fehler, $|\frac{\Delta y_i}{y_i}|$ ist der relative Fehler
		
		\subsection{Konditionszahl (relativ)}
		
		$k_{ij}(x) = \frac{\partial f_i}{\partial x_i}(x) \frac{\Delta x_j}{x_j}$
		
		$\frac{\Delta y_i}{y_i} = \sum_{j = 1}^{m}k_{ij}(x)\frac{\Delta x_j}{x_j}$
		
		$|k_{ij}(x)| >> 1 \Rightarrow$ schlecht konditioniert
		
		$|k_{ij}(x)| << 1 \Rightarrow$ gut konditioniert, ohne Fehlerverstärkung
		
		$|k_{ij}(x)| > 1 \Rightarrow$ Fehlerverstärkung
		
		$|k_{ij}(x)| < 1 \Rightarrow$ Fehlerdämpfung
		
		\newpage
		
		\section{Stabilität eines Algorithmus}
		
		\subsection{stabiler Algorithmus}
		
		akkumulierte Fehler der Rechnung (Rundungsfehler, Auswertungsfehler, etc.) übersteigen den unvermeidbaren Problemfehler der Konditionierung der Aufgabe nicht. Aka Trotz Ungenauigkeiten bei den Eingabe Variablen erhalten wir fast sehr genaue Ergebnisse.
		
		\newpage
		
		\section{Auslöschung}
		
		Verlust von Genauigkeit bei der Subtraktion von Zahlen mit gleichem Vorzeichen
		
		TODO: bei bedarf ein Beispiel
		
		\newpage
		
		\section{Horner-Schema*}
		
		$p(x) = a_0 + x(... + x(a_{n-1} + a_nx)...)$
		
		\subsection{Code}
		
		def horner(Ac\footnote{Vektor mit Koeffizienten}, Ax\footnote{Stützstellen}, n\footnote{Anzahl der Stützstellen}, x\footnote{Auswertungspunkt}): // \footnote{wobei Ac und Ax np Arrays sind, n ein int und x ein double}
		
		y = 0.0
		
		for i in reversed range(n):
		
		\tab y = y * (x - Ax[i]) + Ac[i]
		  
		return y
		
		Immer Horner-Schema zur Auswertung von Polynomen verwenden.
		
		\subsection{Auswertung}
		TODO: subsection
		
		\newpage
		
		\section{Interpolation und Approximation}
		
		\subsection{Grundproblem}
		
		Darstellung und Auswertung von Funktionen
		
		\subsection{Aufgabenstellung}
		
		f(x) nur auf Diskreter Menge von Argumenten $x_0, ..., x_n$ bekannt und soll rekonstruiert werden
		
		analytisch gegebene Funktion soll auf Reelwerte dargestellt werden, damit jederzeit Werte zu beliebigen x berechnet werden können.
		
		Einfach konstruierte Funktionen in Klassen P:
		
		Polynome: $p(x) = a_0 + a_1x + a_2x^2 + ... + a_nx^n$
		
		rationale Funktion: $r(x) = \frac{a_0 + a_1x + ... + a_nx^n}{b_0 + b_1x + ... + b_mx^m}$
		
		trigonometrische Funktion: $t(x) = \frac{1}{2}a_0 + \sum_{k = 1}^{n}(a_kcos(kx) + b_ksin(kx))$
		
		Exponentialsummen: $e(x) = \sum_{k = 1}^{n}a_kexp(b_kx)$
		
		\subsection{Interpolation}
		
		Zuordnung von $g \in P$ zu f durch Fixieren von Funktionswerten
		
		$g(x_i) = y_i = f(x_i), i = 0, ..., n$
		
		\subsection{Approximation}
		
		$g \in P$ beste Darstellung, z.B. 
		
		$max_{a \le x \le b}|f(x) - g(x)| minimal$
		
		$(\int_{a}^{b}|f(x) - g(x)|^2dx)^{\frac{1}{2}} minimal$
		
		\newpage
		
		\section{Lagransche Interpolationsaufgabe}
		
		\subsection{Aufgabe}
		
		Finde zu n + 1 verschiedene Stützstellen/Knoten $x_0, ..., x_n \in \mathbb{R} und WErten y_0, ..., y_n \in \mathbb{R} ein Polynom p \in P_n mit p(x_i) = y_i$
		
		\subsection{Eindeutigkeit + Existenz}
		
		Die Lagransche Interpolationsaufgabe ist eindeutig lösbar
		
		TODO: bei bedarf Beweis rein kopieren den Ich nicht verstanden hab
		
		\subsection{Lagransche Basispolynome}
		
		$L_i^{(n)}(x) = \prod_{j = 0, j \ne i}^{n} \frac{x - x_j}{x_i - x_j} \in P_n, i = 0, ..., n$
		
		\subsection{Eigenschaften}
		
		ortogonal: es gilt $L_i^{(n)}(x_k) = d_{ik} =$ TODO: split over 2 lines    1, i = k 0, sonst
		
		bilden Basis von $P_n$
		
		haben Grad n
		
		\subsection{Lagransche Darstellung}
		
		$p(x) = \sum_{i = 0}^{n}y_iL_i^{(n)}(x) \in P_n$ mit $p(x_j) = y_j$
		
		Nachteil: Bei Hinzunahme von $(x_{n+1}, y_{n+1})$ ändert sich das Basispolynom komplett
		
		
		TODO: Beispiel
		
		\newpage
		
		\section{Newtonsche Basispolynome, dividierte Differenzen}
		
		\subsection{Newton-Polynome}
		
		$N_0(x) = 1, N_i(x) = \prod_{j = 0}^{i - 1}(x - x_j)$ mit $p(x) = \sum_{i = 0}^{n}a_iN_i(x)$
		
		\subsubsection{Auswertung}
		
		$y_0 = p(x_0) = a_0$
		
		$y_1 = p(x_1) = a_0 + a_1 * (x_1 - x_0)$
		
		$\vdots$
		
		$y_n = p(x_n) = a_0 + a_1(x_1 - x_0) + ... + a_n(x_n - x_0) * ... * (x_n - x_{n - 1})$
		
		\subsubsection{Vorteil}
		
		Bei Hinzunahme von $(x_{n + 1}, y_{n + 1})$ muss nur eine neue Rechnung durchgeführt werden, und nicht das gesamte Polynom neu berechnet werden
		
		TODO: Beispiel
		
		\subsection[Newtonsche Darstellung]{Newtonsche Darstellung(stabile Variante)}
		
		$p(x) = \sum_{i = 0}^{n}y[x_0, ..., x_i] N_i(x)$
		
		\subsection{Dividierte Differenzen*}
		
		$y[x_i, ..., x_{k + 1}] = \frac{y[x_{i + 1}, ..., x_{k + 1}] - y[x_i, ..., x_{i + k - 1}]}{x_{i + k} - x_i}$ mit k = 1, ..., j und i = k - j
		
		für beliebige [?] $\sigma:{0, ..., n} \rightarrow {0, ..., n}$ gilt $y[\tilde{x_0}, ..., \tilde{x_n}] = y[x_0, ..., x_n]$
		
		\newpage
		
		\section{Nevillsche Darstellung}
		
		$p_{jj}(x) = y_j \tab j = 0, ..., n \tab k = 1, ..., j \tab i = k - j$
		
		$p_{i, i + k}(x) = p_{i, i + k - 1}(x) + (x - x_i)\frac{p_{i + 1, i + k}(x) - p_{i, i + k - 1}(x)}{x_{i + k} - x_i}$
		
		\subsection{Schema}
		
		\begin{aligant*}
			x && k = 0 &&                 && k = 2    && {}\ldots{} && k = n - 1    && k = n    & \\
			x_0 && y_0   && \longrightarrow && p_{0, 1} && {}\ldots{} && p_{0, n - 1} && p_{0, n} & \\
		\end{aligant*}
\end{document}