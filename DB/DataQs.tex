\documentclass[12pt,a4paper]{article} % using article ensures it starts at 1 and does not have odd numberings for section

\usepackage[a4paper, bindingoffset=0.5cm, left=0.25cm, right=1cm, top=4cm, bottom=3cm, footskip=0.5cm]{geometry}
\usepackage{setspace}

\usepackage{graphicx}
\usepackage[utf8]{inputenc} % this way umlaute are included from the get go
\usepackage[T1]{fontenc}
\usepackage[ngerman]{babel} % german spell check
%\usepackage{lmodern}
\usepackage{datetime}

\usepackage{amsmath} % this package is one option for math lines
\usepackage{enumerate}
\usepackage{enumitem}

\usepackage{hyperref} % these two lines are so that the table of content is clickable
\usepackage{amssymb} % package for Natural Number sign etc
\usepackage{wasysym}
\hypersetup{linktoc=all}

\usepackage{tikz}
\usetikzlibrary{shapes}
\usepackage{tikz-er2}
\usetikzlibrary{automata, positioning}
\tikzstyle{bplus}=[rectangle split, rectangle split horizontal,rectangle split ignore empty parts, rectangle split parts=5,draw]

\tikzstyle{every node}=[bplus]
\tikzstyle{level 1}=[sibling distance=45mm]
\tikzstyle{level 2}=[sibling distance=20mm]

\tikzstyle{every edge} = [link]

\begin{document}

\section{musikDB}
\begin{tabular}{| l | l || l | l |}
	\hline
	\multicolumn{4}{| c |}{AREA\_TYPE} \\ \hline
	\underline{id} & name & \underline{id} & name \\ \hline 
	1 & Country & 5 & District \\ 
	2 & Subdivision & 6 & Island \\
	3 & City & 7 & County  \\
	4 & Municipality & & \\ \hline
\end{tabular}
\begin{tabular}{| l | l | l |}
	\hline
	\multicolumn{3}{| c |}{AREA} \\ \hline
	\underline{id} & name & type \\ \hline 
	432 & England & 1 \\
	1178 & London & 3 \\
	81 & Germany & 1 \\
	334 & Nordrhein-Westfahlen & 2 \\
	7709 & Düsseldorf & 3 \\ \hline
\end{tabular}
\begin{tabular}{| l | l | l |}
	\hline
	\multicolumn{3}{| c |}{AREA\_IN} \\ \hline
	& Contains & Within other \\ \hline 
	\underline{id} & entity0 & entity1 \\ \hline 
	964 & 432 & 1178 \\
	3587 & 432 & 3821 \\
	71 & 81 & 334 \\
	7475 & 334 & 7709 \\ \hline
\end{tabular}
\newline
\begin{tabular}{| l | l | l | l |}
	\hline
	\multicolumn{4}{| c |}{PLACE} \\ \hline
	\underline{id} & name & type & area \\ \hline 
	2418 & The Crystal Palace & 3 & 1178 \\
	2387 & BBC Radio Theatre & 1 & 1178 \\
	789 & ESPRIT area & 4 & 7709 \\ \hline
\end{tabular}
\newline
\begin{tabular}{| l | l | l | l | c | c |}
	\hline
	\multicolumn{6}{| c |}{ARTIST\_ARTIST} \\ \hline
	& Relationship & Person & Band &  & \\ \hline 
	\underline{id} & r\_type & entity0 & entity1 & begin\_year & end\_year \\ \hline 
	32164 & member of band & 285850 & 1352 & 1974 & 1980 \\
	164036 & member of band & 290121 & 1352 & 1973 & \\
	54623 & sibling & 290121 & 316942 & & \\ \hline
\end{tabular}
\begin{minipage}{4.5cm}
	\noindent\fbox{\parbox{4cm}{
			\underline{r\_type possibilities}
			\begin{itemize}[noitemsep]
				\item is person
				\item sibling
				\item married 
				\item member of band
				\item parent 
				\item founder
			\end{itemize}
	}}
\end{minipage}
\newline
\begin{tabular}{| l | l | c | c | c | c | c | l | l | l |}
	\hline
	\multicolumn{10}{| c |}{ARTIST} \\ \hline
	&  & birth &  &  & death &  &  &  & \\ \hline
	\underline{id} & name & b\_year & b\_month & b\_day & e\_year & type & area & gender & b\_area \\ \hline 
	938 & Eric Clapton & 1945 & 3 & 30 &  & p & 221 & m & 53564 \\
	1352 & AC/DC & 1973 & 11 & & & g & 13 & & 5126 \\
	290121 & Angus Young & 1955 & 3 & 31 & & p & 80658 & m & 3855 \\
	341872 & Doro Pesch & 1964 & 6 & 3 & & p & 81 & f & 7709 \\
	285850 & Bon Scott & 1946 & 7 & 9 & 1980 & p & 221 & m & 80661 \\ \hline
\end{tabular}
\newline
\begin{tabular}{| l | l | l | l | c | c | c |}
	\hline
	\multicolumn{7}{| c |}{RELEASE} \\ \hline
	\underline{release\_id} & name & artist\_credit & country & date\_year & date\_month & date\_day \\ \hline 
	1610822 & Back in Black & 1352 & 81 & 1986 & & \\
	1085503 & Back in Black & 1352 & 13 & 1989 & & \\
	1386549 & Atemlos durch die Nacht & 397683 & 81 & 2013 & 11 & 29 \\ \hline
\end{tabular}
\newline
\begin{tabular}{| l | l | l | l | l | l |}
	\hline
	\multicolumn{6}{| c |}{TRACK} \\ \hline
	\underline{id} & release\_id & name & position & lenght & artist\_credit \\ \hline 
	14242945 & 1085503 & Hells Bells & 1 & 312000 & 1352 \\
	18709752 & 1610822 & Hells Bells & 1 & 311866 & 1352 \\
	18709759 & 1610822 & Have a Drink on Me & 8 & 238400 & 1352 \\
	20084778 & 1722097 & I Lost My Heart in Heidelberg & 18 & 198040 & 35001 \\ \hline
\end{tabular}


\newpage
\section{SQL}

\noindent\fbox{\parbox{0.8\textwidth}{
	\underline{EXISTS}
	
\begin{minipage}{0.6\textwidth}
	\begin{tabular}{l}
		SELECT artist.name FROM artist \\
		WHERE EXISTS (SELECT track.name FROM track \\
		WHERE track.artist\_credit = artist.id \\
		AND track.lenght BETWEEN 7200000 AND 14400000 \\
		AND artist.type = 'g')
	\end{tabular}
\end{minipage}
\begin{minipage}{0.07\textwidth}
	\begin{tabular}{|l|l|}
		\hline
		& name \\ \hline
		1 & Exodus \\ \hline 
	\end{tabular}
\end{minipage}}}

\noindent\fbox{\parbox{0.8\textwidth}{
		\underline{String Functions}
		Im Umgang mit Wörtern ist es am sinnvollsten wenn man \textbf{LIKE} statt = verwendet, da wenn man so was wie '\%a\%' haben möchte sucht = nach genau dieser Zeichenfolge und LIKE nach einem a mit sämtlichen Zeichen vor und nach einem a. \\
		\textbf{UPPER} und \textbf{LOWER} sollten selbsterklärend sein \\
		Die Zeichen \% und \_ sind jeweils 'Wildcards' sprich da kann jeder Buchstabe hin. \% ist für 0 bis $\infty$ Charaktere und \_ für genau ein Charakter
}}

\noindent\fbox{\parbox{0.9\textwidth}{
		\underline{HAVING + ORDER BY + GROUP BY}
		
		\begin{minipage}{0.6\textwidth}
			\begin{tabular}{l}
				SELECT area.name, COUNT(artist.name) FROM artist, area \\
				WHERE artist.b\_area = area.id \\
				GROUP BY area.name \\
				HAVING COUNT(artist.name) > 1000 \\
				ORDER BY area.name ASC
			\end{tabular}
		\end{minipage}
		\begin{minipage}{0.07\textwidth}
			$\left.
			\begin{tabular}{|l|l|c|}
				\hline
				& \multicolumn{2}{|r|}{$\downarrow$ GROUP BY}  \\ \hline
				& name & count \\ \hline 
				1 & London & 2094 \\ \hline 
				2 & Los Angeles & 1319 \\ \hline 
				3 & New York & 1403 \\ \hline 
				4 & Tokyo & 1157 \\ \hline 
			\end{tabular}
			\right\Downarrow\rotatebox[origin=c]{90}{ORDER BY ASC}$
		\end{minipage}}
	}

\noindent\fbox{\parbox{0.9\textwidth}{
		\underline{WITH AS + LIMIT}
		
		\begin{minipage}{0.5\textwidth}
			\begin{tabular}{l}
				WITH tmp AS ( \\
				SELECT artist.name as artist, r.name as album \\
				FROM artist, release as r \\
				WHERE r.artist\_credit = artist.id ) \\
				SELECT * FROM tmp \\
				LIMIT 1 \\
			\end{tabular}
		\end{minipage}
		\begin{minipage}{0.07\textwidth}
			\begin{tabular}{|l|l|l|l|}
				\hline
				& artist & album \\ \hline 
				1 & Damn Seagulls & Soul Politics \\ \hline 
			\end{tabular}
\end{minipage}}}

\subsection{DDL/DML}
create Table Unikurs( \\
KursID int, \\
Name varchar(30) not null, \\
AnzahlStudis int check (AnzahlStudis >=5), \\
primary key(KursID) \\
foreign key(Name) references Vorlesung(Name)\\
);

insert into unikurs \\
values(0, 'LA1', 777)

insert into unikurs (KursID, Name) \\
values(1, 'Ana1')

\section{ER-Schema}

\begin{tikzpicture}[node distance=3cm]
\node[entity] (sp) {Spieler};
\node[attribute] (name) [above of=sp] {Name} edge (sp);
\node[attribute] (id) [above right of=sp] {\underline{id}} edge (sp);

\node[relationship] (in) [right=3cm of sp] {spielt in} edge node {[11, 11]} (sp);

\node[entity](mn) [right=3cm of in] {Mannschaft} edge node {[1, 1]} (in);
\end{tikzpicture}


\section{Relationale Algebra}
\underline{Umbenennung $\beta$}

Ganze Tabelle: $\beta_b$(Band). Einzelne Spalte: $\beta_{ph \leftarrow Photo}$(R)\\
\underline{Gruppierung}

$\gamma_{A, sum(a)}$(R) Gruppiert nach A mit der Summe über a (oder andere Aggregat Funktionen)\\
\underline{min/max}

min: $\varPi_a$(R) - ($\beta_{a \leftarrow a2}(\varPi_{a2}(\sigma_{a1 < a2}(\beta_{a1 \leftarrow a}(R) \times \beta_{a2 \leftarrow a}(R))))$) $\Leftrightarrow \gamma_{A, min(a)}$(R) - Man selektiert alle Elemente die noch ein kleineres Element haben und zieht dann alle anderen ab

max: $\varPi_a$(R) - ($\beta_{a \leftarrow a1}(\varPi_{a2}(\sigma_{a1 < a2}(\beta_{a1 \leftarrow a}(R) \times \beta_{a2 \leftarrow a}(R))))$) $\Leftrightarrow \gamma_{A, max(a)}$(R) - Man selektiert alle Elemente die noch ein größeres Element haben und zieht dann alle anderen ab

\underline{Von SQL zu Relationale Algebra}

SELECT DISTINCT artist.name, \\
artist.b\_year \\
FROM area, release, artist \\
WHERE artist.type = 'g' \\ 
AND area.name = artist.name \\
AND area.type = 1; \\
$\varPi_{artist.name, artist.b\_year}(\sigma_{artist.type=’g’ \land area.name=artist.name \land area.type=1 \land artist.id=release.artist_credit \land release.date_year>=2010})$

$(AREA \times RELEASE \times ARTIST))$

\underline{Operator Baum}

\begin{tikzpicture}
	\node {$\varPi_{artist.name, artist.b\_year}$}
	child {node [text width=6cm]{$\sigma_{artist.type=’g’ \land area.name=artist.name \land}$ ${}_{area.type=1 \land artist.id=release.artist\_credit \land}$ $ {}_{release.date\_year>=2010}$}
	child {node {$\times$}
	child {node {$\times$}
	child {node {Area}}
	child {node {Release}}}
	child {node {Artist}}}};
\end{tikzpicture}
\begin{tikzpicture}%[node distance=1cm]
	\node {$\varPi_{artist.name, artist.b\_year}$}
	child {node {$\sigma_{artist.type=’g’}$}
	child {node {$\sigma_{area.name=artist.name}$}
	child {node {$\sigma_{area.type=1}$}
	child {node {$\sigma_{artist.id=release.artist\_credit}$}
	child {node {$\sigma_{release.date\_year>=2010}$}
	child {node {$\times$}
	child {node {$\times$}
	child {node {Area}}
	child {node {Release}}}
	child {node {Artist}}}}}}}};
\end{tikzpicture}
\begin{tikzpicture}[level 2/.style={sibling distance=3cm}]
\node {$\varPi_{artist.name, artist.b\_year}$}
child {node {$\sigma_{artist.type=’g’}$}
	child {node {$\sigma_{area.type=1}$}
		child {node {$\sigma_{release.date\_year>=2010}$}
			child {node {$\bowtie_{artist.id=release.artist\_credit}$}
				child {node {$\bowtie_{area.name=artist.name}$}
					child {node {Area}}
					child {node {Release}}}
				child {node {Artist}}}}}};
\end{tikzpicture}

\begin{tikzpicture}[level 2/.style={sibling distance=4.4cm},level 3/.style={sibling distance=3cm}]
\node {$\varPi_{artist.name, artist.b\_year}$}
	child {node {$\bowtie_{artist.id=release.artist\_credit}$}
		child {node {$\bowtie_{area.name=artist.name}$}
			child {node {$\sigma_{area.type=1}$}
			child {node {Area}}}
			child {node {$\sigma_{artist.type=’g’}$}
				child {node {Artist}}}}
	child {node {$\sigma_{release.date\_year>=2010}$}
		child {node {Release}}}};
\end{tikzpicture}
\begin{tikzpicture}[level 2/.style={sibling distance=5cm},level 3/.style={sibling distance=3cm}]
\node {$\varPi_{artist.name, artist.b\_year}$}
child {node {$\bowtie_{artist.id=release.artist\_credit}$}
	child {node {$\bowtie_{area.name=artist.name}$}
		child {node {$\varPi_{name}$}
			child {node {$\sigma_{area.type=1}$}
				child {node {Area}}}}
		child {node {$\varPi_{name, b\_year, id}$}
			child {node {$\sigma_{artist.type=’g’}$}
				child {node {Artist}}}}}
	child {node {$\varPi_{artist\_credit}$}
		child {node {$\sigma_{release.date\_year>=2010}$}
			child {node {Release}}}}};
\end{tikzpicture}

\section{Tupelkalküle}
Name und Wohnort aller Mitarbeiter, welche in der Forschung arbeiten \\
\{<m.name, m.wohnort> | m $\in$ Mitarbeiter $\land \exists a \in$ Abteilung (a.abtname = 'Forschung' $\land$ a.nr = m.abtnr)\}

\section{Formeln}
\subsection{Selectivität}
Erwartete größe der resultierenden Tabelle: \\
sel(P, R) = $\frac{|\sigma_P(R)|}{|R|}$, sel(B $\bowtie$ C) = $\frac{|B \bowtie C|}{|B| * |C|}$, sel = $\frac{\text{Output - Relation}}{\text{Input - Relation}}$

Erwartete größe der Suche: \\
sel(A = v, R) = $\frac{1}{val_{A, r}}$, sel(A < v, R) = $\frac{v - A_{min}}{A_{max} - A_{min}}$, sel(A > v, R) = $\frac{A_{max} - v}{A_{max} - A_{min}}$

Projektion ohne Duplikateliminierung: |r| = |$\varPi$(r)| $\Rightarrow$ sel = 1
\\
Mit Duplikateliminierung: |$\varPi_k$(r)| = $val_{k, r}$
\\
Spezialfälle sind: Keine gemeinsamen Werte |r $\bowtie$ s| = 0, Fremdschlüsselbeziehung (r $\rightarrow$ s): |r $\bowtie$ s| = |r|, Im Verbundsattribut ist nur ein verschiedenes Element: |r $\bowtie$ s| = |r * s|, sonst |r $\bowtie$ s| = $\frac{|r| * |s|}{\max\{val_{B, r}, val_{B, s}\}}$

\subsection{B+Bäume}
d := Daten, k := (Einordnungs-)Attribut, s := Seite (=Knoten), z := Zeiger, n := Anzahl Datensätze, x := Zellen inneres Knoten, y := Zellen Blattknotens\\ \\
\underline{Ordnung des Blattknoten i}

i = $\lfloor\frac{s - 2z}{2(d + k)}\rfloor$\\ \\
\underline{Ordnung des Innerknotens j}

j = $\lfloor\frac{s - z}{2(d + k)}\rfloor$ \\ \\ 
\underline{Höhe}

$1 + \lceil \log_{2x + 1}(\frac{n}{2y}) \rceil \le n \le 1 + \lceil \log_{2x + 1}(\frac{n}{y}) \rceil$

\section{Anfrage Optimierung}
\underline{Variablen}

\begin{tabular}{r l}
	$b_{size}$ & Blockgröße \\
	mem & Puffergröße (in Anzahl Blöcken) \\
	|r| & Anzahl Tupel in Relation r \\
	$b_r$ & Anzahl Blöcke für Tupel aus r \\
	$size_r$ & Mittlerer Tupel-Größe \\
	$f_r$ & Blockungsfaktor $\frac{b_{size}}{size_r}$ - Wie viele Relationen in einen Block passen \\
	$val_{A, r}$ & Anzahl verschiedener Werte für A in r \\
	$lev_{I(R(A))}$ & Anzahl Indexebenen im $B^+$Baum für Index auf R(A)
\end{tabular}

\underline{Externes Sortieren}

Anzahl Mischläufe: $\lceil \log_{mem - 1}(\frac{b_r}{mem}) \rceil$

Blockzugriff pro Mischlauf: $(L + S) * b_r * (1 + \lceil \log_{mem - 1}(\frac{b_r}{mem}) \rceil)$

Bottleneck bei Anfragebearbeitung sind immer Blockzugriffe. Operationen im Hauptspeicher werden vernachlässigt.

\underline{Scans}

Scans durchlaufen alle Tupel/Sucht nach einem bestimmten Tupel mit den entsprechenden Bedingungen

Kosten eines Relationsscans: $b_r$

Kosten eines Indexscans: $lev_{I(R(A))} + \frac{|r|}{val_{A, r}}$

\underline{Join-Methods}

Block-Nested-Loop-Join: liest so viele Blöcke von r ein, dass der
Buffer fast voll ist (mem-1). In dem letzten Buffer Slot wird Sequenziell jeder Block von S eingelesen. Dies wird mit r gejoined in den Hauptspeicher wieder gehashed. Kosten sind \[b_r + \lceil\frac{b_r}{mem-1}\rceil * b_s\]

Merge-Join: Zunächst werden beide Tabellen nach Join-Attribut sortiert. Dann werden beide linear durchlaufen und gemerged. Kosten sind \[[sortieren] + b_r + b_s \text{ wobei das Sortieren abgeschätzt werden kann mit } b_x \log_{mem}b_x\]

Classic-Hash-Join: Die ganze Relation r wird durchgegangen und ein block nach dem anderen gehashed. Nach jedem hash schritt wird überprüft, ob ein block aus s zusammen gesetzt werden kann, mit der selben hash Funktion. Kosten: \[b_r + p * b_s \text{ mit } p = \lceil\frac{b_r}{mem-1}\rceil\]

\section{Codd'schen Regeln}
\begin{enumerate}
	\item Integration: Einheitliche nicht redundante Datenverwaltung
	\item Operationen: Speicher, Suchen, Ändern
	\item Katalog: Zugriff auf Datenbankbeschreibung im Data Dictionary
	\item Benutzerschichten
	\item Integritätssicherung: Korrektur des Datenbankinhaltes
	\item Datenschutz: Ausschluss unautorisierter Zugriffe
	\item Synchronisation: Parallele Transaktionen Koordinierung 
	\item Datensicherung: Wiederherstellung von Daten nach Systemfehlern
\end{enumerate}

\section{B+ Bäume} 
Eingabe = Dallas, Miami, Memphis, Atlanta, Phoenix, Portland, Detroit, Austin, Columbus, Houston, Chicago, Philadelphia \\
m = 1
\begin{center}
	\begin{tikzpicture}[level 1/.style={sibling distance=5cm}, level 2/.style={sibling distance=4cm},level 3/.style={sibling distance=3cm}]
	\node {Ho}
		child {node {Ch \nodepart{two} De}
			child {node {Au}
				child {node {Atlanta}}
				child {node {Austin}}}
			child {node {Da}
				child {node {Chicago \nodepart{two} Columbus}}
				child {node {Dallas}}}
			child {node {Detroit}}}
		child {node {Mi}
			child {node {Houston \nodepart{two} Memphis}}
			child {node {Pho \nodepart{two} Po}
				child {node {Miami \nodepart{two} Philidelphia}}
				child {node {Phoenix}}
				child {node {Portland}}}}
			
	;
	\end{tikzpicture}
	

\end{center}
\end{document}